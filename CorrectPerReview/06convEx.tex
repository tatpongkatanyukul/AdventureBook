\section{แบบฝึกหัด}

\begin{Parallel}[c]{0.49\textwidth}{0.49\textwidth}
	\selectlanguage{english}
	\ParallelLText{
		``Success is not final, failure is not fatal, it is the courage to continue that counts.''
		\begin{flushright}
			---Winston Churchill
		\end{flushright}
	}
	\selectlanguage{thai}
	\ParallelRText{
		``ความสำเร็จไม่ใช่สิ้นสุด
		ความล้มเหลวไม่ใช่จุดจบ
		มีเพียงความกล้าหาญที่ไปต่อเท่านั้นที่สำคัญ.''
		\begin{flushright}
			---วินสตัน เชอร์ชิล
		\end{flushright}
	}
\end{Parallel}
\index{english}{words of wisdom!Winston Churchill}
\index{english}{quote!courage}


\begin{Exercise}
\label{ex: convolution}

จงตอบคำถามต่อไปนี้ เกี่ยวกับชั้นคอนโวลูชั่น ลำดับชั้น และชุดมิติต่าง ๆ 
\begin{itemize}
\item (ก) อินพุตเป็นเวกเตอร์ นั่นคือ $\bm{x} \in \mathbb{R}^{10}$
และชั้นคอนโวลูชั่นใช้ฟิลเตอร์ $\bm{w}$ มีขนาด $3$ จำนวน $15$ ตัว โดยไม่มีการเติมเต็ม
ขนาดย่างก้าวเป็น $1$ แล้วผลลัพธ์จากชั้นคอนโวลูชั่น จะเป็นเทนเซอร์ขนาดเท่าใด?
\textit{คำใบ้} 
ดูสมการ~\ref{eq: deep conv filter x of D}
(สำหรับฟิลเตอร์แต่ละตัว เอาต์พุต $a_k = b + \sum_j w_j \cdot x_{k+j-1}$ โดย $k = 1, \ldots, H - H_F + 1$).
% เฉลย
% 15 x 8
สังเกต รูปแบบของเทนเซอร์ที่ใช้ คือ ชุดมิติแรกเป็นจำนวนลักษณะสำคัญ และตามด้วยชุดมิติอื่น ๆ (เช่น ชุดมิติลำดับ).
 
\item (ข) อินพุตเป็นเทนเซอร์สองลำดับชั้น คือ $\bm{X} \in \mathbb{R}^{8 \times 10}$.
ชั้นคอนโวลูชั่นใช้ฟิลเตอร์ $\bm{W}$ ขนาด $8 \times 3$ จำนวน $15$ ตัว
โดยทำคอนโวลูชั่น (การเชื่อมต่อท้องถิ่นและใช้ค่าน้ำหนักร่วม) เฉพาะกับชุดมิติที่สอง (ชุดมิติแรกเป็นเสมือนช่องลักษณะสำคัญที่ไม่มีความสัมพันธ์ในเชิงลำดับ).
ไม่มีการเติมเต็มอินพุต และใช้ขนาดย่างก้าวเป็น $1$.
ผลลัพธ์จากชั้นคอนโวลูชั่น จะเป็นเทนเซอร์ขนาดเท่าใด?
\textit{คำใบ้}
สำหรับฟิลเตอร์แต่ละตัว
เอาต์พุต $a_k = b + \sum_c \sum_j w_{c, j} \cdot x_{c, k+j-1}$ 
โดย $c$ แทนดัชนีของช่องลักษณะสำคัญ (ไม่มีความสัมพันธ์ในเชิงลำดับ)
และ $k = 1, \ldots, H - H_F + 1$.
% answer 15 x 8

\item (ค) อินพุตเป็นเทนเซอร์สามลำดับชั้น นั่นคือ $\bm{X} \in \mathbb{R}^{3 \times 100 \times 200}$.
ชั้นคอนโวลูชั่นใช้ฟิลเตอร์ $\bm{W}$ ขนาด $3 \times 5 \times 5$ จำนวน $24$ ตัว
โดยทำคอนโวลูชั่น (การเชื่อมต่อท้องถิ่นและใช้ค่าน้ำหนักร่วม) เฉพาะกับชุดมิติที่สองและสาม
(ชุดมิติแรกเป็นเสมือนช่องลักษณะสำคัญที่ไม่มีความสัมพันธ์ในเชิงลำดับ).
ไม่มีการเติมเต็มอินพุต และใช้ขนาดย่างก้าวเป็น $1$.
ผลลัพธ์จากชั้นคอนโวลูชั่น จะเป็นเทนเซอร์ขนาดเท่าใด?
\textit{คำใบ้} 
ดูสมการ~\ref{eq: deep conv conv CxHxW}
(สำหรับฟิลเตอร์แต่ละตัว เมื่อขนาดก้าวย่างเป็นหนึ่ง
เอาต์พุต
$a_{k,l} = b + \sum_{c} \sum_{i} \sum_{j} w_{c,i,j} \cdot x_{c, k+i-1, l+j-1}$%
%โดย $k = 1, \ldots, H - H_F + 1$ และ $$
). 
% answer 24 x 96 x 196

\item (ง) อินพุตเป็นเทนเซอร์สี่ลำดับชั้น นั่นคือ $\bm{X} \in \mathbb{R}^{4 \times 300 \times 400 \times 50}$.
ชั้นคอนโวลูชั่นใช้ฟิลเตอร์ $\bm{W}$ ขนาด $4 \times 11 \times 11 \times 7$ จำนวน $64$ ตัว
โดยทำคอนโวลูชั่น (การเชื่อมต่อท้องถิ่นและใช้ค่าน้ำหนักร่วม) เฉพาะกับชุดมิติที่สอง ที่สาม และที่สี่
(ชุดมิติแรกเป็นเสมือนช่องลักษณะสำคัญที่ไม่มีความสัมพันธ์ในเชิงลำดับ).
ไม่มีการเติมเต็มอินพุต และใช้ขนาดย่างก้าวเป็น $1$.
ผลลัพธ์จากชั้นคอนโวลูชั่น จะเป็นเทนเซอร์ขนาดเท่าใด?
\textit{คำใบ้} 
สำหรับฟิลเตอร์แต่ละตัว 
เอาต์พุต
$a_{k,l,m} = b + \sum_{c} \sum_{i} \sum_{j} \sum_{q} w_{c,i,j,q} \cdot x_{c, k+i-1, l+j-1, m+q-1}$.
%answer = 64 x 290 x 390 x 44

\end{itemize}

\end{Exercise}

\begin{Exercise}
	\label{ex: convolution strides}
	
จากแบบฝึกหัด~\ref{ex: convolution}
จงประมาณขนาดเทนเซอร์ของเอาต์พุต ในกรณีต่าง ๆ เมื่อใช้ขนาดย่างก้าวเป็น $2$, เป็น $3$ และเป็น $4$.
\textit{คำใบ้} ดูสมการ~\ref{eq: deep size of conv output}  (
$H' = \left\lfloor \frac{H - H_F}{S} \right\rfloor + 1$
).

%Stride = 2
% ก 15 x 4
% ข 15 x 4
% ค 24 x 48 x 98
% ง 64 x 145 x 195 x 22

%Stride = 3
% ก 15 x 3
% ข 15 x 3
% ค 24 x 32 x 66
% ง 64 x 97 x 130 x 15


%Stride = 4
% ก 15 x 2
% ข 15 x 2
% ค 24 x 24 x 49
% ง 64 x 73 x 98 x 11

\end{Exercise}

\begin{Exercise}
	\label{ex: convolution padding}

จากแบบฝึกหัด~\ref{ex: convolution}
จงประมาณการเติมเต็มด้วยค่าศูนย์ (จำนวนค่าศูนย์ที่ต้องเติม) ทั้ง $4$ กรณี
โดย
\begin{itemize}
	\item (แบบที่ 1) เติมให้เอาต์พุตมีขนาดเท่ากับอินพุต เมื่อใช้ขนาดก้าวย่างเป็น $1$ (พิจารณาเฉพาะในชุดมิติที่มีความสัมพันธ์เชิงลำดับ เช่น กรณี ข อินพุต $\bm{X} \in \mathbb{R}^{8 \times 10}$ แต่ชุดมิติแรกไม่มีความสัมพันธ์เชิงลำดับ.
	ดังนั้น สัดส่วนของเอาต์พุตที่ต้องการคือ $15 \times 10$ หรือเอาต์พุต $\bm{A} \in \mathbb{R}^{15 \times 10}$. ขนาด $15$ มาจากจำนวนฟิลเตอร์ที่ใช้ ไม่เกี่ยวกับการเติมเต็มด้วยค่าศูนย์). 
	\item (แบบที่ 2) เติมให้เอาต์พุตมีขนาดเท่ากับ $\left\lceil\frac{H}{S}\right\rceil$
	โดย $H$ คือขนาดอินพุต และ $S$ คือขนาดก้าวย่าง
	เมื่อใช้ขนาดก้าวย่างเป็น $2$, เป็น $3$, และเป็น $4$ ตามลำดับ.
	(พิจารณาเฉพาะในชุดมิติที่มีความสัมพันธ์เชิงลำดับ เช่น กรณี ค อินพุต $\bm{X} \in \mathbb{R}^{3 \times 100 \times 200}$ แต่ชุดมิติแรกไม่มีความสัมพันธ์เชิงลำดับ. 
	ดังนั้น สัดส่วนของเอาต์พุตที่ต้องการคือ $24 \times 50 \times 100$ เมื่อใช้ขนาดก้าวย่าง $2$ และคือ $24 \times 34 \times 67$ เมื่อใช้ขนาดก้าวย่าง $3$ เป็นต้น).
\end{itemize}
\textit{คำใบ้} ดูสมการ~\ref{eq: deep size of padded input}  
ซึ่งคือ $\hat{H} = S \cdot (\hat{H}' - 1) + H_F$
และ $\hat{H} - H$.

\end{Exercise}



\begin{Exercise}
	\label{ex: convolution output with padding}


จงคำนวณขนาดของเอาต์พุตจากชั้นคอนโวลูชั่น สำหรับกรณีต่าง ๆ ดังนี้ 
\begin{itemize}
	\item (ก) อินพุตเป็นเวกเตอร์ นั่นคือ $\bm{x} \in \mathbb{R}^{10}$
	และชั้นคอนโวลูชั่นใช้ฟิลเตอร์ $\bm{w}$ มีขนาด $3$ จำนวน $15$ ตัว โดยเติมเต็มด้วยค่าศูนย์จำนวนรวม $2$ ตัว
	ขนาดย่างก้าวเป็น $1$ แล้วผลลัพธ์จากชั้นคอนโวลูชั่น จะเป็นเทนเซอร์ขนาดเท่าใด?
	% เฉลย
	% 15 x 10
	
	\item (ข) อินพุตเป็นเทนเซอร์สองลำดับชั้น คือ $\bm{X} \in \mathbb{R}^{8 \times 10}$.
	ชั้นคอนโวลูชั่นใช้ฟิลเตอร์ $\bm{W}$ ขนาด $8 \times 3$ จำนวน $15$ ตัว
	โดยทำคอนโวลูชั่น (การเชื่อมต่อท้องถิ่นและใช้ค่าน้ำหนักร่วม) เฉพาะกับชุดมิติที่สอง (ชุดมิติแรกเป็นเสมือนช่องลักษณะสำคัญที่ไม่มีความสัมพันธ์ในเชิงลำดับ).
	มีการเติมเต็มอินพุตด้วยค่าศูนย์จำนวน $7$ ตัว และใช้ขนาดย่างก้าวเป็น $2$.
	ผลลัพธ์จากชั้นคอนโวลูชั่น จะเป็นเทนเซอร์ขนาดเท่าใด?
	% answer 15 x 8
	
	\item (ค) อินพุตเป็นเทนเซอร์สามลำดับชั้น นั่นคือ $\bm{X} \in \mathbb{R}^{3 \times 100 \times 200}$.
	ชั้นคอนโวลูชั่นใช้ฟิลเตอร์ $\bm{W}$ ขนาด $3 \times 5 \times 5$ จำนวน $24$ ตัว
	โดยทำคอนโวลูชั่น (การเชื่อมต่อท้องถิ่นและใช้ค่าน้ำหนักร่วม) เฉพาะกับชุดมิติที่สองและสาม
	(ชุดมิติแรกเป็นเสมือนช่องลักษณะสำคัญที่ไม่มีความสัมพันธ์ในเชิงลำดับ).
	มีการเติมเต็มอินพุตด้วยค่าศูนย์จำนวน $11$ ตัวในแต่ละชุดมิติ (ยกเว้นชุดมิติแรก) และใช้ขนาดย่างก้าวเป็น $3$.
	ผลลัพธ์จากชั้นคอนโวลูชั่น จะเป็นเทนเซอร์ขนาดเท่าใด?
	% answer 24 x 36 x 69
\end{itemize}
\textit{คำใบ้} $H' = \left\lfloor \frac{H - H_F + P}{S} \right\rfloor + 1$ เมื่อ $P$ คือจำนวนศูนย์ที่เติมเข้าไปทั้งหมด.

\end{Exercise}

\begin{Exercise}
\label{ex: receptive field}

จงคำนวณขนาดของสนามรับรู้ของหน่วยย่อยในชั้นสุดท้ายของกรณีต่อไปนี้
\begin{itemize}
	\item (ก) โครงข่ายคอนโวลูชั่นหนึ่งชัั้น ที่ใช้ฟิลเตอร์ขนาด $5 \times 5$ ขนาดก้าวย่างเป็น $1 \times 1$, เป็น $2 \times 2$ และเป็น $3 \times 3$ ตามลำดับ.
%	\\
%	ตัวอย่างนี้ โครงข่ายมีแค่ชั้นเดียว และคำตอบคือ สนามรับรู้มีขนาดเป็น $5 \times 5$ ไม่ว่าจะใช้ขนาดก้าวย่างเท่าไร.
%	
	\item (ข) โครงข่ายคอนโวลูชั่นสองชัั้น ทั้งสองชั้นใช้ฟิลเตอร์ขนาด $5 \times 5$ ขนาดก้าวย่างเป็น $1 \times 1$.
% R_2 = 9x9; 1 + 4(1)(1) + 4(1)	
	\item (ค) โครงข่ายคอนโวลูชั่นสองชัั้น ทั้งสองชั้นใช้ฟิลเตอร์ขนาด $11 \times 11$ ขนาดก้าวย่างเป็น $1 \times 1$.
% R_2 = 21x21; 1 + 10(1)(1) + 10(1)	
	\item (ง) โครงข่ายคอนโวลูชั่นสามชัั้น ทั้งสองชั้นใช้ฟิลเตอร์ขนาด $5 \times 5$ ขนาดก้าวย่างเป็น $1 \times 1$.
% R_3 = 13x13; 1 + 4(1)(1)(1) + 4(1)(1) + 4(1)
	\item (จ) โครงข่ายคอนโวลูชั่นห้าชั้น โดยฟิลเตอร์ชั้นแรก $11 \times 11$ ก้าวย่าง $1 \times 1$,
ฟิลเตอร์ชั้นสอง $5 \times 5$ ก้าวย่าง $1 \times 1$,
ฟิลเตอร์ชั้นสามถึงห้าใช้ฟิลเตอร์แบบเดียวกัน คือ $3 \times 3$ ก้าวย่าง $1 \times 1$.
% R_5 = 21x21; 1 + 10 + 4 + 2 + 2 + 2
	\item (ฉ) โครงข่ายคอนโวลูชั่นสิบชั้น โดยทุกชั้นใช้ฟิลเตอร์แบบเดียวกัน คือ $3 \times 3$ ก้าวย่าง $1 \times 1$.
% R_10 = 21x21; 1 + 2 + ... + 2 = 1 + 2(10)	
	\item (ช) โครงข่ายคอนโวลูชั่นสามชัั้น ชั้นที่หนึ่งและสามใช้ฟิลเตอร์ขนาด $3 \times 3$ ขนาดก้าวย่างเป็น $1 \times 1$
	แต่ชั้นที่สองใช้ฟิลเตอร์ขนาด $2 \times 2$ ก้าวย่าง $2 \times 2$.
% R_3 = 8x8; 1 + 2(2)(1) + 1(1) + 2	
\end{itemize}	
%	
\textit{คำใบ้} ดูสมการ~\ref{eq: receptive field} ($R_k = 1 + \sum_{j=1}^k (F_j - 1) \prod_{i=0}^{j-1} S_i$ และกำหนด $S_0 = 1$)

\end{Exercise}

\begin{Exercise}

จากสมการ~\ref{eq: deep conv conv FxCxHxW} และ~\ref{eq: deep conv 2Dconv Output} สำหรับ\textit{คอนโวลูชั่นสองมิติ}
จงเขียนสมการคำนวณแผนที่ลักษณะสำคัญ (เอาต์พุต) ของชั้นคอนโวลูชั่น
สำหรับ 
\begin{itemize}
	\item (ก) \textit{คอนโวลูชั่นหนึ่งมิติ} (มีชุดลำดับมิติชุดเดียว 
	อินพุต $\bm{X} \in \mathbb{R}^{C \times H}$ โดยชุดมิติแรกไม่มีความสัมพันธ์เชิงลำดับ).
	\item (ข) \textit{คอนโวลูชั่นสามมิติ} (มีชุดลำดับมิติสัมพันธ์สามชุด
	อินพุต $\bm{X} \in \mathbb{R}^{C \times H \times W \times D}$ โดยชุดมิติแรกไม่มีความสัมพันธ์เชิงลำดับ).
	\item (ค) \textit{คอนโวลูชั่นสี่มิติ} (มีชุดลำดับมิติสัมพันธ์สี่ชุด
	อินพุต $\bm{X} \in \mathbb{R}^{C \times H \times W \times D \times E}$ โดยชุดมิติแรกไม่มีความสัมพันธ์เชิงลำดับ).
\end{itemize}

\end{Exercise}

\paragraph{การโปรแกรมตรรกะของโครงข่ายคอนโวลูชั่น.}
การคำนวณของโครงข่ายคอนโวลูชั่น ประกอบด้วยการคำนวณของชั้นคำนวณสามชนิดหลัก ๆ ได้แก่ ชั้นคำนวณคอนโวลูชั่น ชั้นดึงรวม และชั้นเชื่อมต่อเต็มที่.
รายการ~\ref{code: MyConv2D} แสดงตัวอย่างโปรแกรมของชั้นคำนวณคอนโวลูชั่น.
โปรแกรมในรายการ~\ref{code: MyConv2D} อาศัยการจัดเรียงเทนเซอร์ใหม่ และใช้ประโยชน์จากการคูณเมทริกซ์.
รูป~\ref{fig: vectorize conv layer}
แสดงแนวคิด การจัดเรียงเทนเซอร์ใหม่
เพื่อที่การคูณเมทริกซ์จะให้ผลลัพธ์เสมือนการคำนวณคอนโวลูชั่น.
สังเกต สมการ~\ref{eq: deep conv conv FxCxHxW} เอาต์พุต $\bm{a}$ เป็นเทนเซอร์สัดส่วน $M \times H' \times W'$ 
(เมื่อ $M$ เป็นจำนวนลักษณะสำคัญ และ $H'$ กับ $W'$ เป็นขนาดความสูงและกว้างของแผนที่เอาต์พุต)
สำหรับจุดข้อมูลแต่ละจุด.
ดังนั้น สำหรับชุดข้อมูลหมู่ขนาด $N$ ผลลัพธ์จะเป็นเทนเซอร์สัดส่วน $N \times M \times H' \times W'$.
เอาต์พุต จากการคูณเมทริกซ์ $\bm{W}_{M \times C \cdot H_f \cdot W_f} \cdot \bm{X}_{C \cdot H_f \cdot W_f \times H' \cdot W' \cdot N}$
จะเป็นเมทริกซ์ขนาด $M \times H' \cdot W' \cdot N$ ซึ่งสามารถนำมาจัดเรียงเป็นเทนเซอร์สัดส่วน $N \times M \times H' \times W'$ ได้.

หมายเหตุ การเขียนโปรแกรมคำนวณคอนโวลูชั่น เช่น สมการ~\ref{eq: deep conv conv FxCxHxW}
ด้วยการวนลูป ก็สามารถทำได้ แต่การทำงานอาจทำได้ช้ามาก.
ผู้อ่านสามารถทดลองวิธีการเขียนโปรแกรมหลาย ๆ แนวทาง และเปรียบเทียบข้อดีข้อเสีย
ในแง่ต่าง ๆ เช่น ประสิทธิภาพการทำงาน  ความยากง่ายในการแก้ไขและปรับปรุง.

%
\begin{figure}
\begin{center}
	\includegraphics[width=\textwidth]{06Conv/VectorizedConv3.png}
\end{center}
\caption[ตัวอย่างการจัดเทนเซอร์ให้อยู่ในรูปที่การคูณเมทริกซ์เสมือนการคำนวณเทนเซอร์]{ตัวอย่างการจัดเทนเซอร์ของค่าน้ำหนักและอินพุตให้อยู่ในรูปที่การคูณเมทริกซ์เสมือนการคำนวณเทนเซอร์. 
ภาพภายในพื้นหลังสีเหลืองอ่อน แสดงค่าน้ำหนัก  
และอินพุต.
ในภาพ \texttt{m} แทนดัชนีของฟิลเตอร์ หรือลักษณะสำคัญ.
ตัว \texttt{C} เป็นดัชนีของช่อง.
ค่าน้ำหนักซึ่งเป็นเทนเซอร์สี่ลำดับชั้น
$\bm{w} \in \mathbb{R}^{2 \times 3 \times 3 \times 3}$ (สองฟิลเตอร์ แต่ละฟิลเตอร์มีสามช่อง และแต่ละช่องขนาด $3 \times 3$) แสดงด้วยกรอบหกกรอบจัดเรียงเป็นสองแถว (ตามจำนวนฟิลเตอร์) และสามสดมภ์ (ตามจำนวนช่อง).
ตัวเลขภายในกรอบแสดงค่าดัชนีเชิงพื้นที่ในแนวตั้งและแนวนอน (อาจเรียกเป็น $(i,j)$ แต่ไม่ได้ระบุในภาพ).
อินพุตซึ่งเป็นเทนเซอร์สี่ลำดับชั้น  
$\bm{x} \in \mathbb{R}^{3 \times 3 \times 5 \times 5}$
(สามจุดข้อมูล แต่ละจุดข้อมูลมีสามช่อง แต่ละช่องขนาด $5 \times 5$)
ใช้ตัว \texttt{n} แสดงดัชนีของจุดข้อมูล.
พื้นที่แรงเงาสีเขียว (เขียวแก่ เขียวอ่อน และเขียวไข่กา) แสดงความสัมพันธ์ของการคำนวณในก้าวย่างที่สองตามแนวนอน (อาจเรียกเป็น $(k,l) = (0,1)$ เมื่อ
\textit{ขนาดก้าวย่าง} $S = 1$).
%
ลูกศร เชื่อมเทนเซอร์สี่ลำดับชั้นในรูปเดิม กับรูปแบบใหม่ที่สะดวกต่อการคำนวณ.
แต่ละแถวของเมทริกซ์ของค่าน้ำหนัก
แทนฟิลเตอร์แต่ละตัว (ดัชนี \texttt{m} ช่วยระบุ).
แต่ละค่าภายในแถว แสดงด้วยดัชนีสามตัว $(c,i,j)$ สำหรับช่อง ตำแหน่งแนวตั้ง และตำแหน่งแนวนอน.
สำหรับเมทริกซ์ของอินพุตที่จัดใหม่
แต่ละสดมภ์ แทนตำแหน่งของเอาต์พุตและจุดข้อมูล (ดัชนี $(k,l,n)$ สำหรับตำแหน่งเอาต์พุตแนวตั้ง แนวนอน และจุดข้อมูลที่ $n^{th}$).
ภายในสดมภ์ แสดงดัชนีของช่อง ตำแหน่งแนวตั้ง และตำแหน่งแนวนอนของอินพุตเดิม $(c,i,j)$.
สังเกต การจัดเรียงดัชนีในเมทริกซ์อินพุต ที่ทำให้การคูณเมทริกซ์เป็นเสมือนการคำนวณคอนโวลูชั่น.
}
\label{fig: vectorize conv layer}
\end{figure}
%

\lstinputlisting[language=python, caption={ตัวอย่างโปรแกรมชั้นคำนวณคอนโวลูชั่น},
label={code: MyConv2D}]{./06Conv/code/MyConv2D.py}

โปรแกรมในรายการ~\ref{code: net MyConv2D} แสดงตัวอย่างการเรียกใช้ \texttt{MyConv2D}.
โปรแกรม \texttt{MyConv2D} เขียนขึ้นตามรูปแบบของไพทอร์ช \texttt{nn.Conv2d} 
ดังนั้น การใช้งานก็ทำในลักษณะเดียวกันได้.
โปรแกรมในรายการ~\ref{code: train net MyConv2D}
และ~\ref{code: test net MyConv2D} แสดงตัวอย่างฝึกและทดสอบโครงข่าย 
(ค่าอภิมานพารามิเตอร์ต่าง ๆ ใช้ได้ดีกับชุดข้อมูลเอมนิสต์. ดูแบบฝึกหัด~\ref{ex: torch dataloader built-in mnist} สำหรับตัวอย่างการนำเข้าชุดข้อมูลเอมนิสต์).

\lstinputlisting[language=python, caption={[ตัวอย่างการเรียกใช้ชั้นคำนวณคอนโวลูชั่น]ตัวอย่างการเรียกใช้ชั้นคำนวณคอนโวลูชั่น \texttt{MyConv2D}},
label={code: net MyConv2D}]{./06Conv/code/NetMyConv2D.py}

\lstinputlisting[language=python, caption={[ตัวอย่างการฝึกโครงข่ายคอนโวลูชั่น]การฝึกโครงข่ายที่ใช้ชั้นคำนวณคอนโวลูชั่น \texttt{MyConv2D} สามารถทำได้แบบเดียวกับโครงข่ายประสาทเทียมอื่น ๆ},
label={code: train net MyConv2D}]{./06Conv/code/trainMyConvnet.py}

\lstinputlisting[language=python, caption={[ตัวอย่างการทดสอบโครงข่ายคอนโวลูชั่น]การทดสอบโครงข่ายที่ใช้ชั้นคำนวณคอนโวลูชั่น \texttt{MyConv2D} สามารถทำได้ เช่นเดียวกับการทดสอบโครงข่ายประสาทเทียมอื่น ๆ} ,
label={code: test net MyConv2D}]{./06Conv/code/testMyConvnet.py}



\begin{Exercise}
	\label{ex: CNN implementation}
	\index{english}{CNN}
	\index{english}{Convolutional Neural Network}
	\index{thai}{โครงข่ายคอนโวลูชั่น}
	
	จงศึกษาการทำงานของชั้นคำนวณคอนโวลูชั่นและวิธีการเขียนโปรแกรมในรายการ~\ref{code: MyConv2D}
	แล้วทดสอบการทำงานเปรียบเทียบกับโปรแกรมสำเร็จรูป \texttt{nn.Conv2d} รวมถึงทดสอบโครงสร้างแบบอื่น ๆ (เปลี่ยนค่าอภิมานพารามิเตอร์ เช่น ขนาดฟิลเตอร์ จำนวนฟิลเตอร์ ขนาดก้าวย่าง จำนวนการเติมเต็มด้วยศูนย์) อภิปรายและสรุป.
	หมายเหตุ ในทางปฏิบัติ การใช้โปรแกรมสำเร็จรูปจะสะดวกกว่า การอ้างอิงก็ทำได้ง่ายกว่า ถูกยอมรับดีกว่า (โปรแกรมมาตราฐาน เชื่อว่าได้รับการตรวจสอบมาดีกว่า) และดังเช่นที่จะได้เห็นจากการทดลอง
	ในกรณีนี้ โปรแกรมสำเร็จรูป \texttt{nn.Conv2d} ทำงานได้มีประสิทธิภาพมากกว่าอย่างเห็นได้ชัด (การเขียนโปรแกรมประสิทธิภาพสูง อาจต้องอาศัยการโปรแกรมระดับล่าง ซึ่งอยู่นอกเหนือขอบเขตของหนังสือเล่มนี้).
	แต่การศึกษาโปรแกรมในรายการ~\ref{code: MyConv2D} ทำเพื่อให้เข้าใจกลไกการทำงานของชั้นคำนวณคอนโวลูชั่นอย่างกระจ่างแจ้ง.

\end{Exercise}

\begin{Exercise}
	\label{ex: CNN FC layer with backward}
	
	การเขียนโปรแกรมชั้นเชื่อมต่อเต็มที่ก็สามารถทำได้ในลักษณะเดียวกัน.
รายการ~\ref{code: MyFCBack} แสดงตัวอย่างโปรแกรมเชื่อมต่อเต็มที่ที่เขียนการแพร่กระจายย้อนกลับเอง
โดยการคำนวณจริงทำผ่านการเรียกฟังก์ชัน \texttt{fcf} ที่เขียนดังในรายการ~\ref{code: fcf}.
การใช้งานชั้นเชื่อมต่อเต็มที่ \texttt{MyFCBack} ก็ทำเช่นเดียวกับการเรียกใช้ชั้นคำนวณ \texttt{nn.Linear}
เช่น การเปลี่ยนบรรทัดคำสั่ง \texttt{self.fc1 = nn.Linear(8*7*7, 10)} ในรายการ~\ref{code: net MyConv2D}
เป็น \texttt{self.fc1 = MyFCBack(8*7*7, 10)} เท่านั้น
ที่เหลือก็สามารถดำเนินงานสร้างโครงข่าย ฝึก และทดสอบได้เช่นเดิม.

\lstinputlisting[language=python, caption={[ตัวอย่างโปรแกรมชั้นเชื่อมต่อเต็มที่]ตัวอย่างโปรแกรมการคำนวณการเชื่อมต่อเต็มที่และการแพร่กระจายย้อนกลับ \texttt{fcf}},
label={code: fcf}]{./06Conv/code/fcf.py}

\lstinputlisting[language=python, caption={[ตัวอย่างโปรแกรมชั้นเชื่อมต่อเต็มที่ ที่เขียนการแพร่กระจายย้อนกลับเอง]ตัวอย่างโปรแกรมชั้นเชื่อมต่อเต็มที่ที่เขียนการแพร่กระจายย้อนกลับเอง \texttt{MyFCBack}.
ตัวอย่างนี้ เรียกใช้ฟังก์ชัน \texttt{fcf} ที่นิยามในรายการ~\ref{code: fcf}.},
label={code: MyFCBack}]{./06Conv/code/MyFCBack.py}
	
จงทดสอบการทำงานของชั้นเชื่อมต่อเต็มที่ \texttt{MyFCBack} เปรียบเทียบกับโปรแกรมสำเร็จรูป \texttt{nn.Linear}
ทั้งในเชิงการทำงาน และเวลาในการทำงาน.
รวมถึง จงทดลองแก้การคำนวณในฟังก์ชัน \texttt{fcf} เพื่อตรวจสอบดูว่าการคำนวณการเชื่อมต่อและการคำนวณแพร่กระจายย้อนกลับ ว่าได้ทำผ่าน
\texttt{fcf.forward} และ \texttt{fcf.backward} จริง.
ตัวอย่างเช่น ทดลองแก้บรรทัดคำสั่ง \texttt{return dEzp, dEw, dEb} เป็น \texttt{return 0*dEzp, 0*dEw, 0*dEb} และสังเกตผล.
สรุป และอภิปราย.

หมายเหตุ แม้การเขียนโปรแกรมชั้นเชื่อมต่อเต็มได้ถูกอภิปรายไปแล้วในหัวข้อ~\ref{section: deep exercises}
	การทบทวนอีกครั้งในแบบฝึกหัด เพื่อให้คุ้นเคยกับรูปแบบการเขียนโปรแกรมชั้นคำนวณ% 
	เพื่อใช้กับไพทอร์ช ที่ระบุการคำนวณการแพร่กระจายย้อนกลับด้วย.
	การทบทวนนี้ จะคาดว่าจะช่วยผู้อ่านเข้าใจกลไกของการเขียนโปรแกรมชั้นคำนวณพร้อมการระบุการแพร่กระจายย้อนกลับของไพทอร์ช ก่อนที่จะเขียนโปรแกรมชั้นคอนโวลูชั่น ซึ่งซับซ้อนขึ้นในแบบฝีกหัด~\ref{ex: Conv layer with backward}.
	
\end{Exercise}

\begin{Exercise}
	\label{ex: Conv layer with backward}
	
	คล้ายกับแบบฝึกหัด~\ref{ex: CNN FC layer with backward} แบบฝึกหัดนี้ศึกษาการเขียนโปรแกรมชั้นคอนโวลูชั่นทั้งการคำนวณ และการแพร่กระจายย้อนกลับ.
รายการ~\ref{code: MyConv2DB} แสดงตัวอย่างโปรแกรมชั้นคอนโวลูชั่นที่เขียนการแพร่กระจายย้อนกลับเอง
โดยการคำนวณจริงทำผ่านการเรียกฟังก์ชัน \texttt{convf} ที่เขียนดังในรายการ~\ref{code: convf}%
\footnote{
	รหัสโปรแกรมนี้ดัดแปลงจากรหัสโปรแกรม\textit{ฮิปสเตอร์เน็ต} (Hipsternet), 
	จาก \url{https://github.com/wiseodd/hipsternet/tree/master/hipsternet},
	ปรับปรุงล่าสุด 12 ก.พ. 2017.
}.
%
โปรแกรมชั้นคอนโวลูชั่น \texttt{MyConv2DB} รับมรดกมาจาก \texttt{MyConv2D} (รายการ~\ref{code: MyConv2D})
เพื่อลดความซ้ำซ้อน ที่จะต้องกำหนดค่าเริ่มต้นค่าน้ำหนัก (ภายในเมท็อด \verb|__init__|).
การใช้งานชั้นคอนโวลูชั่น \texttt{MyConv2DB} ก็ทำเช่นเดียวกับ \texttt{MyConv2D}
เช่น การเปลี่ยนบรรทัดคำสั่ง \texttt{self.conv1 = MyConv2D(1, 16, 5, 1, 2)} 
และบรรทัดคำสั่ง \texttt{self.conv2 = MyConv2D(16, 8, 3, 1, 1)}
ในรายการ~\ref{code: net MyConv2D}
เป็น \texttt{self.conv1 = MyConv2DB(1, 16, 5, 1, 2)} 
และ \texttt{self.conv2 = MyConv2D(16, 8, 3, 1, 1)} ตามลำดับ เท่านั้น
ที่เหลือก็สามารถดำเนินงานสร้างโครงข่าย ฝึก และทดสอบได้เช่นเดิม.

\lstinputlisting[language=python, caption={[ตัวอย่างโปรแกรมชั้นคอนโวลูชั่น]ตัวอย่างโปรแกรมการคำนวณคอนโวลูชั่นพร้อมการแพร่กระจายย้อนกลับ \texttt{convf}},
label={code: convf}]{./06Conv/code/convf.py}

\lstinputlisting[language=python, caption={[ตัวอย่างโปรแกรมชั้นคอนโวลูชั่น ที่เขียนการแพร่กระจายย้อนกลับเอง]ตัวอย่างโปรแกรมชั้นคอนโวลูชั่นที่เขียนการแพร่กระจายย้อนกลับเอง \texttt{MyConv2DB}
ซึ่งกลไกการคำนวณจริงทำผ่านฟังก์ชัน \texttt{convf} ที่นิยามในรายการ~\ref{code: convf}.},
label={code: MyConv2DB}]{./06Conv/code/MyConv2DB.py}

จงทดสอบชั้นคำนวณ \texttt{MyConv2DB} ทั้งในเชิงผลการทำงาน และประสิทธิภาพการทำงาน (วัดเวลาทำงาน)
รวมถึงทดสอบว่า การแพร่กระจายย้อนกลับทำผ่าน \texttt{convf.backward} จริง (ดูแบบฝึกหัด~\ref{ex: CNN FC layer with backward} ประกอบ).
แล้วเปรียบเทียบกับโปรแกรมสำเร็จรูป \texttt{nn.Conv2d}.

หมายเหตุ การเขียนโปรแกรมเองในที่นี้เพื่อความกระจ่างในกลไกการทำงาน 
แต่ในทางปฏิบัติ แนะนำให้ใช้โปรแกรมสำเร็จรูป 
ด้วยเหตุผลด้านความสะดวก ประสิทธิภาพ การทดสอบที่ดีและครอบคลุมกว่า รวมถึงความยอมรับและความไว้วางใจของผู้เกี่ยวข้อง.

\end{Exercise}


\begin{Exercise}
	\label{ex: Maxpool layer with backward}

แบบฝึกหัดนี้ศึกษาการเขียนโปรแกรมชั้นดึงรวมแบบมากที่สุด ทั้งการคำนวณ และการแพร่กระจายย้อนกลับ.
รายการ~\ref{code: MyMaxpool} แสดงตัวอย่างโปรแกรมชั้นดึงรวมแบบมากที่สุด ที่เขียนการแพร่กระจายย้อนกลับเอง
โดยการคำนวณจริงทำผ่านการเรียกฟังก์ชัน \texttt{maxpoolf} ที่เขียนดังในรายการ~\ref{code: maxpoolf}%
\footnote{
	รหัสโปรแกรมนี้ดัดแปลงจากรหัสโปรแกรม\textit{ฮิปสเตอร์เน็ต} (Hipsternet), 
	จาก \url{https://github.com/wiseodd/hipsternet/tree/master/hipsternet},
	ปรับปรุงล่าสุด 12 ก.พ. 2017.
}.
%
การใช้งานชั้นดึงรวม \texttt{MyMaxpool} ก็ทำเช่นเดียวกับโปรแกรมสำเร็จรูป \texttt{nn.MaxPool2d}
เช่น การเปลี่ยนบรรทัดคำสั่ง \texttt{self.pool1 = nn.MaxPool2d(2,2)} 
และ %บรรทัดคำสั่ง 
\texttt{self.pool2 = nn.MaxPool2d(2, 2)}
ในรายการ~\ref{code: net MyConv2D}
เป็น \texttt{self.pool1 = MyMaxpool(2, 2, 0)} 
และ \texttt{self.pool2 = MyMaxpool(2, 2, 0)} ตามลำดับ เท่านั้น.
ส่วนที่เหลือก็สามารถดำเนินงานสร้างโครงข่าย ฝึก และทดสอบได้เช่นเดิม.

\lstinputlisting[language=python, caption={[ตัวอย่างโปรแกรมชั้นดึงรวมแบบมากที่สุด]ตัวอย่างโปรแกรมการคำนวณชั้นดึงรวมแบบมากที่สุดพร้อมการแพร่กระจายย้อนกลับ \texttt{maxpoolf}},
label={code: maxpoolf}]{./06Conv/code/maxpoolf.py}

\lstinputlisting[language=python, caption={[ตัวอย่างโปรแกรมชั้นดึงรวมแบบมากที่สุด ที่เขียนการแพร่กระจายย้อนกลับเอง]ตัวอย่างโปรแกรมชั้นชั้นดึงรวมแบบมากที่สุดที่เขียนการแพร่กระจายย้อนกลับเอง \texttt{MyMaxpool}
	ซึ่งกลไกการคำนวณจริงทำผ่านฟังก์ชัน \texttt{maxpoolf} ที่นิยามในรายการ~\ref{code: maxpoolf}.},
label={code: MyMaxpool}]{./06Conv/code/MyMaxpool.py}

จงทดสอบชั้นคำนวณ \texttt{MyMaxpool} ทั้งในเชิงผลการทำงาน และประสิทธิภาพการทำงาน (วัดเวลาทำงาน)
รวมถึงทดสอบว่า การแพร่กระจายย้อนกลับทำผ่าน \texttt{maxpoolf.backward} จริง.
แล้วเปรียบเทียบกับโปรแกรมสำเร็จรูป \texttt{nn.MaxPool2d}.

หมายเหตุ การเขียนโปรแกรมเองในที่นี้เพื่อความกระจ่างในกลไกการทำงาน 
แต่ในทางปฏิบัติ แนะนำให้ใช้โปรแกรมสำเร็จรูป 
ด้วยเหตุผลด้านความสะดวก ประสิทธิภาพ การทดสอบที่ดีและครอบคลุมกว่า รวมถึงความยอมรับและความไว้วางใจของผู้เกี่ยวข้อง.

\end{Exercise}


%{\small
%	\begin{shaded}
%		\paragraph{\small เกร็ดเทโลเมียร์ และการฝึกสมาธิ}
%		\index{Telomere}\index{เทโลเมียร์}
%		\index{Practice of Meditation}\index{การฝึกสมาธิ}
%		
%		\index{side discourse}
%		
%		\begin{center}
%			\begin{tabular}{ >{\arraybackslash}m{3.2in}  >{\arraybackslash}m{2.4in} }
%				``Dummy English Quote.''
%				&
%				``Dummy Thai Quote''
%				\\
%				Dummy attributed to who said the quote
%				&
%				who said (in Thai)
%			\end{tabular} 
%		\end{center}
%		\index{words of wisdom}
%		
%		https://www.youtube.com/watch?v=2wseM6wWd74
%		April 2017 The science of cells that never get old | Elizabeth Blackburn
%		TED Conference
%		Pond scum (Tetrahymena)
%		"Our findings meant that people's life events and the way we respond to these events can change how you maintain your telomeres."
%		
%		"they ere resilient to stress. Somehow they were able to experience their circumstances not as a threat day in and day out, but as a challenge,"
%		
%		"a study from the University of California, Los Angeles of people who are caring for a relative with dementia, long-term and looked at their caregiver's telomere maintenance capacity and found that it was improved by them practicing a form of meditation for as little as 12 minutes a day for two months."
%		
%		"... curious about a question we don't even know today is a question?"
%		
%	\end{shaded}
%	
%}%small

% LATER
%\paragraph{อเล็กซ์เน็ต}
%
%\begin{Exercise}
%	\label{ex: Alexnet}
%
%
%\url{https://pytorch.org/hub/pytorch_vision_alexnet/}
%
%\end{Exercise}