\chapter{การหาค่าดีที่สุดกับการรู้จำรู้แบบและการเรียนรู้ของเครื่อง}
\label{opt optimization in PRML}

%เพื่อให้เห็นอย่างชัดเจนว่า
\textit{การหาค่าดีที่สุด}นั้น 
เป็นพื้นฐานเบื้องหลังของ\textit{การรู้จำรู้แบบ}และ\textit{การเรียนรู้ของเครื่อง}
เนื้อหาส่วนนี้ ยกตัวอย่างภารกิจและวิธี\textit{การรู้จำรู้แบบ}แบบต่างๆ และอธิบายในมุมมองของศาสตร์\textit{การหาค่าดีที่สุด}

\section{การหาค่าถดถอย (ทั่วๆไป)}
\index{การหาค่าถดถอย}\index{regression}

\textit{การหาค่าถดถอย} (regression)
เป็นการอนุมานค่าเอาต์พุต $y \in \mathbb{R}$ จากค่าอินพุต $\bm{x}$
โดยคำนวณจาก $y = f(\bm{x}, \bm{\theta})$ เมื่อ $f(\cdot, \cdot)$ เป็นโมเดลที่เลือกใช้ และ $\bm{\theta}$ เป็นค่าพารามิเตอร์ของโมเดล.

ค่าพารามิเตอร์ $\bm{\theta}$ หาได้จากค่าที่ทำให้\textit{ผลรวมค่าผิดพลาดกำลังสอง}มีค่าน้อยที่สุด.
นั่นคือ
\begin{eqnarray}
\bm{\theta} = \arg\min_{\bm{w}} \frac{1}{2} \sum_{n=1}^N \left( f(\bm{\tilde{x}}_n, \bm{w}) - \tilde{y}_n \right)^2
\label{eq: opt regression loss}
\end{eqnarray}
โดยข้อมูลเก่า $\{\bm{\tilde{x}}_1, \ldots, \bm{\tilde{x}}_N\}$ และ $\{\tilde{y}_1, \ldots, \tilde{y}_N\}$ คือตัวอย่างอินพุตและเอาต์พุตตามลำดับ (จำนวน $N$ ตัวอย่าง).

ภาระกิจคือ การทำนายค่าตัวเลขจากค่าอินพุต.
ประสบการณ์ได้จากข้อมูลเก่า.
สมถนะวัดได้จาก\textit{ผลรวมค่าผิดพลาดกำลังสอง} (sum of squared errors, คำย่อ SSE).
\index{sum of squared errors}
\index{ผลรวมค่าผิดพลาดกำลังสอง}

\section{การจำแนกกลุ่ม (ทั่วๆไป)}
\index{การจำแนกกลุ่ม}\index{classification}

\textit{การจำแนกกลุ่ม} (classification)
เป็นการอนุมาน\textit{ฉลากของกลุ่ม} (class label หรือมักเรียกย่อว่า \textit{ฉลาก} หรือ label) \index{ฉลาก}\index{label}
$\bm{y}$ จากค่าอินพุต $\bm{x}$
โดย\textit{ฉลาก} $\bm{y}$ มักถูกแสดงใน\textit{รูปแบบรหัสหนึ่งร้อน} 
และค่าของ $\bm{y}$ คำนวณได้จาก $\bm{y} = f(\bm{x}, \bm{\theta})$ เมื่อ 
$\bm{\theta}$ เป็นค่าพารามิเตอร์ของโมเดล
$f(\cdot, \cdot)$ เป็นโมเดลที่เลือกใช้ 
และ $f$ ให้เอาต์พุตออกมาใน\textit{รูปแบบรหัสหนึ่งร้อน}.

\textit{รูปแบบรหัสหนึ่งร้อน} (one-hot coding หรือ one-of-K coding)%
\index{รูปแบบรหัสหนึ่งร้อน}\index{one-hot coding}\index{one-of-K coding}
ออกแบบมาสำหรับงานจำแนกกลุ่ม.
แนวคิดคือ \textit{รหัสหนึ่งร้อน}หนึ่งรูปแบบจะระบุกลุ่มหนึ่งกลุ่มจากกลุ่มทั้งหมด $K$ กลุ่ม 
โดยสำหรับการจำแนกกลุ่ม $K$ กลุ่ม รหัสจะประกอบด้วยตัวเลข $K$ ตัวเลข
และในจำนวน $K$ ตัวเลขนั้นจะมีแค่หนึ่งเลขเท่านั้นที่มีค่าเป็น $1$ ที่เหลือเป็น $0$ ทั้งหมด.
ตำแหน่งของตัวเลขที่มีค่าเป็น $1$ จะระบุกลุ่มของอินพุต.
ในทางปฏิบัติ\textit{รูปแบบรหัสหนึ่งร้อน} ถูกควบคุมอย่างหลวมๆ ด้วย
เงื่อนไข $\bm{y} = [y_1, \ldots, y_K]^T, y_k \in [0,1]$ สำหรับ $k = 1, \ldots, K$ และ $\sum_{k=1}^K y_k = 1$.

ค่าพารามิเตอร์ $\bm{\theta}$ หาได้จากค่าที่ทำให้\textit{ผลรวมค่าเอนโทรปีผสม}น้อยที่สุด.
นั่นคือ
\begin{eqnarray}
  \bm{\theta} &=& \arg\min_{\bm{w}} \sum_{n=1}^N \mathrm{loss}_n
  \label{eq: opt classification loss}, \\
%
  \mathrm{loss}_n &=& - \sum_{k=1}^K \tilde{y}_{k,n} \cdot \log f^{(k)}(\bm{\tilde{x}}_n, \bm{w})
  \label{eq: opt classification cross-entropy}
\end{eqnarray}
โดย
$f^{(k)}(\cdot, \cdot)$ แทนส่วนประกอบที่ $k^{th}$ ของผลลัพธ์จากฟังชั่น $f$
และ
ข้อมูลเก่า $\{\bm{\tilde{x}}_1, \ldots, \bm{\tilde{x}}_N\}$ และ $\{\bm{\tilde{y}}_1, \ldots, \bm{\tilde{y}}_N\}$ คือตัวอย่างอินพุตและเอาต์พุตตามลำดับ (จำนวน $N$ ตัวอย่าง).
ตัวอย่างเอาต์พุต $\bm{\tilde{y}}_n = [\tilde{y}_{1,n}, \ldots, \tilde{y}_{K,n}]^T$ สำหรับ $n=1, \ldots, N$.

ภาระกิจคือ การทำนายกลุ่มของค่าอินพุต.
ประสบการณ์ได้จากข้อมูลเก่า.
สมถนะวัดได้จาก\textit{ผลรวมค่าเอนโทรปีผสม} (sum of cross entropy).
\index{ผลรวมค่าเอนโทรปีผสม}
\index{sum of cross entropy}

\section{การจำแนกกลุ่มแบบสองกลุ่ม (ซัพพอร์ตเวคเตอร์แมชชีน)}
\index{การจำแนกกลุ่มแบบสองกลุ่ม}\index{binary classification}
\index{ซัพพอร์ตเวคเตอร์แมชชีน}\index{Support Vector Machine}
\index{SVM}

\textit{การจำแนกกลุ่มแบบสองกลุ่ม} (bi-class or binary classification)
เป็นการอนุมานค่าเอาต์พุต $y \in \{-1,+1\}$ จากค่าอินพุต $\bm{x}$.
\textit{ซัพพอร์ตเวคเตอร์แมชชีน} (Support Vector Machine คำย่อ SVM) เป็นวิธีหนึ่งที่ออกแบบมาสำหรับงาน\textit{การจำแนกกลุ่มแบบสองกลุ่ม}.
\textit{ซัพพอร์ตเวคเตอร์แมชชีน}อนุมานค่าเอาต์พุตจาก
\begin{eqnarray}
y &=& \mathrm{sign}\left(\sum_{i=1}^K \hat{a}_i \cdot \hat{y}_i \cdot \kappa(\bm{x}, \bm{\hat{x}}_i) + b \right)
\label{eq: opt svm inference}
\end{eqnarray}
เมื่อ 
$\mathrm{sign}(u) = -1$ ถ้า $u < 0$ นอกนั้น $\mathrm{sign}(u) = 1$.
ตัวแปร $\bm{\hat{x}}_i, \hat{y}_i$ เป็นข้อมูลเก่า (หรือข้อมูลเก่าที่ถูกเลือกออกมา ที่เรียกว่า \textit{เวคเตอร์สนับสนุน} หรือ support vectors\cite{CL2011}. 
$K$ คือจำนวน\textit{เวคเตอร์สนับสนุน})
และ
$\kappa(\cdot, \cdot)$ คือ \textit{ฟังชั่นแก่น} (kernel function)
เช่น 
\textit{แก่นเชิงเส้น} (linear kernel) ใช้
$\kappa(\bm{u}, \bm{v}) = \langle \bm{u}, \bm{v} \rangle = \bm{u}^T \bm{v}$
หรือ
\textit{แก่นเรเดียลเบซิส} (radial basis kernel) ใช้
$\kappa(\bm{u}, \bm{v}) = \exp( - \gamma || \bm{u} - \bm{v} ||^2 )$ และ $\gamma > 0$ เป็นค่าที่เลือกใช้. 

ค่าพารามิเตอร์ 
$\hat{a}_1, \ldots, \hat{a}_K$ 
ถูกเลือกมาจาก $a_1, \ldots, a_N$ 
โดย $\hat{a}_i$ คือค่าพารามิเตอร์ที่สอดคล้องกับ\textit{เวคเตอร์สนับสนุน} $\bm{\hat{x}}_i$ กับ $\hat{y}_i$ ที่ถูกเลือก.
นั่นคือ หากเลือก $\bm{\tilde{x}}_j$ กับ $\tilde{y}_j$
มาเป็น
\textit{เวคเตอร์สนับสนุน} $\bm{\hat{x}}_i$ กับ $\hat{y}_i$ 
แล้ว 
ดังนั้น $\hat{a}_i = a_j$.
ส่วนค่า $\bm{a} = [a_1, \ldots, a_N]^T$
และ $b$ หาได้จากการเลือก\textit{ขอบเขตตัดสินใจ} (decision hyperplane) ที่ทำให้ระยะห่างระหว่างขอบเขตของกลุ่มข้อมูลทั้งสองมีค่ามากที่สุด.
จากทฤษฎี\textit{การโปรแกรมเชิงเส้น} (linear programming) การคำนวณดังกล่าวเทียบเท่ากับกับการคำนวณใน\textit{ทวิภาค} (dual form) ดังนี้

%
\begin{eqnarray}
\bm{a} &=& \arg\min_{\bm{a}} \sum_{i=1}^N a_i - \frac{1}{2} \sum_{i=1}^N \sum_{j=1}^N a_i a_j \tilde{y}_i \tilde{y}_j \kappa(\tilde{x}_i, \tilde{x}_j)
\label{eq: opt SVM dual obj} \\
\mbox{s.t.} &\;& 
\nonumber \\
&\;& \sum_{i=1}^N \tilde{y}_i \cdot a_i = 0
%\label{eq: opt SVM dual obj criterion 1}, \\
\nonumber , \\
&\;& 0 \leq a_i \leq C, \mbox{ for all } i\mbox{'s}
%\label{eq: opt SVM dual obj criterion 2}. \\
\nonumber . 
\end{eqnarray}

และ
\begin{eqnarray}
b &=& \frac{1}{|M|}  \sum_{i \in M} 
\left\{ \tilde{y}_i - \sum_{j \in S} a_j \cdot \tilde{y}_j \kappa(\bm{\tilde{x}}_i, \bm{\tilde{x}}_j) \right\}
\label{eq: opt SVM b}
\end{eqnarray}

เมื่อ $M = \{i: 0 \leq a_i \leq C \}$
และ
$S = \{i: a_i > 0\}$.
สัญญลักษณ์ $|M|$ คือขนาดของเซต $M$.
ข้อมูลเก่า $\{\bm{\tilde{x}}_1, \ldots, \bm{\tilde{x}}_N\}$ และ $\{\tilde{y}_1, \ldots, \tilde{y}_N\}$ คือตัวอย่างอินพุตและเอาต์พุตตามลำดับ (จำนวน $N$ ตัวอย่าง).
ตัวแปร $C$ เป็นตัวแปรที่เลือกได้ โดยค่าที่ใหญ่จะบังคับให้โมเดลลดการทายกลุ่มผิดลง
แต่หากค่า $C$ ใหญ่เกินไปอาจนำไปสู่การ\textit{โอเวอร์ฟิต}กับข้อมูลเก่า และทำให้สูญเสีย\textit{คุณสมบัติความทั่วไป}ได้.

ภาระกิจคือ การทำนายฉลากกลุ่มจากค่าอินพุต.
ประสบการณ์ได้จากข้อมูลเก่า.
สมถนะวัดได้จากระยะห่างระหว่างขอบเขตของกลุ่มทั้งสอง.

%LATER
%\section{การลดขนาดมิติ (พีซีเอ)}
%\index{พีซีเอ}\index{PCA}\index{Principal Component Analysis}
%
%\textit{พีซีเอ} (PCA ซึ่งย่อมาจาก Principal Component Analysis)
%
%BREAK HERE


%LATER
%\section{การประมาณความหนาแน่น (การประมาณแก่นความหนาแน่น)}
%\index{การประมาณความหนาแน่น}\index{Density Estimation}
%\index{การประมาณแก่นความหนาแน่น}\index{Kernel Density Estimation}
%
%\textit{การประมาณแก่นความหนาแน่น} (Kernel Density Estimation, คำย่อ KDE)

\section{การแบ่งกลุ่ม (เคมีนส์)}
\index{การแบ่งกลุ่ม}\index{clustering}
\index{เคมีนส์}\index{K-Means}

\textit{การแบ่งกลุ่ม} (clustering)
เป็นการอนุมานกลุ่มให้กับข้อมูล $\bm{x}$
เพื่อให้ความต่างของข้อมูลในกลุ่มมีค่าน้อยที่สุด.
%
วิธี\textit{เคมีนส์} (K-Means) เป็นวิธีการอนุมานกลุ่มให้กับข้อมูล
โดยมองเสมือนว่าแต่ละกลุ่มมี\textit{จุดกึ่งกลาง} (centroid) ของกลุ่มอยู่ เพียงแต่ค่า\textit{จุดกึ่งกลาง}เป็น\textit{ตัวแปรแฝง} (latent variable) ที่ไม่รู้ค่าแน่นอน.
วิธี\textit{เคมีนส์}ใช้วิธีการประเมิน\textit{จุดกึ่งกลาง}ของกลุ่ม $\bm{\mu}_k$
ร่วมกับการมอบหมายกลุ่ม $r_{nk}$
เพื่อให้ได้ผลรวมระยะห่างระหว่างจุดข้อมูลต่างๆในกลุ่มเดียวกันน้อยที่สุด.
%
%โดยจัดจุดข้อมูลเข้าไปอยู่ในกลุ่มที่ระยะห่างระหว่าง\textit{จุดกึ่งกลาง}กับจุดข้อมูลน้อยที่สุด.
%จนกระทั่ง\textit{จุดกึ่งกลาง}ของกลุ่มสัมพันธ์กับข้อมูลในกลุ่ม.

นั่นคือ สำหรับข้อมูล $\bm{x}_1, \ldots, \bm{x}_N$ ที่ต้องการแบ่งเป็น $K$ กลุ่ม.
วิธี\textit{เคมีนส์}จะหาการมอบหมายกลุ่ม $r_{nk} \in \{0, 1\}$
%, n = 1, \ldots, N; k = 1, \ldots, K$
และ $\sum_{k=1}^K r_{nk}$ สำหรับ $n = 1, \ldots, N$
ที่ทำให้
ผลรวมระยะห่างระหว่างจุดข้อมูลในกลุ่มและ\textit{จุดกึ่งกลาง}มีค่าน้อยที่สุด
ซึ่งคือ หา $\bm{R} = [r_{nk}]$ โดย
\begin{eqnarray}
\bm{R} &=& \arg\min_{\bm{R}} \sum_{n=1}^N \sum_{k=1}^K r_{nk} || \bm{x}_n - \bm{\mu}_k ||^2
\label{eq: opt K-means obj}, \\
\mbox{s.t.} &\;& 
\nonumber \\
&\;& \sum_{k=1}^K r_{nk} = 1, \mbox{ for all } n\mbox{'s}
\nonumber
\end{eqnarray}
เมื่อ
\begin{eqnarray}
\bm{\mu}_k = \frac{\sum_{n=1}^N r_{nk} \bm{x}_n}{\sum_{n=1}^N r_{nk}}
\label{eq: opt K-means centroid}.
\end{eqnarray}

ภาระกิจคือ การทำนายการมอบหมายกลุ่มให้กับข้อมูล.
ประสบการณ์ได้จากตัวข้อมูลเอง.
สมถนะวัดได้จากผลรวมระยะห่างระหว่างจุดข้อมูลต่างๆในกลุ่มเดียวกัน.

\section{การเรียนรู้สถานะแฝง (วิธีค่าคาดหมายการหาค่ามากที่สุด)}

\textit{วิธีค่าคาดหมายการหาค่ามากที่สุด} (Expectation-Maximization คำย่อ EM) เป็นแนวทางหรือวิธีกว้างๆที่ใช้แก้ปัญหาการหาค่าดีที่สุด 
ที่มี
%ต้องหาค่า\texit{ตัวทำมากที่สุด} (maximizer) 
%เกี่ยวข้องกับ
\textit{ตัวแปรแฝง} (latent variable) เกี่ยวข้องด้วย
ในลักษณะที่ ต้องการหาค่าตัวแปรควบคุมเพื่อทำค่าจุดประสงค์ให้ดีสุด 
แต่ค่าตัวแปรควบคุมขึ้นอยู่กับค่าตัวแปรแฝง 
และในขณะเดียวกัน ค่าของตัวแปรแฝงก็ขึ้นกับค่าของตัวแปรควบคุม.
นั่นคือ ต้องการหาค่าตัวควบคุม $\bm{w}$ ที่ทำให้ฟังชั่นจุดประสงค์ $J$ มีค่ามากที่สุด แต่ฟังชั่น $J$ ขึ้นกับค่าตัวแปรแฝง $\bm{z}$ ซึ่งตัวแปรแฝงขึ้นกับค่าตัวควบคุม $\bm{z} = h(\bm{w})$
เมื่อ $h$ เป็นฟังชั่นความสัมพันธ์ระหว่าง $\bm{w}$ และ $\bm{z}$.
ปัญหานี้เขียนได้เป็น
\begin{eqnarray}
\bm{w} &=& \arg\max_{\bm{w}} J(\bm{w}, \bm{z})
\label{eq: opt EM obj}, \\
\bm{z} &=& h(\bm{w})
\label{eq: opt EM dependent}.
\end{eqnarray}

\textit{วิธีค่าคาดหมายการหาค่ามากที่สุด}จะใช้การทำงานสองขั้นตอนสลับกัน เรียกว่า \textit{ขั้นตอนอี} (E Step) และ\textit{ขั้นตอนเอ็ม} (M Step)
โดย
\textit{ขั้นตอนอี} จะประมาณค่าตัวแปรแฝง ในขณะที่ตรึงค่าของตัวแปรควบคุมให้คงที่.
นั่นคือ คำนวณ $\bm{z} = h(\bm{w})$
\textit{ขั้นตอนเอ็ม} จะหาค่าตัวแปรควบคุมที่ทำมากที่สุด 
ในขณะที่ตรึงค่าของตัวแปรแฝงคงที่.
นั่นคือ หา $\bm{w} = \arg\max_{\bm{w}} J(\bm{w}, \bm{z})$.
และทำขั้นตอนทั้งสองนี้สลับกันไป จนค่าของตัวแปรทั้งหมดลู่เข้า (หรือจนกว่าจะเข้าเกณฑ์การหยุดคำนวณ)

ตัวอย่างเช่น ภาระกิจของวิธี\textit{เคมีนส์} ซึ่ง
ต้องการหาค่าตัวแปรควบคุม $r_{nk}$ ที่ขึ้นกับค่าตัวแปรแฝง $\bm{\mu}_k$
และค่าตัวแปรแฝง $\bm{\mu}_k$ ก็ขึ้นกับค่าตัวแปรควบคุม $r_{nk}$.
วิธี\textit{เคมีนส์} จะเริ่มจาก
(1) \textit{ขั้นตอนอีเริ่มต้น} (Initial E Step) จะประมาณค่าตัวแปรแฝง
โดยการสุ่มจุดข้อมูลออกมาเป็นจุดกึ่งกลาง.
นั่นคือ สำหรับแต่ละกลุ่มที่ $k^{th}$ ประมาณ  $\bm{\mu}_k = \bm{x}_{\tilde{n}}$ เมื่อ $\bm{x}_{\tilde{n}}$ คือข้อมูลที่ถูกสุ่มขึ้นมา.
(2) \textit{ขั้นตอนเอ็ม} จะจัดทุกจุดข้อมูล $\bm{x}_n$ เข้ากลุ่ม $k^{th}$
หรือกำหนดค่า $r_{nk} = 1$ และ $r_{nj} = 0$
สำหรับทุกๆ $j = 1, \ldots, K$ ที่ $j \neq k$
ถ้า
$d_{n,k} \leq d_{n,j}$ 
เมื่อ $d_{n,k} = ||\bm{x}_n - \bm{\mu_k}||^2$.
ค่า $\bm{\mu_k}$ จะใช้ค่าล่าสุดที่ได้จาก\textit{ขั้นตอนอี}.
(3) \textit{ขั้นตอนอี} จะประมาณค่าตัวแปรแฝง $\bm{\mu}_k = \frac{\sum_n r_{nk} \bm{x}_n}{\sum_n r_{nk}}$.
ค่า $r_{nk}$ จะใช้ค่าล่าสุดที่ได้จาก\textit{ขั้นตอนเอ็ม}.
และจะทำขั้นตอนเอ็มและอีเช่นนี้ สลับกันไปจนค่าต่างๆลู่เข้า.

%LATER
%\section{การหาสินค้าแนะนำ (การกรองเชิงความร่วมมือ)}
%
%\textit{การกรองเชิงความร่วมมือ} (Collaborative Filtering\cite{SK2009})

%LATER
%\section{การหาหัวข้อเรื่อง (แอลดีเอ)}
%
%\textit{แอลดีเอ} (LDA ซึ่งย่อมาจาก Latent Dirichlet Allocation\cite{BNM2003})

%LATER
%\section{การรู้จำเชิงลำดับ (เอชเอ็มเอ็ม)}
%\textit{เอชเอ็มเอ็ม} (HMM ซึ่งย่อมาจาก Hidden Markov Model)
%

%LATER
%\section{การรู้จำเชิงลำดับ (แอลเอสทีเอ็ม)}
%\textit{แอลเอสทีเอ็ม} (LSTM ซึ่งย่อมาจาก Long Short-Term Memory\cite{GSC2000})

%LATER
%\section{การเรียนรู้แบบเสริมกำลัง (ทั่วไป)}
%
%\textit{การเรียนรู้แบบเสริมกำลัง} (Reinforcement Learning\cite{SB1998})

%LATER
%\section{การตรวจหาความใหม่}
%Novelty/Anomaly Detection\cite{PCCT2014}

%LATER
%\section{การจัดลำดับหน้าเวป (เพจแรงค์)}
%PageRank\cite{Page98thepagerank}

\section{การอนุมานเชิงสร้าง (โครงข่ายปรปักษ์เชิงสร้าง)}.

\textit{การอนุมานเชิงสร้าง} (Generative Inference) เป็นการเรียนรู้ข้อมูล เพื่อการสร้างข้อมูลขึ้นมาใหม่ โดยให้มีคุณลักษณะเช่นเดียวกับข้อมูลที่ได้เรียนรู้มา
หรือจากมุมมองทางทฤษฎี ก็คือการเรียนรู้ธรรมชาติ เรียนรู้คุณลักษณะ
หรือเรียนรู้ความน่าจะเป็นของข้อมูล.

\textit{โครงข่ายปรปักษ์เชิงสร้าง} (Generative Adversarial Network, คำย่อ GAN\cite{GPaMXWfOCB2014})
เป็น\textit{กรอบความคิด} (framework) สำหรับการสร้างโมเดลสำหรับ\textit{การอนุมานเชิงสร้าง} โดยใช้โมเดล สองโมเดล.
โมเดลหนึ่งฝึกเป็น\textit{โมเดลเชิงสร้าง} (Generative Model) $G$ 
ที่เรียนรู้ความน่าจะเป็นของข้อมูล
และอีกโมเดลหนึ่งฝึกเป็น\textit{โมเดลจำแนก} (Discriminative Model) $D$ ที่เรียนรู้ที่จะจำแนกข้อมูลจริงออกจากข้อมูลที่ถูกสร้างขึ้นมา.
โมเดลทั้งสองจะถูกฝึกในลักษณะแข่งกันในสภาพที่จุดประสงค์ขัดแย้งกัน.
นั่นคือ \textit{โมเดลจำแนก} $D$ จะถูกฝึกให้ประมาณค่าความน่าจะเป็นที่
จุดข้อมูล $\bm{x}$ มาจากชุดข้อมูลจริง ไม่ใช่ถูกสร้างมาจาก\textit{โมเดลเชิงสร้าง} $G$.
ในขณะที่\textit{โมเดลเชิงสร้าง} $G$ จะถูกฝึกให้สร้างตัวอย่าง $\bm{x}$ ที่ทำให้ความน่าจะเป็นที่\textit{โมเดลจำแนก} $D$ จะอนุมานถูก มีค่าน้อยที่สุด.

การอนุมานหรือการสร้างจุดข้อมูลใหม่ ทำได้โดยการคำนวณ
\begin{equation}
\bm{x}' = G(z, \bm{\theta}_g)
\label{eq: opt GAN generation}
\end{equation}
เมื่อ $\bm{x}'$ แทนจุดข้อมูลใหม่ที่ถูกสร้างขึ้น
%$G$ เป็น\textit{โมเดลเชิงสร้าง}.
ส่วน $\bm{\theta}_g$ เป็นค่าพารามิเตอร์ของโมเดลเชิงสร้าง
และ $z \in \mathbb{R}$ เป็นค่าที่สุ่มขึ้นมา เช่น $z \sim \mathcal{U}(0,1)$ สำหรับค่า $z$ ที่ถูกสุ่มแบบ\textit{การแจกแจงเอกรูป} (uniform distribution).

ค่าพารามิเตอร์ $\bm{\theta}_g$ สามารถหาได้จากการหาค่าดีที่สุดของฟังชั่นจุดประสงค์ของปัญหา\textit{เกมส์น้อยสุดของมากสุดแบบสองผู้เล่น} (two-player minimax game)
\begin{eqnarray}
\min_{\bm{\theta}_g} \max_{\bm{\theta}_d} \mathbb{E}_{\bm{x} \sim p_{data}(\bm{x})}\left[ \log D(\bm{x}, \bm{\theta}_d) \right] 
+ \mathbb{E}_{z \sim p_z(z)}\left[ \log (1 - D(G(z, \bm{\theta}_g), \bm{\theta}_d) ) \right]
\end{eqnarray}
เมื่อ $\bm{\theta}_d$ คือค่าพารามิเตอร์ของ\textit{โมเดลจำแนก} $D$.
เทอม $\mathbb{E}_{\bm{x} \sim p_{data}(\bm{x})}\left[ \log D(\bm{x}, \bm{\theta}_d) \right]$ คือ\textit{ค่าคาดหมาย} (expected value) ของความเป็นไปได้ที่\textit{โมเดลจำแนก} $D$ จะทายจุดข้อมูลจริง $\bm{x}$ ว่ามาจากข้อมูลจริง.
เทอม $\mathbb{E}_{z \sim p_z(z)}\left[ \log (1 - D(G(z, \bm{\theta}_g), \bm{\theta}_d) ) \right]$
คือ\textit{ค่าคาดหมาย}ของความเป็นไปได้ที่\textit{โมเดลจำแนก} $D$ จะทายจุดข้อมูลที่สร้างขึ้นมาจาก\textit{โมเดลเชิงสร้าง} ว่าไม่ใช่ข้อมูลจริง.

\textit{โมเดลจำแนก} $D$ จะถูกฝึก หรือค่า $\bm{\theta}_d$ จะถูกปรับเพื่อให้\textit{ค่าคาดหมาย}ของความเป็นไปได้ที่\textit{โมเดลจำแนก} $D$ จะทายถูก มีค่ามากที่สุด 
ทั้งสองกรณี.
นั่นคือ \textit{โมเดลจำแนก} $D$ พยายามจะทายให้ถูกว่าจุดข้อมูลจริง $\bm{x}$ มาข้อมูลจริง (ค่าคาดหมายเทอมแรก) และพยายามจะทายให้ถูกว่าจุดข้อมูลที่สร้างขึ้นด้วย $G(z, \bm{\theta}_g)$ ไม่ได้มาข้อมูลจริง (ค่าคาดหมายเทอมหลัง).
ในขณะที่
\textit{โมเดลเชิงสร้าง} $G$ จะถูกฝึก หรือค่า $\bm{\theta}_g$ จะถูกปรับเพื่อให้
\textit{ค่าคาดหมาย}ของความเป็นไปได้ที่\textit{โมเดลจำแนก} $D$ จะทายถูก มีค่าน้อยที่สุด.
นั่นคือ $G(z, \bm{\theta}_g)$ พยายามจะสร้างจุดข้อมูล $\bm{x}'$ ให้คล้ายกับว่า $\bm{x}'$ มาจากข้อมูลจริง หรือ $\bm{x}' \sim p_{data}(\bm{x})$.
การทำเช่นนี้เทียบเท่า \textit{โมเดลเชิงสร้าง} $G$ พยายามจะแปลงค่าสุ่ม $z \sim p_z(z)$ ไปเป็นจุดตัวอย่างในการแจกแจง $p_{data}(\bm{x})$ หรือเขียนสรุปง่ายได้ว่า $G: z \sim p_z(z) \mapsto x \sim p_{data}(\bm{x})$.

ภารกิจคือการสร้างจุดข้อมูลใหม่ที่มีคุณลักษณะเหมือนกับข้อมูลเดิม.
ประสบการณ์ได้จากข้อมูลเดิม.
สมถนะวัดได้จากความใกล้เคียงระหว่างค่าฟังชั่นจุดประสงค์ของ\textit{เกมส์น้อยสุดของมากสุดแบบสองผู้เล่น}
กับค่าฟังชั่นจุดประสงค์ที่\textit{จุดดีที่สุดท้องถิ่น} (local optimum) ซึ่งกูดเฟลโล่และคณะ\cite{GPaMXWfOCB2014} วิเคราะห์ไว้ว่าคือ $-\log 4$.

%\section{การอนุมานเชิงสร้าง (พิกเซลอาร์เอ็นเอ็น)}.
%\textit{พิกเซลอาร์เอ็นเอ็น} (PixelRNN)

%\subsection{การติดตามภาพวัตถุ (วิธีขยับค่าเฉลี่ย)}
%\textit{วิธีขยับค่าเฉลี่ย} (Mean Shift\cite{CRM2000})

%LATER
%\section{การอนุมานเชิงสร้าง (วาริเอชั่นนอลออตโตเอนโคดเดอร์)}.
%\textit{วาริเอชั่นนอลออตโตเอนโคดเดอร์} (Variational Auto-Encoder)

\section{การเข้าใจสิ่งที่อ่านของเครื่อง (เอมอาร์ซี) และฟิวชั่นเน็ต}
\index{การเข้าใจสิ่งที่อ่านของเครื่อง}
\index{Machine Reading Comprehension}

\textit{การเข้าใจสิ่งที่อ่านของเครื่อง} หรือ\textit{เอมอาร์ซี} (Machine Reading Comprehension คำย่อ MRC)\cite{HuangEtAl2017a}\cite{ChenEtAl2017a} 
ได้แก่
\textit{ภารกิจ} ที่ให้เครื่องอ่าน ประมวล และทำความเข้าใจ \textit{ข้อความ} (text) และสามารถตอบคำถามที่เกี่ยวข้องได้%
\footnote{%
	ถอดความจาก \textit{ฮวงและคณะ}\cite{HuangEtAl2017a}.
}%
.
%
หรือหากพูดเจาะจงลงไป นั่นคือ
หากกำหนด\textit{บริบท}(หรือข้อความที่เป็นแหล่งความรู้)ให้ และถาม\textit{คำถาม}
เครื่องต้องอ่าน ทำความเข้าใจ\textit{ข้อความบริบท}ที่มี และสามารถหาคำตอบให้กับ\textit{คำถาม}ได้.

ภารกิจนี้ อาจถูกตีกรอบดังนี้\cite{HuangEtAl2017a}.
บริบท สามารถแสดงได้ในรูปของ
ลำดับของ\textit{โทเค็นคำ} (word tokens)
นั่นคือ 
บริบท
%\begin{equation}
$\bm{C} = \{\bm{w}_1^C, \bm{w}_2^C, \bm{w}_3^C, \ldots, \bm{w}_m^C\}
\label{eq: opt MRC context}$
%\end{equation}
และ
คำถาม ก็สามารถแสดงได้ในลักษณะเดียวกัน
นั่นคือ
คำถาม
$\bm{Q} = \{\bm{w}_1^Q, \bm{w}_2^Q, \bm{w}_3^Q, \ldots, \bm{w}_n^Q\}
\label{eq: opt MRC context 2}$
%
โดย $m$ แทนจำนวนคำในบริบท.
ตัวแปร $n$ แทนจำนวนคำในคำถาม.
และ $\bm{w}_i$ คือ \textit{โทเค็นคำ} ที่แต่ละโทเค็นแสดงอยู่ในรูปเวคเตอร์ของเลขจำนวนจริง
$\bm{w}_i \in \mathbb{R}^d$ เมื่อ $d$ คือขนาดของเวคเตอร์.
คำหนึ่งคำ จะแทนด้วย \textit{โทเค็นคำ}หนึ่งโทเค็น.
\textit{โทเค็นคำ} อาจได้มาจาก\textit{วิธีถุงคำ} (bag of words) หรือ\textit{วิธีรหัสคำ} (word embedding) ในแบบต่างๆก็ได้ 
\index{word embedding}
\index{วิธีรหัสคำ}
(เช่น 
\textit{เวคเตอร์ทั่วไป}
%Global Vector
\cite[Global Vector]{PSM2014a}, 
\textit{เวคเตอร์คำผ่านบริบท}
%Contextualized Word Vector
\cite[Contextualized Word Vector]{McCannEtAl2017a}).
%
โดยทั่วไปแล้ว $m >> n$ ส่วนวิธีการหาคำตอบก็แตกต่างกันไป
โดยคำตอบ ก็อยู่ในรูป $\bm{A} = \{\bm{w}_1^A, \ldots, \bm{w}_k^A \}$ เมื่อ $k$ แทนจำนวนคำของคำตอบ. 
%ซึ่งวิธีการหา $\bm{w}_i^A, i = 1, \ldots, k$ 

แทนที่จะหาคำตอบทั่วๆไป เพื่อให้ปัญหาง่ายพอในการแก้ \textit{ราชพฤคาร์และคณะ}\cite{RajpurkarEtAl2016a} ตีกรอบปัญหาเป็นว่า
คำตอบ $\bm{A}$ จะเป็นส่วนหนึ่งของลำดับของคำในบริบท $\bm{C}$.
นั่นคือ $\bm{A} = \{\bm{w}_i^C, \ldots, \bm{w}_{i+k-1}^C$ โดย $k \leq m$ และ $i \in \{1, \ldots, m+1-k\}$.
ดังนั้นการตอบคำถาม $\bm{Q}$ คือ การหาตำแหน่งเริ่มต้น $i$ และจำนวนคำ $k$ ในบริบท $\bm{C}$.
\textit{ฮวงและคณะ}\cite{HuangEtAl2017a} เสนอ\textit{ฟิวชั่นเน็ต} (FusionNet) เพื่อแก้ปัญหานี้.

เทียบเท่าการหาค่า $i$ กับ $k$
\textit{ฮวงและคณะ}หาค่า $i$ และ $i^e$ (ซึ่ง $i^e = i + k -1$) จาก
\begin{eqnarray}
i^\ast, i^{e \ast} =& \arg\max_{i, i^e} & Pr[b=i|\bm{C}, \bm{Q}] \cdot Pr[e=i^e|\bm{C}, \bm{Q}] 
\label{eq: opt MRC objective function}
\nonumber \\
&\; s.t. & 0 \leq i^e - i \leq k - 1
\label{eq: opt MRC objective function condition}
\end{eqnarray}
เมื่อ $k$ คือจำนวนคำที่กำหนด%
\footnote{%
	ค่า $k$ เป็นเหมือนค่าความกระชับของคำตอบ.
	ถ้าค่า $k$ เล็ก ก็คือคำตอบจะกระชับ 
	ถ้าค่า $k$ ใหญ่ ก็คือคำตอบอาจจะเยิ่นเย้อ.
	นอกจากนั้น ค่า $k$ ยังเป็นกลไกที่ช่วยป้องกัน ไม่ให้โมเดลเลือกทั้งบริบทมาเป็นคำตอบอีกด้วย.
	\textit{ฮวงและคณะ}กำหนดค่า $k$ นี้เป็น $15$ คำ.
	นั่นคือ คำตอบจะไม่เกิน $15$ คำ.
}.
ค่า $Pr[b=i|\bm{C}, \bm{Q}]$ เป็น ค่าความน่าจะเป็น ที่ตำแหน่งเริ่มต้นจะเป็น $i$ สำหรับบริบท $\bm{C}$ และคำถาม $\bm{Q}$.
และค่า $Pr[e=i^e|\bm{C}, \bm{Q}]$ เป็น ค่าความน่าจะเป็น ที่ตำแหน่งสุดท้ายจะเป็น $i^e$ สำหรับบริบท $\bm{C}$ และคำถาม $\bm{Q}$.
\textit{ฮวงและคณะ}ประมาณ
ค่า $Pr[b=i|\bm{C}, \bm{Q}]$ 
ด้วย $\tilde{P}[b=i|\bm{C}, \bm{Q}]$
และ ค่า $Pr[e=i^e|\bm{C}, \bm{Q}]$ 
ด้วย $\tilde{P}[e=i^e|\bm{C}, \bm{Q}]$
โดย
%
\begin{eqnarray}
\tilde{P}[b=i|\bm{C}, \bm{Q}] & \propto & 
\exp{( (\bm{u}^Q)^T \cdot \bm{W}_B \cdot \bm{u}_i^C) }
\label{eq: opt fusion-net PS}
\nonumber
\\
\tilde{P}[e=i|\bm{C}, \bm{Q}] & \propto & 
\exp{( (\bm{v}^Q)^T \cdot \bm{W}_E \cdot \bm{u}_i^C) }
\label{eq: opt fusion-net PE}
\nonumber
\end{eqnarray}
เมื่อ $\bm{u}^Q$ เป็นเวคเตอร์สรุปความเข้าใจคำถามโดยรวม.
เวคเตอร์ $\bm{v}^Q$
เป็นเวคเตอร์สรุปความเข้าใจคำถามทั้งหมด 
โดยคำนึงถึง
ความคาดหมายของจุดเริ่มต้นของคำตอบที่ตำแหน่งต่างๆ.
เวคเตอร์ $\bm{u}_i^C$ เป็นเวคเตอร์แทนความเข้าใจบริบทในตำแหน่งทีี่ $i^{th}$.
ตัวแปร $\bm{W}_B \in \mathbb{R}^{d \times d}$
และ $\bm{W}_E \in \mathbb{R}^{d \times d}$
แทนเมตริกซ์ของค่าน้ำหนักความสัมพันธ์ 
ซึ่งค่าหาได้จาก\textit{กระบวนการฝึกของเครื่อง} (training).

เวคเตอร์สรุปความเข้าใจคำถามโดยรวม $\bm{u}^Q = \sum_i \beta_i \bm{u}_i^Q$ 
โดย $\beta_i \propto \exp{(\bm{w}^T \bm{u}_i^Q)}$
และ $\bm{w}$ ก็เป็นค่าน้ำหนักอีกชุด ที่หาได้จาก\textit{กระบวนการฝึกของเครื่อง}.
เวคเตอร์สรุปความเข้าใจคำถาม
โดยคำนึงถึงจุดเริ่มต้นของคำตอบ
$\bm{v}^Q = \mathrm{GRU}(\bm{u}^Q, \sum_i \tilde{P}[b=i|\bm{C}, \bm{Q}]
\cdot
 \bm{u}_i^C; \bm{\phi}_{vq})$
เมื่อ $\mathrm{GRU}$ แทนการคำนวณ\textit{กรู} (GRU, Gated Recurrent Unit\cite{ChoEtAl2014a})
ที่ใช้ $\bm{u}^Q$ เป็น\textit{ความจำ}
และใช้ $\sum_i \tilde{P}[b=i|\bm{C}, \bm{Q}] \cdot \bm{u}_i^C$ เป็นอินพุตของ\textit{กรู}
ส่วน $\bm{\phi}_{vq}$ %{\color{red}
%
เป็นพารามิเตอร์ของ\textit{กรู}ที่หาได้จาก\textit{กระบวนการฝึก.
%}.
%(to confirm after code check 2018 Nov 18)
}
%
เหล่าเวคเตอร์ $\{\bm{u}_i^C\}_{i=1, \ldots, m}$ และ
$\{\bm{u}_i^Q\}_{i=1, \ldots, n}$
ได้มาจาก\textit{ฟิวชั่นเน็ต}.

\paragraph{ฟิวชั่นเน็ต.}
\textit{ฟิวชั่นเน็ต}ทำหน้าที่แปลง
อินพุต ซึ่งประกอบด้วย
\textit{บริบท} $\bm{C} = \{\bm{w}_1^C, \ldots, \bm{w}_m^C \}$
และ\textit{คำถาม} $\bm{Q} = \{\bm{w}_1^Q, \ldots, \bm{w}_n^Q \}$
ไปเป็น
ความเข้าใจบริบท  (context understanding)
%(context understanding vectors) 
%จากมุมมองของคำถาม
$\bm{U}^C = \{\bm{u}^C_1, \ldots, \bm{u}^C_m \}$
% = \{ \bm{u}^C_i \}_{i=1, \ldots, m}$
และ
ความเข้าใจคำถาม 
(question understanding) 
$\bm{U}^Q = \{\bm{u}^Q_1, \ldots, \bm{u}^Q_n \}$.
%= \{ \bm{u}^Q_i \}_{i=1, \ldots, n}$

\textit{ฮวงและคณะ}\cite{HuangEtAl2017a}เชื่อว่า
การเชื่อมโยงความสัมพันธ์
ระหว่าง\textit{บริบท}
และ\textit{คำถาม}
ในหลายๆระดับ 
ตั้งแต่\textit{ระดับคำ} (word level)
\textit{ความหมายระดับล่าง} (low-level  semantics)
ไปจนถึง
\textit{มโนทัศน์ระดับสูง} (high-level concept)
จะช่วยให้สามารถหา\textit{คำตอบ}ได้ดีขึ้น.
%
ดังนั้น
ในสังเคราะห์\textit{ความเข้าใจบริบท}
$\bm{U}^C$
จาก\textit{บริบท} $\bm{C}$ 
และ\textit{คำถาม} $\bm{Q}$
จะเป็นเชื่อมโยงความสัมพันธ์ของ
\textit{บริบท}และ\textit{คำถาม}
ในหลายๆระดับ.

\textit{ฮวงและคณะ}แทน แต่ละคำทั้งในบริบทและคำถามด้วย
\textit{เวคเตอร์รหัสคำ}ที่มีส่วนประกอบเป็น
\textit{เวคเตอร์ทั่วไป}\cite{PSM2014a}
ขนาด $300$ มิติ
และ
\textit{เวคเตอร์คำผ่านบริบท}\cite{McCannEtAl2017a} ขนาด $600$ มิติ.
แต่คำในบริบทจะมีส่วนประกอบเพิ่มเติม คือ
\textit{รหัสคำ}จาก\textit{ชนิดของคำ} (part of speech คำย่อ POS) ขนาด $12$ มิติ
\textit{รหัสคำ}จาก\textit{สัญญาชืื่ออัตตลักษณ์} (Named Entity Recognition คำย่อ NER) ขนาด $8$ มิติ
\textit{ค่าบรรทัดฐานของความถี่คำ} (normalized term frequency) ขนาด $1$ มิติ.
นั่นคือ
$\bm{w}_i^C = [\bm{g}_i^C; \bm{c}_i^C; \bm{pos}_i^C; \bm{ner}_i^C; \bm{f}_i^C] \in \mathbb{R}^{300+600+12+8+1}$
และ
$\bm{w}_i^Q = [\bm{g}_i^Q; \bm{c}_i^Q] \in \mathbb{R}^{300+600}$
เมื่อ $\bm{g}$ แทน\textit{เวคเตอร์ทั่วไป}\cite{PSM2014a}.
ตัวแปร $\bm{c}$ แทน\textit{เวคเตอร์คำผ่านบริบท}\cite{McCannEtAl2017a}.
ตัวแปร $\bm{pos}$ แทน\textit{รหัสคำ}จาก\textit{ชนิดของคำ}.
ตัวแปร $\bm{ner}$ แทน\textit{รหัสคำ}จาก\textit{สัญญาชืื่ออัตตลักษณ์}
และ $\bm{f}$ แทน\textit{ค่าบรรทัดฐานของความถี่คำ}.

\paragraph{การทำความเข้าใจคำถาม.}
คำถามจะถูกกลั่นกรองออกมาเป็น\textit{มโนทัศน์ระดับล่าง} (low-level concept) $\{\bm{h}^{Ql}_i\}_{i=1,\ldots,n}$
และ\textit{มโนทัศน์ระดับสูง} (high-level concept) $\{\bm{h}^{Qh}_i\}_{i=1,\ldots,n}$
โดย\textit{แอลเอสทีเอ็มสองทิศทาง} (ฺBi-directional LSTM\cite{GravesSchmidhuber2005a}) ดังนี้
%
\begin{eqnarray}
\bm{h}^{Ql}_1, \ldots, \bm{h}^{Ql}_n 
&=& \mathrm{BiLSTM}(\bm{w}^Q_1, \ldots, \bm{w}^Q_n; \bm{\phi}_{Ql}),
\label{eq: opt fusion net hQ low}
\\
\bm{h}^{Qh}_1, \ldots, \bm{h}^{Qh}_n 
&=& \mathrm{BiLSTM}(\bm{h}^{Ql}_1, \ldots, \bm{h}^{Ql}_n; \bm{\phi}_{Qh}),
\label{eq: opt fusion net hQ high}
\end{eqnarray}
เมื่อ $\bm{\phi}_{Ql}$ และ 
$\bm{\phi}_{Qh}$ เป็นพารามิเตอร์ของ $\mathrm{BiLSTM}$ ซึ่งหาได้จาก\textit{กระบวนการฝึก}

\textit{ความเข้าใจคำถาม} ก็หาได้จาก
%
\begin{eqnarray}
\bm{U}^Q = \{\bm{u}^Q_1, \ldots, \bm{u}^Q_n \} = 
\mathrm{BiLSTM}(
[\bm{h}^{Ql}_1; \bm{h}^{Qh}_1], \ldots, 
[\bm{h}^{Ql}_n; \bm{h}^{Qh}_n]
; \bm{\phi}_{UQ})
\end{eqnarray}
เมื่อ $\bm{\phi}_{UQ}$ เป็นพารามิเตอร์ของ $\mathrm{BiLSTM}$ ซึ่งหาได้จาก\textit{กระบวนการฝึก}

\paragraph{การอ่านบริบทในระดับคำ.}
เพื่อเพิ่มความเกี่ยวข้องกับคำถามเข้าไปในบริบท
\textit{ฟิวชั่นเน็ต}เสริมคำของบริบท เป็น
$\bm{\tilde{w}}^C_i = [\bm{w}^C_i, e_i, \bm{\hat{g}}^C_i]$
เมื่อ
$e_i$ ระบุว่าคำที่ $i^{th}$ ในบริบทมีปรากฎอยู่ในคำถามหรือไม่
และ
$\bm{\hat{g}}^C_i$
แทน\textit{ความสนใจ} (attention).
เวคเตอร์ $\bm{\hat{g}}^C_i$ เป็นความสัมพันธ์\textit{ในระดับคำ}
ระหว่างคำในบริบท และคำในคำถาม ซึ่งหาได้จาก
%
\begin{eqnarray}
\bm{\hat{g}}^C_i &=& \sum_j \alpha_{ij} \bm{g}_j^Q,
\label{eq: opt fusion-net g hat alpha}
\\
\alpha_{ij} &\propto& \exp{(S(\bm{g}^C_i, \bm{g}^Q_j))}
\label{eq: opt fusion-net g hat alpha 2}
\end{eqnarray}
เมื่อ ฟังชั่นความสนใจ (attention function)
\begin{equation}
S(\bm{x},\bm{y}) = \mathrm{ReLU}(\bm{W} \bm{x})^T \mathrm{ReLU}(\bm{W} \bm{y})
\label{eq: opt funsion net attention function}
\end{equation}
โดย $\bm{W} \in \mathbb{R}^{300 \times 300}$ เป็นเมตริกซ์ของค่าน้ำหนักซึ่งค่าจะหาได้จาก\textit{กระบวนการฝึก}
และ $\mathrm{ReLU}([x_1, \ldots, x_k]) = [\mathrm{ReLU}(x_1), \ldots, \mathrm{ReLU}(x_k)]$ 
ซึ่ง $\mathrm{ReLU}(x) = \max\{0, x\}$.

\paragraph{การอ่านบริบทในระดับมโนทัศน์.}
ในทำนองเดียวกับมโนทัศน์ของคำถาม
มโนทัศน์ของบริบทหาได้จาก
%
\begin{eqnarray}
\bm{h}^{Cl}_1, \ldots, \bm{h}^{Cl}_m 
&=& \mathrm{BiLSTM}(\bm{\tilde{w}}^C_1, \ldots, \bm{\tilde{w}}^C_m; \bm{\phi}_{Cl}),
\label{eq: opt fusion net hC low}
\\
\bm{h}^{Ch}_1, \ldots, \bm{h}^{Ch}_m 
&=& \mathrm{BiLSTM}(\bm{h}^{Cl}_1, \ldots, \bm{h}^{Cl}_m; \bm{\phi}_{Ch}),
\label{eq: opt fusion net hC high}
\end{eqnarray}
เมื่อ $\bm{\phi}_{Cl}$ และ 
$\bm{\phi}_{Ch}$ เป็นพารามิเตอร์ของ $\mathrm{BiLSTM}$ ซึ่งหาได้จาก\textit{กระบวนการฝึก}
%
\textit{ฮวงและคณะ}\cite{HuangEtAl2017a}กำหนดขนาดให้ทุกๆ\textit{มโนทัศน์}เป็น $\bm{h} \in \mathbb{R}^{250}$. 

\paragraph{การหลอมรวมบริบทและคำถาม} (fusion)
เป็นการผสานข้อมูลบริบทในหลายๆระดับ เข้ากับข้อมูลคำถามในหลายๆระดับ.
\textit{ฮวงและคณะ}\cite{HuangEtAl2017a}เสนอ 
\textit{ประวัติคำ} (History of Word) สำหรับอ้างถึง ข้อมูลในหลายๆระดับ 
%
\textit{ฮวงและคณะ} กำหนด\textit{ประวัติคำ} $\bm{r}^C_i$ และ $\bm{r}^Q_i$ เป็น
%
\begin{eqnarray}
\bm{r}^C_i = [\bm{g}^C_i; \bm{c}^C_i; \bm{h}^{Cl}_i; \bm{h}^{Ch}_i]
\label{app fusion net HoW C}
\\
\bm{r}^Q_i = [\bm{g}^Q_i; \bm{c}^Q_i; \bm{h}^{Ql}_i; \bm{h}^{Qh}_i]
\label{app fusion net HoW Q} .
\end{eqnarray}

ทั้ง $\bm{r}^C_i, \bm{r}^Q_i \in \mathbb{R}^{1400}$.
สังเกตุ $\bm{g}^C_i$, $\bm{g}^Q_i$ และ $\bm{c}^C_i$, $\bm{c}^Q_i$ เป็น\textit{ลักษณะสำคัญ} (features) ของคำในตำแหน่งที่ $i^{th}$.
ส่วน $\bm{h}^{Cl}$, $\bm{h}^{Ql}$,
$\bm{h}^{Ch}$ และ $\bm{h}^{Qh}$
เป็น\textit{ลักษณะสำคัญ}ระดับล่าง
และระดับสูงในตำแหน่งที่ $i^{th}$.

จากนั้น\textit{ฟิวชั่นเน็ตหลอมรวม}\textit{ประวัติคำ}ของทั้งบริบทและคำถามเข้าด้วยกัน 
โดย
%
\begin{eqnarray}
\bm{\hat{h}}^{Cl}_i &=& \sum_j \alpha^l_{ij} \bm{h}^{Ql}_j
\label{eq: opt fusion net fusion l}
\\
\bm{\hat{h}}^{Ch}_i &=& \sum_j \alpha^h_{ij} \bm{h}^{Qh}_j
\label{eq: opt fusion net fusion h}
\\
\bm{\hat{u}}^C_i &=& \sum_j \alpha^u_{ij} \bm{u}^Q_j
\label{eq: opt fusion net fusion ul}
\end{eqnarray}
เรียก
$\bm{\hat{h}}^{Cl}_i$,
$\bm{\hat{h}}^{Ch}_i$
และ $\bm{\hat{u}}^C_i$
เป็น \textit{การหลอมรวมระดับล่าง}
\textit{การหลอมรวมระดับสูง}
และ
\textit{การหลอมรวมความเข้าใจ}
ตามลำดับ
เมื่อ
%
\begin{eqnarray}
\alpha^l_{ij} & \propto & \exp (S^l(\bm{r}^C_i, \bm{r}^Q_j))
\label{eq: opt funsion net alpha la},
\\
\alpha^h_{ij} & \propto & \exp(S^h(\bm{r}^C_i, \bm{r}^Q_j))
\label{eq: opt funsion net alpha lb},
\\
\alpha^u_{ij} & \propto & \exp(S^u(\bm{r}^C_i, \bm{r}^Q_j))
\label{eq: opt funsion net alpha lc}.
\end{eqnarray}
\textit{ฟังชั่นความสนใจ} $S$ นิยามในสมการ~\ref{eq: opt funsion net attention function}.
หมายเหตุ แต่ละ\textit{ฟังชั่นความสนใจ}จะมี\textit{ค่าน้ำหนัก}ที่ต้องเรียนรู้
นั่นคือ ในกระบวนการนี้จะมี
ค่าน้ำหนัก $\bm{W}_l, \bm{W}_h, \bm{W}_u$ ที่ได้จาก\textit{กระบวนการฝึก}.
\textit{ฟิวชั่นเน็ต}กำหนดขนาดของค่าน้ำหนักเหล่านี้เป็น $250 \times 1400$.

\paragraph{การทำความเข้าใจบริบทจากมุมมองของคำถาม.}
\textit{ตัวแทนบริบทที่ผ่านการหลอมรวมกับคำถาม} $\{\bm{v}^C_i\}_{i=1,\ldots,m}$ หาได้จาก
%
\begin{equation}
\{\bm{v}^C_1, \ldots, \bm{v}^C_m \}
=
\mathrm{BiLSTM}(
[\bm{h}_1^{Cl}; \bm{h}_1^{Ch};
\bm{\hat{h}}_1^{Cl}; \bm{\hat{h}}_1^{Ch};
\bm{\hat{u}}_1^C],
\ldots,
[\bm{h}_m^{Cl}; \bm{h}_m^{Ch};
\bm{\hat{h}}_m^{Cl}; \bm{\hat{h}}_m^{Ch};
\bm{\hat{u}}_m^C];
\bm{\phi}_v
)
\label{eq: opt fusion net v}
\end{equation}
เมื่อ $\bm{\phi}_v$ เป็นพารามิเตอร์ที่ได้จาก\textit{กระบวนการเรียนรู้}.

นอกจากนั้น  \textit{ฮวงและคณะ}ยังเพิ่มกลไก\textit{การหลอมรวมเสริมตนเอง}(self-boosted fusion) เข้าไปอีก โดยสร้าง\textit{ประวัติของคำรวมเสริมตน}เข้าไป
$\bm{s}^C_i = [\bm{g}^C_i; \bm{c}^C_i;
\bm{h}^{Cl}_i; \bm{h}^{Ch}_i;
\bm{\hat{h}}^{Cl}_i; \bm{\hat{h}}^{Ch}_i;
\bm{\hat{u}}^C_i; \bm{v}^C_i
] \in \mathbb{R}^{2400}$.
ทบทวน $\bm{g}^C_i, \bm{c}^C_i$ เป็นข้อมูลของบริบทระดับคำ.
$\bm{h}^{Cl}_i, \bm{h}^{Ch}_i$
เป็นข้อมูลของบริบทระดับล่างและระดับสูง.
$\bm{\hat{h}}^{Cl}_i, \bm{\hat{h}}^{Ch}_i,
\bm{\hat{u}}^C_i$
เป็นข้อมูลของบริบทที่ผ่านการหลอมรวมกับคำถามในระดับล่าง ระดับกลาง ระดับความเข้าใจ ตามลำดับ.
$\bm{v}^C_i$ เป็นข้อมูลความเข้าใจบริบท.

\textit{ค่าความสนใจที่ตระหนักรู้ตัวเต็มที่} (fully-aware attention)
ได้จากการหลอมรวม\textit{ความเข้าใจบริบท}และ\textit{ความสนใจ}ผ่านคำถามและการเสริมตนเข้าไป
นั่นคือ
$\bm{\hat{v}}^C_i = \sum_j \alpha^s_{ij} \bm{v}^C_j$
เมื่อ $\alpha^s_{ij} \propto \exp(S^s(\bm{s}^C_i, \bm{s}^C_j))$.
ในกระบวนการนี้จะมี\textit{ค่าน้ำหนัก} $\bm{W}_s$ ที่ต้องผ่าน\textit{กระบวนการฝึก}อยู่.
\textit{ความเข้าใจบริบททั้งหมด}จะหาได้จาก
\begin{equation}
\bm{U}^C = \{ \bm{u}^C_1, \ldots, \bm{u}^C_m \} = \mathrm{BiLSTM}([\bm{v}^C_1;\bm{\hat{v}}^C_1], \ldots, [\bm{v}^C_m;\bm{\hat{v}}^C_m]; \bm{\phi}_u)
\label{eq: opt fusion net context understanding}
\end{equation}
เมื่อ $\bm{\phi}_u$ เป็นพารามิเตอร์ที่ได้จาก\textit{กระบวนการเรียนรู้}
และ $\bm{u}^C_i \in \mathbb{R}^{250}$.

\paragraph{กระบวนการฝึก.}
จากกรอบปัญหาของ\textit{ฮวงและคณะ} และโครงสร้างของ\textit{ฟิวชั่นเน็ต}
มีค่าพารามิเตอร์ต่างๆ ที่ต้องหาในกระบวนการฝึก 
ได้แก่ $\bm{W}_B$, $\bm{W}_E$, $\bm{w}$,
$\bm{\phi}_{vq}$, $\bm{\phi}_{Ql}$, 
$\bm{\phi}_{Qh}$, $\bm{\phi}_{UQ}$,
$\bm{W}$, 
$\bm{\phi}_{Cl}$, $\bm{\phi}_{Ch}$,
$\bm{W}_l$, $\bm{W}_h$, $\bm{W}_u$,
$\bm{\phi}_v$, $\bm{W}_s$,
และ $\bm{\phi}_u$.
เพื่อความกระชับ พารามิเตอร์ทั้งหมดนี้จะถูกรวมเรียกเป็น $\bm{\theta}$.
\textit{ฮวงและคณะ}ฝึก\textit{ฟิวชั่นเน็ต} ด้วย
\begin{eqnarray}
\max_{\bm{\theta}}
 \sum_{i=1}^N \left( 
 \log (\tilde{P}^{\bm{\theta}}[b=j^b_i|\bm{C}_i, \bm{Q}_i])
 +
 \log (\tilde{P}^{\bm{\theta}}[e=j^e_i|\bm{C}_i, \bm{Q}_i])
 \right)
\end{eqnarray}
เมื่อ $N$ คือจำนวนตัวอย่างการฝึกทั้งหมด
และ
$\bm{C}_i$ กับ $\bm{Q}_i$ 
คือ บริบทและคำถามของตัวอย่างที่ $i^{th}$
และ
$j^b_i$ กับ $j^e_i$
คือ ตำแหน่งจุดเริ่มและจุดจบของคำตอบสำหรับตัวอย่างที่ $i^{th}$.

%LATER
%\section{การรู้จำใบหน้า (เฟซเน็ต)}.
%\textit{เฟซเน็ต} (FaceNet)
