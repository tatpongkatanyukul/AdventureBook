\section{แบบฝึกหัด}
\label{section: nlp exercises}

\begin{Parallel}[c]{0.5\textwidth}{0.45\textwidth}
	\selectlanguage{english}
	\ParallelLText{
		``If you talk to a man in a language he understands, that goes to his head. 
		If you talk to him in his language, that goes to his heart.''
		\begin{flushright}
			---Nelson Mandela
		\end{flushright}
	}
	\selectlanguage{thai}
	\ParallelRText{
		``ถ้าคุณคุยกับคนด้วยภาษาที่เขาฟังรู้เรื่อง สิ่งที่คุณพูดจะเข้าไปในหัวเขา.
		ถ้าคุณคุยกับเขาด้วยภาษาของเขา สิ่งที่คุณพูดจะเข้าไปในใจเขา.''
		\begin{flushright}
			---เนลสัน แมนเดลา
		\end{flushright}
	}
\end{Parallel}
\index{english}{words of wisdom!Nelson Mandela}
\index{english}{quote!native language}


\begin{Exercise}
	\label{ex: seq RNN cf ANN}
	
	จงอภิปรายถึง
	(1) แนวทางต่าง ๆ เพื่อประยุกต์ใช้โครงข่ายประสาทเทียม เช่น \textit{เพอร์เซปตรอนหลายชั้น}
	ที่ไม่มีกลไกการเวียนกลับ
	กับข้อมูลเชิงลำดับ
	พร้อมอภิปรายข้อดีและข้อเสีย สำหรับแนวทางต่าง ๆ ที่เสนอมา
	พร้อมยกตัวอย่างให้เห็นภาพ
	(2) ข้อดีและข้อเสียเปรียบเทียบกับการใช้โครงข่ายประสาทเทียมเวียนกลับ.
	
%	why RNN cf ANN
	
\end{Exercise}

\begin{Exercise}
	\label{ex: seq RNN cf CNN}
	
%	discussion
%	why RNN cf CNN

	จงอภิปรายถึง
(1) แนวทางต่าง ๆ เพื่อประยุกต์ใช้โครงข่ายคอนโวลูชัน ที่ไม่มีกลไกการเวียนกลับ
กับข้อมูลเชิงลำดับ
พร้อมอภิปรายข้อดีและข้อเสีย สำหรับแนวทางต่าง ๆ ที่เสนอมา
พร้อมยกตัวอย่างให้เห็นภาพ
(2) ข้อดีและข้อเสียเปรียบเทียบกับการใช้โครงข่ายประสาทเทียมเวียนกลับ.
	
\end{Exercise}


\begin{Exercise}
	\label{ex: seq RNN on 2D data}
	
	%	discussion
	
	จงอภิปรายถึง
	(1) แนวทางต่าง ๆ เพื่อประยุกต์ใช้โครงข่ายประสาทเวียนกลับ
	กับข้อมูลที่มีความสัมพันธ์เชิงท้องถิ่นหลายมิติ เช่น ข้อมูลภาพ
	พร้อมอภิปรายข้อดีและข้อเสีย สำหรับแนวทางต่าง ๆ ที่เสนอมา
	พร้อมยกตัวอย่างให้เห็นภาพ
	(2) ข้อดีและข้อเสียเปรียบเทียบกับการใช้โครงข่ายคอนโวลูชัน.
	
\end{Exercise}


\begin{figure}
	\begin{center}		
		\includegraphics[height=2in]{08seq/RNN/unfolding_MT.png}	
		\caption[แผนภาพคลี่ลำดับของสถาปัตยกรรมตัวเข้ารหัสตัวถอดรหัส]{แผนภาพคลี่ลำดับของสถาปัตยกรรมตัวเข้ารหัสตัวถอดรหัส.
			สถาปัตยกรรมตัวเข้ารหัสตัวถอดรหัส โดยใช้เอาต์พุตจากตัวเข้ารหัส นำมาเป็นอินพุตสำหรับตัวถอดรหัส.
		}
		\label{fig: rnn unfolding diagram encoder-decoder}
	\end{center}
\end{figure}
%

\begin{Exercise}
	\label{ex: seq naive encoder-decoder}
	
	รูป~\ref{fig: rnn unfolding diagram encoder-decoder} แสดงการจัดโครงสร้างสถาปัตยกรรมตัวเข้ารหัสตัวถอดรหัส
	โดยใช้เอาต์พุตจากตัวเข้ารหัส นำมาเป็นอินพุตสำหรับตัวถอดรหัส
	
	จงอภิปราย ข้อดี ข้อเสีย และปัญหาของการจัดโครงสร้างแบบนี้ เปรียบเทียบแบบใช้สถานะภายในเป็นรหัส (รูป~\ref{fig: rnn unfolding diagram encoder-decoder coded hidden state})


%หมายเหตุ แม้ว่าในรูป~\ref{fig: rnn unfolding diagram encoder-decoder} จะแสดงเอาต์พุตจากตัวเข้ารหัส ซึ่งในรูปคือ $\bm{y}'(T_x)$ ไปเป็นอินพุตลำดับแรกของตัวถอดรหัส.
%ในทางปฏิบัติ
%\textit{สถาปัตยกรรมตัวเข้ารหัสตัวถอดรหัส} อาจทำงานได้ดีกว่า
%หากจัดโครงสร้างดังแสดงในรูป~\ref{fig: rnn unfolding diagram encoder-decoder coded hidden state}.
	
\end{Exercise}



\begin{Exercise}
\label{ex: seq encoder-decoder configuration}

นอกจาก การจัดโครงสร้างดังแสดงในรูป~\ref{fig: rnn unfolding diagram encoder-decoder coded hidden state}
แล้วสถาปัตยกรรมตัวเข้ารหัสตัวถอดรหัส
อาจผ่าน\textit{รหัสข้อความ} ไปเป็นส่วนหนึ่งของอินพุตของตัวถอดรหัสได้ (รูปแบบที่สอง)
หรือ แม้แต่จะผ่าน\textit{รหัสข้อความ} ไปเป็นทั้งสถานะเริ่มต้น และส่วนหนึ่งของอินพุต
ดังแสดงในรูป~\ref{fig: nlp encoder-decoder without attention}
(รูปแบบที่สาม)

จงอภิปราย ข้อดี ข้อเสีย การจัดโครงสร้างแบบต่าง ๆ นี้
รวมถึงอภิปรายปัจจัยที่เกี่ยวข้อง สถานการณ์ที่บางรูปแบบอาจทำงานได้ดีกว่า
พร้อมออกแบบการทดลอง ดำเนินการทดลอง และนำเสนอผล.

%
%see Goodfellow 10.4
%
%1. code --> z(0)
%2. code --> x(1), x(2), ...
%3. both: code --> z(0), x(1), x(2), ...

\end{Exercise}


\begin{Exercise}
	\label{ex: nlp sentiment analysis}
\index{thai}{การประมวลผลภาษาธรรมชาติ!การจำแนกอารมณ์}
\index{english}{NLP!sentiment analysis}	
	
จงศึกษาภารกิจการจำแนกอารมณ์ แนวทางปฏิบัติ การวัดผล และข้อมูลที่นิยม
และสร้างระบบการจำแนกอารมณ์ พร้อมประเมินผล.
	
\end{Exercise}


\begin{Exercise}
	\label{ex: nlp POS}
\index{thai}{การประมวลผลภาษาธรรมชาติ!การระบุหมวดคำ}
\index{english}{NLP!POS tagging}	
	
จงศึกษาภารกิจการระบุหมวดคำ แนวทางปฏิบัติ การวัดผล และข้อมูลที่นิยม
และสร้างระบบการจำแนกอารมณ์ พร้อมประเมินผล.
	
	
\end{Exercise}



\begin{Exercise}
	\label{ex: seq RNN cf CNN wavenet}
	
	จงศึกษาการทำงานของระบบสังเคราะห์เสียง เช่น เวฟเน็ต (Wavenet\cite{Wavenet})
	อภิปรายถึงแนวทางและวิธีที่ใช้ รวมถึงการประเมินผลและข้อมูล.
			
\end{Exercise}

\begin{Exercise}
	\label{ex: seq encoder-decoder code length}
	
	จงศึกษาสถาปัตยกรรมตัวเข้ารหัสตัวถอดรหัส (อาจเริ่มจากบทความที่ทรงอิทธิพล\cite{ChoEtAl2014b, SutskeverEtAl2014})
	ออกแบบการทดลอง เพื่อศึกษาปัจจัยความยาวของชุดลำดับข้อมูลกับประสิทธิภาพของระบบ
	ดำเนินการทดลอง รายงานผล และสรุป.
		
\end{Exercise}

\begin{Exercise}
	\label{ex: nlp MT address limit sentence length}
	
จงรวมกลุ่มระดมความคิด และอภิปรายแนวทางที่จะแก้ปัญหาระบบแปลภาษา 
เพื่อแก้ปัญหาคอขวดในสถาปัตยกรรมตัวเข้ารหัสตัวถอดรหัส

หมายเหตุ แบบฝึกหัดนี้ ต้องการฝึกการคิดเชิงวิพากษ์ และฝึกความคิดสร้างสรรค์
อีกทั้งจะทำให้เห็นคุณค่าของกลไกความใส่ใจด้วย
แต่เพื่อให้แบบฝึกหัดนี้ บรรลุจุดประสงค์ได้ ควรทำแบบฝึกหัดนี้ก่อนที่จะศึกษาเรื่องกลไกความใส่ใจ
หรือหากได้ศึกษาเรื่องกลไกความใส่ใจไปแล้ว อาจจะลองเปิดใจมองหาแนวทางอื่น ๆ ที่อาจจะช่วยบรรเทาปัญหานี้ได้.

\end{Exercise}

\begin{Exercise}
	\label{ex: nlp Attention and Transformer}
	
	จงศึกษาพัฒนาการและกลไกที่สำคัญของศาสตร์แบบจำลองชุดข้อมูลลำดับ
	จากโครงข่ายประสาทเวียนกลับ แบบจำลองความจำระยะสั้นที่ยาว
	สถาปัตยกรรมตัวเข้ารหัสตัวถอดรหัส กลไกความใส่ใจ และตัวแปลงชนิดต่าง ๆ (เช่น Transformer, BERT, GPT-3)
สรุปประเด็น แนวทางทีี่สำคัญ ลักษณะปัญหา และการประยุกต์ใช้ รวมถึงช่วงเวลา	
และอภิปราย เหตุผล แรงขับดันเบื้องหลังพัฒนาการเหล่านี้.
	
\end{Exercise}

\begin{Exercise}
	\label{ex: nlp overview}

จงศึกษาศาสตร์การประมวลผลภาษาธรรมชาติ ในเชิงกว้าง
ถึงภารกิจต่าง ๆ ที่สำคัญ และบริบทในแง่ความต้องการของสังคม
รวมถึง ความก้าวหน้าในแต่ละภารกิจ เมื่อเทียบกับจุดมุ่งหมาย
และอภิปรายโอกาสการประยุกต์ใช้ต่าง ๆ ที่อาจจะนอกเหนือจากประยุกต์ใช้ดั่งเดิม
และความท้าทายต่าง ๆ ในงานวิจัย เพื่อบรรลุจุดประสงค์
รวมถึงเชื่อมโยงสิ่งที่ได้เรียนรู้ ความก้าวหน้า การประยุกต์ใช้ จากบริบทในแง่ความต้องการของสังคม.

\end{Exercise}


\begin{Exercise}
	\label{ex: nlp open task}
	
	จงเลือกภารกิจการประมวลผลภาษาธรรมชาติที่สนใจ
	และศึกษาภารกิจ แนวทางปฏิบัติ ปัจจัยที่เกี่ยวข้อง การวัดผล และข้อมูลที่นิยม
	และทดลองสร้างระบบสำหรับภารกิจที่เลือก ประเมินผล ให้ความเห็น และสรุปสิ่งที่ได้เรียนรู้.
	
\end{Exercise}
