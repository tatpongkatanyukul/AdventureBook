\section{แบบฝึกหัด}

\begin{Parallel}[c]{0.48\textwidth}{0.45\textwidth}
	\selectlanguage{english}
	\ParallelLText{
		``Try not to become a man of success,\\
		but rather try to become a man of value.''
		\begin{flushright}
			---Albert Einstein
		\end{flushright}
	}
	\selectlanguage{thai}
	\ParallelRText{
		``อย่าพยายามเป็นคนประสบความสำเร็จ\\
		แต่ให้พยายามเป็นคนที่มีคุณค่า.''
		\begin{flushright}
			---อัลเบิร์ต ไอน์สไตน์
		\end{flushright}
	}
\end{Parallel}
\index{english}{words of wisdom!Albert Einstein}
\index{english}{quote!value}
\vspace{1cm}

%
%\begin{Parallel}[c]{0.52\textwidth}{0.44\textwidth}
%	\selectlanguage{english}
%	\ParallelLText{
%		``Try not to become a man of success,
%		but rather try to become a man of value.''\\
%		---Albert Einstein
%	}
%	\selectlanguage{thai}
%	\ParallelRText{
%		``อย่าพยายามเป็นคนประสบความสำเร็จ
%		แต่ให้พยายามเป็นคนที่มีคุณค่า'' \\
%		---อัลเบิร์ต ไอน์สไตน์
%	}
%\end{Parallel}
%\index{words of wisdom!Albert Einstein}
%\index{quote!value}


%\begin{Exercise}
%	\label{ex: opencv read}
%	
%\end{Exercise}
%
%\begin{Exercise}
%	\label{ex: opencv write}
%	
%\end{Exercise}
%
%\begin{Exercise}
%	\label{ex: opencv vdo}
%	
%\end{Exercise}
%
%\begin{Exercise}
%	\label{ex: hog}
%	
%\end{Exercise}

%LATER
%\begin{Exercise}
%	\label{ex: sliding window}
%	\index{sliding window}
%	\index{วิธีหน้าต่างเลื่อน}
%	
%\end{Exercise}

%LATER
%\begin{Exercise}
%	\label{ex: simple obj det}
%	\index{การตรวจจับภาพวัตถุ}
%	\index{object detection}
%	
%\end{Exercise}

\begin{Exercise}
	\label{ex: kde}
	\index{english}{KDE}
	\index{english}{Kernel Density Estimation}
	\index{thai}{วิธีการประมาณความหนาแน่นแก่น}

จงสร้างข้อมูลหนึ่งมิติขึ้นมา 
(อาจใช้คำสั่ง เช่น \verb|np.random.normal|)
และใช้\textit{วิธีการประมาณความหนาแน่นแก่น} (สมการ~\ref{eq: KDE shell} และ~\ref{eq: KDE kernel})
เพื่อประมาณความหนาแน่นความน่าจะเป็น จากข้อมูลนั้น
โดยทดลองค่า $\sigma$ หลาย ๆ ค่า.
สังเกตผล อภิปราย และสรุป.

รายการ~\ref{code: kde} แสดงโปรแกรมวิธีการประมาณความหนาแน่นแก่น.
ตัวอย่างการเรียกใช้ เช่น
\begin{Verbatim}[fontsize=\small]
datax = np.random.normal(3, 2, 100).reshape((1,-1))
xs = np.linspace(-2, 8, 50).reshape((1,-1))
pdx = kde(xs, datax, sigma=1)
plt.plot(xs[0,:], pdx[0,:], 'k')
\end{Verbatim}
บรรทัดแรกเป็นคำสั่งเพื่อสร้างข้อมูล \verb|datax| ขึ้นมา จากการแจกแจงแบบเกาส์เซียน จำนวน $100$ จุดข้อมูล โดยมีค่าเฉลี่ยเป็น $3$ และค่าเบี่ยงเบนมาตราฐานเป็น $2$.
บรรทัดที่สอง สร้างค่า $x$ ที่ต้องการถาม และบรรทัดที่สาม คือการเรียกโปรแกรม \verb|kde| เพื่อประมาณความหนาแน่นความน่าจะเป็นของข้อมูล \verb|datax|.
บรรทัดสุดท้ายเป็นการวาดกราฟแสดงความหนาแน่นที่ค่าอินพุตต่าง ๆ.
%
ดูตัวอย่างจากรูป~\ref{fig: kde sigma's}.

\lstinputlisting[language=Python, caption={[วิธีการประมาณความหนาแน่นแก่น]โปรแกรมวิธีการประมาณความหนาแน่นแก่น}, label={code: kde}]{04Classic/code/code_kde.py}
\index{english}{kernel density estimation!code}
\index{thai}{วิธีการประมาณความหนาแน่นแก่น!โปรแกรม}

\end{Exercise}


\begin{Exercise}
	\label{ex: kde 2D}
	\index{english}{KDE}
	\index{english}{Kernel Density Estimation}
	\index{thai}{วิธีการประมาณความหนาแน่นแก่น}
	
	จงสร้างข้อมูลสองมิติขึ้นมา 
	โดยอาจใช้คำสั่ง เช่น
\begin{Verbatim}[fontsize=\small]
mu = [6, 9]
cov = [[8, -5], [-5, 9]]
x = np.random.multivariate_normal(mu, cov, 500).reshape((2,-1))
\end{Verbatim}
เมื่อ \verb|mu| และ \verb|cov| แทนค่าเฉลี่ยและค่าความแปรปรวนร่วมเกี่ยว ตามลำดับ
และสร้างเป็นข้อมูลสองมิติจำนวน $500$ จุดข้อมูล.
และใช้\textit{วิธีการประมาณความหนาแน่นแก่น} (สมการ~\ref{eq: KDE shell} และ~\ref{eq: KDE kernel})
เพื่อประมาณความหนาแน่นความน่าจะเป็น จากข้อมูลนั้น
โดยทดลองค่า $\sigma$ หลาย ๆ ค่า.
สังเกตผล อภิปราย และสรุป.

รูป~\ref{fig: kde 2D} แสดงตัวอย่างข้อมูลสองมิติที่สร้างขึ้น และค่าความหนาแน่นที่ประมาณออกมา.
%
\begin{figure}[H]
	\begin{center}
	\begin{tabular}{cc}
	\includegraphics[width=0.48\columnwidth]{04Classic/kde/data2D.png}
&
	\includegraphics[width=0.48\columnwidth]{04Classic/kde/kde_2D.png}
	\end{tabular}
	\end{center}
	\caption[ตัวอย่างการประมาณค่าความหนาแน่นความน่าจะเป็นสำหรับข้อมูลสองมิติ]{ตัวอย่างการประมาณค่าความหนาแน่นความน่าจะเป็นสำหรับข้อมูลสองมิติ.
ภาพซ้าย แสดงจุดข้อมูลตัวอย่าง.
ภาพขวา แสดงค่าประมาณความหนาแน่นความน่าจะเป็นของข้อมูล.
ระดับสีแทนค่าความหนาแน่นความน่าจะเป็น ซึ่งค่าแสดงด้วยแถบสีด้านข้าง.
	}
	\label{fig: kde 2D}
\end{figure}
	
\end{Exercise}

%LATER
%\begin{Exercise}
%	\label{ex: heatmap}
%	\index{heatmap}
%	\index{แผนที่ความร้อน}
%	
%In practice, it is more convenient for marketing personnel to be able to adjust a color scheme so that some ranges of visiting frequencies become more striking at desired degrees.
%Instead of directly changing a color scheme,
%this can be achieved easily by introducing another parameter to globally manipulate heat values.
%Then, the manipulated heat values can be mapped on a same color scheme, but the resulting hot zone map appears as if it is produced on a different mapping color scale. 
%A manipulated heat value is called ``heat intensity.''
%One simple manipulation is to power a normalized heat value to a fraction of the manipulation parameter.
%That is, given heat value $z$ and parameter $u$, heat intensity is calculated by, 
%\begin{eqnarray}
%z' = \left( \frac{z - z_{\min}}{z_{\max} - z_{\min}} \right)^{\frac{1}{u}}
%\label{eq: hot zone intensity}.
%\end{eqnarray}
%
%Fig.~\ref{fig: hot zone map} shows examples of hot zone maps produced from the same heat values, but different values of parameter $u$.
%It should be noted that using $u > 1$ leads to a heat intensification effect, which allows lower visiting frequencies to be more noticeable.
%The left most picture shows a hot zone map without intensification ($u=1$).
%Without intensification, only the most frequently visited area, which is around cashier counter, is noticeable.
%This is trivial and provides virtually no marketing insight.
%With different degrees of intensification, the second most and other less frequently visited areas can be identified and examined, as shown in other pictures ($u=2, 3, 5, 8, 10$).
%
%\begin{figure}[H]
%\begin{center}
%	\begin{tabular}{cccccc}
%		\includegraphics[width=0.15\textwidth]{04Classic/GTheatmap1b.png} &
%		\includegraphics[width=0.15\textwidth]{04Classic/GTheatmap2b.png} &
%		\includegraphics[width=0.15\textwidth]{04Classic/GTheatmap3b.png} &
%		\includegraphics[width=0.15\textwidth]{04Classic/GTheatmap5b.png} &
%		\includegraphics[width=0.15\textwidth]{04Classic/GTheatmap8b.png} &
%		\includegraphics[width=0.15\textwidth]{04Classic/GTheatmap10b.png}
%		%
%		\\
%		$u = 1$ &
%		$u = 2$ &
%		$u = 3$ &
%		$u = 5$ &
%		$u = 8$ &
%		$u = 10$ 
%	\end{tabular} 
%\end{center}
%	\caption{Hot zone maps at various intensities.}
%	\label{fig: hot zone map}
%\end{figure}
%	
%\end{Exercise}


\begin{Exercise}
	\label{ex: svm primal}
	\index{english}{SVM}
	\index{english}{Support Vector Machine}
	\index{thai}{ซัพพอร์ตเวกเตอร์แมชชีน}
แบบฝึกหัดนี้ เราจะศึกษา\textit{รูปปฐม}ของซัพพอร์ตเวกเตอร์แมชชีน.
จงสร้างข้อมูลขึ้นมาจำนวน $200$ จุดข้อมูล โดยเป็นกลุ่มบวกและกลุ่มลบอย่างละครึ่ง
จุดข้อมูลอยู่ในปริภูมิสองมิติ และข้อมูลสามารถแบ่งแยกได้อย่างสมบูรณ์เชิงเส้น
เช่น ข้อมูลที่แสดงในรูป~\ref{fig: linearly separable datapoints}.
แล้วแก้ปัญหาใน\textit{รูปปฐม}ของซัพพอร์ตเวกเตอร์แมชชีน 
เพื่อหาอภิระนาบแบ่งแยก นั่นคือ หาค่าพารามิเตอร์ $\bm{w}$ และ $b$
ดังแสดงในรูป~\ref{fig: linearly separable primal gd}.

ทบทวนปัญหาปฐม สำหรับข้อมูลที่สามารถแบ่งแยกได้โดยสมบูรณ์เชิงเส้น
คือ
\begin{eqnarray}
\underset{\bm{w}, b}{\mathrm{minimize}} &  \frac{1}{2} \bm{w}^T \bm{w} &
\nonumber \\
\mbox{s.t.} 
& y_i (\bm{w}^T \bm{x}_i + b) \geq 1 & \mbox{ for } i =1, \ldots, N.
%\label{eq: svm primal separable}
\nonumber
\end{eqnarray}

หมายเหตุ ปัญหานี้ เราจะใช้ฟังก์ชันเอกลักษณ์เป็นลักษณะสำคัญ นั่นคือ $\bm{z} = \phi(\bm{x}) = \bm{x}$
(ซึ่งเทียบเท่าการใช้เคอร์เนลเชิงเส้น $k(\bm{x}, \bm{x}') = \bm{x}^T \bm{x}'$).

เราอาจสามารถแก้ปัญหาได้ด้วย\textit{วิธีการลงโทษ}
เช่นอาจกำหนด\textit{ฟังก์ชันลงโทษ} เป็น
\[
P(\bm{w}, b) = \sum_i \mathrm{relu}\left(1 - y_i \cdot (\bm{w}^T \bm{x}_i + b)\right)
\]
เมื่อ 
\begin{eqnarray}
\mathrm{relu}(a) = \left\{ 
\begin{array}{ll}
a & \mbox{ เมื่อ } a \geq 0, \\
0 & \mbox{ หากเป็นกรณีอื่น} .
\end{array}
\right.
\nonumber
\end{eqnarray}
\index{english}{relu}
\index{thai}{เรลู}

จากปัญหาปฐมของซัพพอร์ตเวกเตอร์แมชชีน ตัวอย่างคำสั่งข้างล่าง
\begin{Verbatim}[fontsize=\small]
la = 100
loss_adaptor = lambda wb: loss(wb[:2], wb[2], la)[0,0]
dloss_adaptor = lambda wb: dloss(wb[:2], wb[2], la)
wb0 = np.zeros((3,1))
wbo, gd_losses, wbs = gd(dloss_adaptor, wb0, loss_adaptor, 
    step_size = 0.0001, Nmax = 5000)
\end{Verbatim}
ใช้ค้นหาอภิระนาบ (ระบุด้วย $\bm{w}$ และ $b$ ซึ่งคือ \verb|wbo[:2]| และ \verb|wbo[2]| ตามลำดับ).
โปรแกรม \verb|loss| และ \verb|dloss| 
รวมถึงโปรแกรมอื่นที่เกี่ยวข้องกำหนดดังแสดงในรายการ~\ref{code: svm primal gd}.
โปรแกรม \verb|gd| (แสดงในรายการ~\ref{code: gd as function}) คำนวณวิธีลงเกรเดียนต์.

\lstinputlisting[language=Python, caption={[โปรแกรมค้นหาอภิระนาบ ในปัญหาปฐมของซัพพอร์ตเวกเตอร์แมชชีน]ตัวอย่างโปรแกรมการค้นหาอภิระนาบ จากปัญหาปฐมของซัพพอร์ตเวกเตอร์แมชชีน ด้วยวิธีลงเกรเดียนต์. 
ตัวแปร \texttt{datax} และ \texttt{datay} แทนข้อมูลอินพุตและฉลากกลุ่ม ตามลำดับ
โดยทั้งคู่เป็น \texttt{np.array} สัดส่วน \texttt{(2,N)} และ \texttt{(1,N)} เมื่อ \texttt{N} เป็นจำนวนจุดข้อมูล.
หมายเหตุ ตัวแปร \texttt{datax} และ \texttt{datay} เป็น\textit{ตัวแปรส่วนกลาง} (global variables).
ดูแบบฝึกหัด~\ref{ex: svm primal scipy.optimize} สำหรับตัวอย่างการทำเป็น\textit{โปรแกรมเชิงวัตถุ}
ซึ่งมีการจัดการข้อมูลที่เป็นสัดเป็นส่วนมากกว่า.
}, label={code: svm primal gd}]{04Classic/code/code_primal_gd.py}
\index{english}{support vector machine!primal!code}
\index{thai}{ซัพพอร์ตเวกเตอร์แมชชีน!ปัญหาปฐม!โปรแกรม}
\index{english}{relu!code}
\index{thai}{เรลู!โปรแกรม}


\lstinputlisting[language=Python, caption={โปรแกรมวิธีลงเกรเดียนต์}, label={code: gd as function}]{02Background/code/code_gd_fn.py}
\index{english}{gradient descend method!code}
\index{thai}{วิธีลงเกรเดียนต์!โปรแกรม}

\begin{figure}[H]
	\begin{center}
\includegraphics[width=0.5\textwidth]{04Classic/svm/ex_primal_separable.png}
	\end{center}
	\caption[ตัวอย่างข้อมูลที่สามารถแบ่งแยกได้โดยสมบูรณ์เชิงเส้น]{ตัวอย่างข้อมูลที่สามารถแบ่งแยกได้โดยสมบูรณ์เชิงเส้น.
	}
	\label{fig: linearly separable datapoints}
\end{figure}

\begin{figure}[H]
	\begin{center}
		\includegraphics[width=0.5\textwidth]{04Classic/svm/ex_primal_separable_GD.png}
	\end{center}
	\caption[ตัวอย่างอภิระนาบที่หาได้จากรูปปฐม]{ตัวอย่างอภิระนาบที่หาได้จากรูปปฐม สำหรับข้อมูลที่สามารถแบ่งแยกได้โดยสมบูรณ์เชิงเส้น.
	เส้นทึบสีน้ำเงิน แสดงอภิระนาบที่หาได้.
	}
	\label{fig: linearly separable primal gd}
\end{figure}

\end{Exercise}

\paragraph{มอดูลไซไพ.}
หัวข้อนี้ แนะนำมอดูล\textit{ไซไพ} (Scipy) ซึ่งมีเครื่องมือหลายอย่างสำหรับงานคำนวณทางวิทยาศาสตร์
รวมถึงมอดูล \verb|optimize|. 
มอดูล \verb|optimize| มีเครื่องมือการหาค่าดีที่สุดอยู่หลายวิธี.
แม้วิธีลงเกรเดียนต์ (หัวข้อ~\ref{sec: opt grad desc}) เป็นวิธีการที่ใช้งานได้
แต่ในทางปฏิบัติ มีวิธีที่มีประสิทธิภาพมากกว่าวิธีลงเกรเดียนต์อยู่มากมาย
และด้วยแบบจำลอง \verb|optimize| เราสามารถไปใช้วิธีเหล่านั้นได้ โดยไม่ต้องใช้เวลาในการศึกษารายละเอียดของวิธีเหล่านั้นมากนั้น.
มอดูล \verb|optimize| สามารถนำเข้าได้ด้วยคำสั่ง
\begin{Verbatim}[fontsize=\small]
from scipy import optimize
\end{Verbatim}
แบบฝึกหัด~\ref{ex: svm primal scipy.optimize} จะแก้ปัญหาเดียวกับแบบฝึกหัด~\ref{ex: svm primal} เพียงแต่จะใช้เครื่องมือจากแบบจำลอง \verb|optimize| แทนวิธีลงเกรเดียนต์.

\begin{Exercise}
	\label{ex: svm primal scipy.optimize}
	\index{english}{SVM}
	\index{english}{Support Vector Machine}
	\index{thai}{ซัพพอร์ตเวกเตอร์แมชชีน}

เช่นเดียวกับแบบฝึกหัด~\ref{ex: svm primal}
จงสร้างข้อมูลขึ้นมาจำนวน $200$ จุดข้อมูล โดยเป็นกลุ่มบวกและกลุ่มลบอย่างละครึ่ง
จุดข้อมูลอยู่ในปริภูมิสองมิติ และข้อมูลสามารถแบ่งแยกได้อย่างสมบูรณ์เชิงเส้น
แล้วแก้ปัญหาใน\textit{รูปปฐม}ของซัพพอร์ตเวกเตอร์แมชชีน 
เพื่อหาอภิระนาบแบ่งแยก.

ศึกษาการเขียนโปรแกรมด้วยการใช้มอดูล \verb|optimize| (ดังแสดงในตัวอย่างของแบบฝึกหัดนี้) เปรียบเทียบกับด้วยวิธีลงเกรเดียนต์ (แบบฝึกหัด~\ref{ex: svm primal}) 
ทดลองใช้งาน สังเกตผล อภิปราย และสรุป.

ตัวอย่างคำสั่งข้างล่าง ฝึกแบบจำลอง และรายงานผลอภิระนาบที่พบ
\begin{Verbatim}[fontsize=\small]
svm = primal_SVM()
svm.phi = lambda xq: xq # identity projection
res = svm.train(datax, datay)
print('w=', svm.wopt.T)
print('b=', svm.bopt)
\end{Verbatim}
โดย
บรรทัดแรก เป็นการสร้างตัวแปรวัตถุ \verb|svm|
จาก\textit{คลาส} \verb|primal_SVM| ซึ่งแสดงในรายการ~\ref{code: primal SVM}.
บรรทัดที่สอง กำหนดฟังก์ชันลักษณะสำคัญเป็นฟังก์ชันเอกลักษณ์.
บรรทัดที่สาม เป็นการฝึกซัพพอร์ตเวกเตอร์แมชชีน ด้วยข้อมูล \verb|datax| และ \verb|datay|.
ผลลัพธ์ที่ได้จากการฝึกจะสรุปออกมาเป็นค่าของพารามิเตอร์ \verb|svm.wopt| และ \verb|svm.bopt| ที่ใช้บรรยายอภิระนาบ.
สังเกตว่า \textit{เมท็อด} \verb|train| มีการเรียกใช้
\begin{Verbatim}[fontsize=\small]
res = optimize.minimize(minf, wb0, method='SLSQP', jac=gradf,
          constraints=ineq_cons, options=options)
\end{Verbatim}
ซึ่งเป็นเครื่องมือการหาค่าดีที่สุด โดยระบุวิธีเป็น \verb|'SLSQP'| (ศึกษารายละเอียดเพิ่มเติมจากเวปไซต์ของ\textit{ไซไพ} หากสนใจ)
โดยภายใน \textit{เมท็อด} \verb|train| 
ได้กำหนดค่า \verb|minf| และ \verb|gradf|
ซึ่งคือ ฟังก์ชันจุดประสงค์ และฟังก์ชันเกรเดียนต์ ตามลำดับ
พร้อมทั้ง\textit{ข้อจำกัด}แบบอสมการ \verb|ineq_cons|.

หมายเหตุ การใช้งานจริง สิ่งที่ต้องการ คือการทำนายกลุ่มของข้อมูล.
\textit{เมท็อด} \verb|decision_score| ใช้คำนวณค่า\textit{ฟังก์ชันแบ่งแยก}
ซึ่งจะนำไปใช้ตัดสินกลุ่ม.
ส่วนค่าของ $\bm{w}$ และ $b$ นั้น เป็นรายละเอียดภายใน ไม่จำเป็นต้องรายงาน
และการใช้งานจริงของซัพพอร์ตเวกเตอร์แมชชีน ก็จะไม่มีการคำนวณค่า $\bm{w}$ ออกมา
เพราะว่า รูปแบบการคำนวณถูกแปลงไป เพื่อใช้ประโยชน์จากฟังก์ชันเคอร์เนล.
ดูแบบฝึกหัด~\ref{ex: svm dual} สำหรับโปรแกรมซัพพอร์ตเวกเตอร์แมชชีนที่ใช้งานจริง.

\lstinputlisting[language=Python, caption={[ซัพพอร์ตเวกเตอร์แมชชีน ปัญหาปฐม กรณีแบ่งแยกได้โดยสมบูรณ์]โปรแกรมซัพพอร์ตเวกเตอร์แมชชีน จากปัญหาปฐม สำหรับกรณีแบ่งแยกได้โดยสมบูรณ์}, label={code: primal SVM}]{04Classic/code/primal_SVM.py}
\index{english}{support vector machine!primal!code}
\index{thai}{ซัพพอร์ตเวกเตอร์แมชชีน!ปัญหาปฐม!โปรแกรม}

\end{Exercise}

\begin{Exercise}
	\label{ex: svm dual}
	\index{english}{SVM}
	\index{english}{Support Vector Machine}
	\index{thai}{ซัพพอร์ตเวกเตอร์แมชชีน}

ปัญหาเดียวกับแบบฝึกหัด~\ref{ex: svm primal scipy.optimize}
แต่แก้ปัญหาจาก\textit{ปัญหาคู่}.
\begin{eqnarray}
\underset{\bm{\alpha}}{\mathrm{maximize}} &  \sum_{i=1}^N \alpha_i 
- \frac{1}{2}\sum_{i=1}^N \sum_{j=1}^N \alpha_i \alpha_j y_i y_j k(\bm{x}_i, \bm{x}_j) 
\nonumber \\
\mbox{s.t.} & 
\nonumber \\
&
\begin{array}{ll}
\sum_{i=1}^N \alpha_i y_i = 0, &  \\
0 \leq \alpha_i \leq C & \mbox{ สำหรับ } i =1, \ldots, N.
\end{array}
\nonumber 
\end{eqnarray}
เนื่องจาก \textit{ปัญหาคู่} สำหรับกรณีแบ่งแยกได้โดยสมบูรณ์ กับกรณีทั่วไปต่างกัน เพียงข้อกำหนด $\alpha_i \leq C$ ซึ่งหากเลือกค่า $C$ ขนาดใหญ่ก็จะให้ผลแบบเดียวกับกรณีแบ่งแยกได้โดยสมบูรณ์
ตัวอย่างข้างล่างนี้ จึงใช้รูปแบบที่มีอภิมานพารามิเตอร์ $C$.

โปรแกรมในรายการ~\ref{code: cSVM}
แสดงตัวอย่างโปรแกรมซัพพอร์ตเวกเตอร์แมชชีน.
สังเกต วิธีการเขียนโปรแกรมจะใช้การทำ\textit{เวคตอไรเซชั่น}มากที่สุดเท่าที่จะทำได้
เนื่องจากประสิทธิภาพการคำนวณและความยืดหยุ่น.
ตัวอย่าง เช่น 
หากการคำนวณค่าพารามิเตอร์ $b$ คือ
$b_o = y_i  - \sum_{j \in S} \alpha_j y_j k(\bm{x}_j, \bm{x}_i)$ สำหรับ $i \in \{j: 0 < \alpha_j < C \}$.
นั่นคือ เทียบเท่า
$b_o = y_i - (\bm{\alpha}^T \odot \bm{y}) \cdot \bm{K}_{:,i}$ สำหรับ $i \in \{j: 0 < \alpha_j < C \}$.
หมายเหตุ โปรแกรม~\ref{code: cSVM} คำนวณค่า $b$ ด้วยค่าเฉลี่ย ซึ่งจะซับซ้อนกว่าตัวอย่างการทำ\textit{เวคตอไรเซชั่น}ที่อภิปรายข้างต้นเล็กน้อย.

%นอกจากนั้น สังเกตว่า โปรแกรมในรายการ~\ref{code: cSVM} ไม่มีการคำนวณค่า $\bm{w}$ ออกมา
%และใช้\textit{เมท็อด} \verb|decision_score| ในการคำนวณค่าฟังก์ชันแบ่งแยกสำหรับข้อมูลที่สนใจ.

คล้ายกับตัวอย่างในแบบฝึึกหัด~\ref{ex: svm primal scipy.optimize}
คำสั่งข้างล่าง แสดงตัวอย่างการฝีก และการอนุมานด้วยซัพพอร์ตเวกเตอร์แมชชีน
\begin{Verbatim}[fontsize=\small]
svm = cSVM()
svm.kernel = lambda xq, xp: np.dot(xq.T, xp) # linear kernel
C = 1
res = svm.train(datax, datay, C=C)
svm.decision_score(testx)
\end{Verbatim}
ในตัวอย่าง ใช้อภิมานพารามิเตอร์ $C=1$ ฝึกด้วยข้อมูล \verb|datax| และ \verb|datay|
ที่ต้องเป็นชนิด \verb|np.array| สัดส่วน \verb|(D, N)| และ \verb|(1, N)| ตามลำดับ
เมื่อ \verb|D| และ \verb|N| แทนจำนวนมิติของอินพุต และจำนวนจุดข้อมูล ตามลำดับ.
คำสั่งสุดท้าย ใช้คำนวณค่าฟังก์ชันแบ่งแยกสำหรับ ข้อมูลทดสอบ \verb|testx|.

จงศึกษาวิธีการเขียนโปรแกรม ทดลองใช้งาน สังเกตผล อภิปราย และสรุป.
ทดลองค่า $C$ ต่าง ๆ และรายงานดังตัวอย่างในรูป~\ref{fig: csvm different Cs}.


\lstinputlisting[language=Python, caption={ซัพพอร์ตเวกเตอร์แมชชีน}, label={code: cSVM}]{04Classic/code/code_cSVM.py}
\index{english}{support vector machine!code}
\index{thai}{ซัพพอร์ตเวกเตอร์แมชชีน!โปรแกรม}

\end{Exercise}

\begin{Exercise}
	\label{ex: svm dual gaussian}
	\index{english}{SVM}
	\index{english}{Support Vector Machine}
	\index{thai}{ซัพพอร์ตเวกเตอร์แมชชีน}

จงสร้างข้อมูลที่ไม่สามารถแบ่งแยกสมบูรณ์ได้เชิงเส้น
โดยจะสร้างให้มีลักษณะใดก็ได้ แต่ควรทำให้สามารถตรวจดูได้สะดวก เช่น อินพุตควรจะเป็นสองมิติ.
ตัวอย่างอาจจะเช่นที่แสดงในรูป~\ref{fig: non-linearly-separable datapoints}.
ทดลองใช้ซัพพอร์ตเวกเตอร์แมชชีน กับเคอร์เนลเชิงเส้น และเคอร์เนลเกาส์เซียนที่ค่า $\sigma$ ต่าง ๆ.
ดูรูป~\ref{fig: csvm linear kernel on wave datapoints} กับ~\ref{fig: csvm gaussian kernels} สำหรับตัวอย่าง.
ออกแบบการทดลอง เพื่อศึกษาผลของ $C$ และ $\sigma$.
ทดลอง สังเกตผล อภิปราย และสรุป.

ตัวอย่าง คำสั่งข้างล่างแสดงการกำหนดค่าเคอร์เนลของ\textit{ตัวแปรวัตถุ} (จาก\texttt{cSVM})
ให้เป็นเคอร์เนลเกาส์เซียน (ใช้ $\sigma = 2$)
\begin{Verbatim}[fontsize=\small]
svm.kernel = lambda xq, xp: gaussian(xq, xp, sigma=2)
\end{Verbatim}
โดยโปรแกรม \verb|gaussian| แสดงในรายการ~\ref{code: gaussain kernel}.

\lstinputlisting[language=Python, caption={[เคอร์เนลเกาส์เซียน]โปรแกรมเคอร์เนลเกาส์เซียน สำหรับซัพพอร์ตเวกเตอร์แมชชีน}, label={code: gaussain kernel}]{04Classic/code/gaussian.py}
\index{english}{support vector machine!gaussian kernel!code}
\index{thai}{ซัพพอร์ตเวกเตอร์แมชชีน!เกาส์เซียนเคอร์เนล!โปรแกรม}

%
\begin{figure}[H]
	\begin{center}
		\includegraphics[width=0.4\textwidth]{./04Classic/svm/nonlinsep.png}
		\caption[ตัวอย่างข้อมูลที่ไม่สามารถแบ่งแยกสมบูรณ์ได้เชิงเส้น]{ต้วอย่างข้อมูลที่ไม่สามารถแบ่งแยกสมบูรณ์ได้เชิงเส้น.}
		\label{fig: non-linearly-separable datapoints}
	\end{center}
\end{figure}
%
	
\end{Exercise}

%LATER
%\paragraph{มอดูลไซคิทเลิร์น.}

% LATER
%\begin{Exercise}
%	\label{ex: svm pulsar}
%	\index{SVM}
%	\index{Support Vector Machine}
%	\index{ซัพพอร์ตเวกเตอร์แมชชีน}
%	
%\end{Exercise}


%\begin{Exercise}
%	\label{ex: mAP}
%
%
%
%\end{Exercise}
