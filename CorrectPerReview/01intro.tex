\chapter{บทนำ}
\label{chapter: introduction}

การประยุกต์ใช้ที่ทำให้ศาสตร์\textit{การเรียนรู้ของเครื่อง}
เป็นที่รู้จักอย่างกว้างขวาง คือ การรู้จำภาพ การรู้จำคำพูด การประมวลผลภาษาธรรมชาติ.
การประยุกต์ใช้เหล่านี้แม้แตกต่างอย่างมาก ในเชิงสิ่งที่แสดงออก.
การรู้จำภาพ สัมผัสการมองเห็น. 
การรู้จำคำพูด สัมผัสการได้ยิน.
การประมวลผลภาษาธรรมชาติ สัมผัสภาษา ซึ่งเป็นตัวแทนของความคิด.
แต่ทั้งหมดล้วนมีจุดร่วมกันที่สำคัญ
คือ ทั้งหมดเป็น\textit{การรู้จำรูปแบบ}.

สำหรับการเรียนรู้
การอ่านทฤษฎี เป็นวิธีที่ดีที่สุด ที่(อาจ)จะช่วยให้รู้เรื่อง แต่ไม่เข้าใจ.
การลงมือทำโดยไม่สนใจทฤษฎี เป็นวิธีที่เร็วที่สุด ที่(อาจ)จะช่วยให้ทำได้ แต่ไม่รู้เรื่อง.
การสังเกตและไตร่ตรอง เป็นวิธีที่ธรรมชาติที่สุด ที่(อาจ)จะช่วยให้เข้าใจ แต่อาจผิด.
ศาสตร์\textit{การเรียนรู้ของเครื่อง}และ\textit{การรู้จำรูปแบบ}
ไม่ได้ต่างจากศาสตร์อื่น ๆ เลย ในแง่ที่วิธีที่ดีที่สุดในการเรียนรู้ 
คือ การหาสมดุลระหว่างการศึกษาทฤษฎี การลงมือปฏิบัติ และการสังเกตและไตร่ตรอง.

เจเรมี โฮเวิร์ด (Jeremy Howard) หนึ่งในผู้เชี่ยวชาญทางด้านการเรียนรู้ของเครื่อง ได้ระบุสี่คุณสมบัติที่สำคัญสำหรับผู้ที่เหมาะกับงานการเรียนรู้ของเครื่อง ได้แก่
หนึ่ง ความอึด (tenacity), 
สอง ความอยากรู้อยากเห็น (curiosity),
สาม ความคิดสร้างสรรค์ (creativity) และสุดท้ายสี่ ทักษะ (skills) ซึ่งหมายถึงคณิตศาสตร์และการเขียนโปรแกรมคอมพิวเตอร์.
คุณสมบัติทั้งสี่นี้เรียงตามลำดับ
นั่นคือ ความอึดสำคัญที่สุด.

อย่างไรก็ตาม บางคนอาจเริ่มด้วยคุณสมบัติที่เหมาะกับงาน 
แต่บางคนอาจใช้งานเป็นแรงจูงใจในการพัฒนาคุณสมบัติให้เกิดขึ้นในตัวเอง
ดังคำกล่าวของ {เจมส์} {แอนโธนี} {ฟรูด} ว่า ``ความคิดฝันไม่อาจเนรมิตตัวคุณให้เป็นคนที่คุณนับถือได้ คุณต้องมุ่งมั่นบากบั่นพัฒนาตัวให้เป็นให้ได้''
(James Anthony Froude: ``You cannot dream yourself into a character; you must hammer and forge yourself one.'')

\section{รูปแบบ}

% start two column, bilingual environment
\begin{Parallel}[c]{0.52\textwidth}{0.44\textwidth}
\selectlanguage{english}
\ParallelLText{
``There are only patterns, patterns on top of patterns, patterns that affect other patterns. Patterns hidden by patterns. Patterns within patterns. 
If you watch close, history does nothing but repeat itself. 
What we call chaos is just patterns we haven't recognized. 
What we call random is just patterns we can't decipher. 
What we can't understand we call nonsense. 
What we can't read we call gibberish.''
%There is no free will. 
%There are no variables.''
\begin{flushright}
---Chuck Palahniuk
\end{flushright}
}
\selectlanguage{thai}
\ParallelRText{
``รูปแบบเท่านั้น รูปแบบของรูปแบบ รูปแบบที่มีผลกับรูปแบบอื่น รูปแบบที่ซ่อนในรูปแบบ รูปแบบซ้อนในรูปแบบ
ถ้าคุณดูดี ๆ ประวัติศาสตร์ไม่มีอะไร นอกจากซ้ำตัวมันเอง
สิ่งที่เราเรียก ความยุ่งเหยิง ก็เป็นแค่รูปแบบที่เรายังมองไม่ออก 
สิ่งที่เราเรียก ไร้แบบแผน ก็เป็นแค่รูปแบบที่เรายังอ่านไม่ออก
อะไรที่เราไม่เข้าใจ เราว่าไร้สาระ
อะไรที่เราอ่านไม่ออก เราว่าไม่มีความหมาย.''
%\\
%มันไม่มีหรอกเจตจำนงเสรี
%มันไม่มีหรอกความผันแปร'' \\
\begin{flushright}
---ชัก ปาลาห์นีอุค
\end{flushright}
}
\end{Parallel}
\index{english}{quote!pattern}
\index{english}{words of wisdom!Chuck Palahniuk}
\vspace{0.5cm}

รูปแบบมีอยู่ในทุก ๆ อย่าง ไม่ว่าจะธรรมชาติ เอกภพ ชีวิต หรือจิตปัญญา.
ไม่ว่าจะประวัติศาสตร์ สงคราม การเอาตัวรอด กีฬา การเต้นรำ การเคลื่อนไหว การคิด ดนตรี ศิลปะ ความรู้ หรือภาษา ล้วนมีรูปแบบอยู่.
\textbf{รูปแบบ} (pattern) หรือ\textit{การซ้ำเชิงโครงสร้าง} (structural repetition) ช่วยทำให้เราเข้าใจความเป็นไปต่าง ๆ ช่วยทำให้เราจดจำผู้คน อาหาร อันตราย วิธีเอาตัวรอด ภาษา สถานที่ สัญลักษณ์ วัตถุ แนวคิด ไปจนถึง เรื่องราวต่าง ๆ ได้.
\index{english}{pattern}
\index{thai}{รูปแบบ}

การที่เรารู้ว่าภาพที่เห็นเป็นภาพของอะไร มีวัตถุอะไรอยู่บ้าง อยู่ที่ไหน หรือภาพบอกเล่าเหตุการณ์อะไร 
เป็นเพราะมีรูปแบบของภาพที่เราจำได้หรือรู้จักอยู่.
การที่เราเข้าใจเสียงที่ได้ยินว่าเป็นเสียงของอะไร เสียงพูดของใคร กำลังพูดภาษาอะไร สำเนียงของที่ไหน พูดถึงอะไร อารมณ์เป็นอย่างไร
เป็นเพราะมีรูปแบบของเสียง ของภาษาที่เราจำได้อยู่.
การที่เราได้ฟังหรืออ่านข้อความของภาษา แล้วเข้าใจความหมาย
เป็นเพราะมีรูปแบบของภาษา ของความหมายที่เราจำได้อยู่
รวมถึงมีรูปแบบของการเชื่อมความสัมพันธ์ต่าง ๆ เข้าด้วยกัน
และรูปแบบการสร้างรูปแบบใหม่ ที่เรารับรู้อยู่ ไม่ว่าจะรับรู้ในระดับจิตสำนึก หรือระดับจิตใต้สำนึก.
ดังนั้นอาจกล่าวได้ว่า \textit{การรู้จำรูปแบบ}  เป็นความสามารถที่สำคัญที่สุดอย่างหนึ่งของสติปัญญา.

\section{การรู้จำรูปแบบและการเรียนรู้ของเครื่อง}
\label{sec: intro ML}
\index{english}{pattern recognition}
\index{thai}{การรู้จำรูปแบบ}
\index{english}{machine learning}
\index{thai}{การเรียนรู้ของเครื่อง}

\textbf{การรู้จำรู้แบบ} (pattern recognition) โดยทั่วไป หมายถึง
ภารกิจ เพื่อระบุว่า
\textit{ข้อมูลที่สำรวจ}มีสิ่งที่สนใจอยู่หรือไม่ 
หรือ
เพื่อระบุว่า
\textit{ข้อมูลที่สำรวจ}เป็นสิ่งที่สนใจประเภทใด
หรือ
เพื่อระบุว่า
สิ่งที่สนใจอยู่ที่ใดใน\textit{ข้อมูลที่สำรวจ}
หรือ
เพื่อประเมินค่าที่สนใจออกมาจาก\textit{ข้อมูลที่สำรวจ}
เป็นต้น.

\textit{การรู้จำรูปแบบ}นั้น
มีอยู่ทั่วไปในธรรมชาติ
เป็นความสามารถของสิ่งมีชีวิต
เป็นส่วนสำคัญในพฤติกรรมทางสังคม
แต่ในที่นี้
\textit{การรูปจำรูปแบบ}
จะเจาะจงเฉพาะกับ
วิธีการที่จะทำให้คอมพิวเตอร์
มีความสามารถในการรู้จำรู้แบบ.
%
การรู้จำรูปแบบด้วยคอมพิวเตอร์
อาจทำได้หลายแนวทาง.
%
แนวทางหนึ่งคือ
แนวทางการกำหนดเกณฑ์ที่ชัดเจน
(criteria-based or rule-based approach)
รวมไปถึง
การจับคู่กับแผ่นแบบ (template matching).
สำหรับบางงาน
เกณฑ์แม้จะชัดเจน 
แต่อาจจะไม่เจาะจงที่ตัวรูปแบบเอง
ตัวอย่างเช่น 
การค้นหารูปแบบที่ปรากฏบ่อย ๆ
%
ในการศึกษาด้านพันธุกรรม บางครั้งอาจต้องการหาโมทีฟ (motif) หรือ ลำดับของสารพันธุกรรมสายยาว ๆ ที่พบได้บ่อยที่สุด  ซึ่งเกณฑ์อาจจะระบุชัดเจน เรื่องความยาวของสายพันธุกรรม และเรื่องความถี่ที่ปรากฏ แต่ไม่ได้ระบุลำดับของรหัสพันธุกรรมที่ค้นหา.
%(ดูแบบฝึกหัด~\ref{})

อย่างไรก็ดี
\textit{รูปแบบ}ซึ่งเป็นการซ้ำเชิงโครงสร้าง
อาจเป็นผลมาจากความสัมพันธ์สำคัญ ที่เชื่อมโยงข้อมูลกับรูปแบบนั้น.
ดังนั้น
การรู้จำรูปแบบ 
ก็เปรียบเสมือนการเรียนรู้ความสัมพันธ์สำคัญ
ที่เชื่อมโยงระหว่างข้อมูลนำเข้าและรูปแบบที่สนใจนั้น.
%
แนวทางหลักของการรู้จำรูปแบบที่จะอภิปรายในที่นี้
คือ
แนวทางของ\textit{การเรียนรู้ของเครื่อง}.
%
วิธี\textit{การเรียนรู้ของเครื่อง} 
จะไม่ได้พึ่งกฎหรือเกณฑ์ที่ชัดเจน 
ดังแนวทางที่กล่าวไปข้างต้น
แต่ใช้ความสามารถในการทำงานกับข้อมูลมาก ๆ ของคอมพิวเตอร์ ประกอบกับแบบจำลองทางคณิตศาสตร์
เพื่อช่วยในการค้นหา 
หรือช่วยในการเรียนรู้ความสัมพันธ์ระหว่าง\textit{ข้อมูลนำเข้า} 
และรูปแบบที่มักเรียกว่า \textit{ข้อมูลนำออก} 
%
โดยเฉพาะสำหรับความสัมพันธ์ที่มีลักษณะซับซ้อน
และยากที่จะกำหนดเป็นกฎหรือเกณฑ์ดังกล่าว.

ลักษณะเด่นของวิธีการเรียนรู้ของเครื่อง
อาจปรากฏชัดอยู่ในตัวอย่างงานของอาร์เธอร์ ซามูเอล (Arthur Samuel) 
ในปี ค.ศ. 1959 
ที่
เขาเขียนโปรแกรมคอมพิวเตอร์เพื่อเล่นหมากฮอร์ส\cite{SamuelML} 
%(Checker playing program) 
โดยที่ ตัวซามูเอลเองเล่นหมากฮอร์สไม่เก่งเลย.
หากซามูเอลเขียนโปรแกรมด้วยแนวคิดดั่งเดิม
เขาจะต้องหัดเล่นหมากฮอร์สให้เก่ง
และโปรแกรมวิธีเดินหมากอย่างละเอียดให้กับคอมพิวเตอร์.
ซามูเอลไม่ได้เลือกทำแบบนั้น
เขาเลือกที่จะโปรแกรมคอมพิวเตอร์
ให้เล่นแข่งกันเอง และให้คอมพิวเตอร์เก็บผลว่า
ตำแหน่งของหมากอย่างไรที่เป็นตำแหน่งที่ดี ซึ่งนำไปสู่ชัยชนะ 
หรือตำแหน่งไหนเป็นตำแหน่งไม่ดี และมักจะทำให้แพ้ 
แล้วให้โปรแกรมเลือกเดินหมากตามผลที่เก็บนั้น.
หลังจากซามูเอลปล่อยให้โปรแกรมเล่นแข่งกันเองหลายหมื่นกระดาน โปรแกรมเล่นหมากฮอร์สของซามูเอลก็สามารถเล่นหมากฮอร์สได้ดีมาก 
และเล่นได้ดีกว่าตัวของซามูเอลเอง.
ซึ่งกรณีเช่นนี้ แทบจะเป็นไปไม่ได้เลยกับวิธีการเขียนโปรแกรมแบบดั่งเดิม.
ดังนั้น ณ ตอนนั้น วิธีการสร้างโปรแกรมเล่นหมากฮอร์สของซามูเอล
เป็นเหมือนการปฏิวัติแนวทางใหม่เลยทีเดียว. 
และนี่คือลักษณะเด่นของการเรียนรู้ของเครื่อง 
คือแทนที่จะเขียนโปรแกรมวิธีทำอย่างละเอียดให้คอมพิวเตอร์ 
กลับเขียนโปรแกรมให้คอมพิวเตอร์มีความสามารถในการเรียนรู้วิธีทำ สร้างสิ่งแวดล้อมให้คอมพิวเตอร์ได้เรียนรู้
แล้วปล่อยให้คอมพิวเตอร์เรียนรู้วิธีทำเอง.

อาร์เธอร์ ซามูเอล\cite{SamuelML} ได้นิยาม\textbf{การเรียนรู้ของเครื่อง} (machine learning) ไว้ว่า 
%``Field of study that gives computers the ability to learn without being explicitly programmed.''
%
\textit{การเรียนรู้ของเครื่อง} คือ ศาสตร์ของการทำให้คอมพิวเตอร์มีความสามารถที่จะเรียนรู้ได้ โดยที่ไม่ต้องเขียนโปรแกรมวิธีทำตรง ๆ.
%
\index{english}{machine learning}
\index{thai}{การเรียนรู้ของเครื่อง}
%
ทอม มิทเชล\cite{Mitchell1997a} %นักวิจัยชั้นนำทางด้านการเรียนรู้ของเครื่อง 
ได้ช่วยขยายความ
โดยนิยามว่า
%การเรียนรู้ของเครื่อง คือ 
%\begin{verse}
โปรแกรมคอมพิวเตอร์จะเรียกได้ว่า มีการเรียนรู้จาก\textit{ประสบการณ์} $E$ ซึ่งเกี่ยวข้องกับ\textit{ภารกิจ} $T$ 
และ\textit{สมรรถนะของผลสัมฤทธิ์}ที่วัดได้ $P$
ก็ต่อเมื่อ\textit{สมรรถนะ}ของการทำภารกิจ $T$ ที่วัดด้วย $P$ ปรับปรุงขึ้นได้จากประสบการณ์ $E$.
%
%\end{verse}
%\begin{verse}
%``''
%``Well-posed Learning Problem: A computer program is said to learn from experience E with respect to some task T and some performance measure P, if its performance on T, as measured by P, improves with experience E.''\cite{Mitchell1997a} \\
%\end{verse}
%
%คำนิยามนี้มีนัยทางรูปธรรมอยู่มาก.
%สรุปใจความได้ว่า การเรียนรู้ของเครื่อง คือ โปรแกรมคอมพิวเตอร์ ที่เรียนรู้จากประสบการณ์ $E$ เพื่อจะทำงาน $T$ ที่มีตัววัดผลการทำงาน $P$ และ สามารถทำให้ผลการทำงาน $T$ ที่วัดด้วย $P$ ดีขึ้นได้จากประสบการณ์ที่ได้จาก $E$.

จากตัวอย่าง โปรแกรมเล่นหมากฮอร์สของซามูเอล 
ประสบการณ์ $E$ คือ การได้เล่นแข่งเล่นแข่งกันเอง
ภารกิจ $T$ คือการเล่นหมากฮอร์ส
และสมรรถนะ $P$ วัดได้จากการที่โปรแกรมเล่นชนะ.
%
%ตัวอย่างที่สอง โปรแกรมเลือกหัวข้อสำหรับข้อความ\cite{BleiEtAl2003a} 
%ประสบการณ์ $E$ คือการลองเลือกคำในข้อความไปเปรียบเทียบกับเนื้อหาในข้อความอื่น ๆ
%ภารกิจ $T$ คือการเลือกคำในข้อความมาเป็นหัวข้อ
%และสมรรถนะ $P$ คือความน่าจะเป็นที่คำที่เลือกมาจะเป็นตัวแทนเนื้อหาของข้อความ

%โปรแกรมกรองอีเมล์ขยะ % (spam email filtering) 
%ที่เวลา เราเข้าไปใช้อีเมล์ แล้ว เราอาจจะคลิก ``spam'' เพื่อแจ้งโปรแกรมว่าอีเมลล์ที่เราได้เป็นอีเมล์ขยะ และ โปรแกรมสามารถ เรียนรู้ จากการรายงานของเรา เพื่อที่จะกรองอีเมล์ได้ดีขึ้น.
%ประสบการณ์ $E$ คือ โปรแกรม ดู การคลิกรายงานอีเมล์ขยะของเรา.
%งาน $T$ คือ การจำแนกแยกอีเมล์ ว่าเป็นอีเมล์ขยะ หรือ อีเมล์ไม่ใช่ขยะ.
%ตัววัดการทำงาน $P$ คือ อัตราส่วน จำนวนอีเมล์ที่ถูกแยกได้อย่างถูกต้อง ต่อ จำนวนอีเมล์ทั้งหมด
%


ปัจจุบัน \textit{การเรียนรู้ของเครื่อง} ถูกประยุกต์ใช้อย่างกว้างขวาง
ในวงการธุรกิจ อุตสาหกรรม การทหาร วงการวิทยาศาสตร์ วิศวกรรม การแพทย์ การเกษตร บันเทิง ศิลปะ การกีฬา รวมถึงการประยุกต์ใช้ชีวิตประจำวัน
ตัวอย่างเช่น 
การค้นหาข้อมูลบนเวป (web search), 
การกรองข้อมูลบนสื่อสังคมออนไลน์ (content filtering on social media),
%ลักษณะงานที่เป็น\textit{การทำเหมืองข้อมูล} 
การตรวจสอบหารูปแบบการใช้บัตรเครดิตที่ผิดปกติ\cite{DeviEtAl2014a} ซึ่งอาจเนื่องมาจากการที่บัตรถูกขโมยไป,
การบริหารการลงทุนทางการเงิน\cite{TanEtAl2011a},
งานแอพพลิเคชั่นที่ไม่สามารถโปรแกรมตรง ๆ ได้ (หรือ ทำได้ยากมาก) ได้แก่ ระบบอ่านลายมือเขียน\cite{LeCunEtAl1990a},
การควบคุมเฮลิคอปเตอร์ไร้นักบิน\cite{CoatesEtAl2009a}, 
การควบคุมหุ่นยนต์ที่มีการเครื่องไหวที่ซับซ้อน\cite{AkiyamaEtAl2010a},
การบริหารจัดการทรัพยากรนำ้\cite{CastellettiEtAl2013a},
การปรับตั้งค่าของเวอร์ชัวร์แมชชีน\cite{RaoEtAl2009a},
การพัฒนารถยนต์ที่ขับเคลื่อนได้เองโดยไร้คนขับ\cite{ZhuEtAl2014a},
การติดตามลักษณะโครงสร้างใต้น้ำอัตโนมัติ\cite{MagazzeniEtAl2014a},
การระบุหารังสีแกมม่าจากข้อมูลกล้องโทรทัศน์\cite{BockEtAl2004a},
ระบบตรวจสอบการสั่นสะเทือนของแผ่นดินไหว\cite{RuanoEtAl2014a},
การหารูปแบบในข้อมูลชีวสารสนเทศ\cite{KelchtermansEtAl2014a},
การแปลภาษาอัตโนมัติ\cite{CostaFarrus2014a},
ระบบรู้จำคำพูด\cite{SarikayaEtAl2014a},
ระบบรู้จำความก้าวหน้าของคอร์ดดนตรี\cite{YuElAl2013a},
การประยุกต์ใช้กับงานศิลปะ\cite{CuljakEtAl2011a},
การประยุกต์ใช้กับกีฬา\cite{HolstJanasson2013a},
ระบบรู้จำใบหน้า\cite{BarnardEtAl2013a},
ระบบตรวจสอบความผิดปกติของสัญญาณคลื่นไฟฟ้าหัวใจ\cite{LiEtAl2012a},
การแยกอีเมล์ที่เป็นสแปม\cite{BlanzieriBryl2008a},
ระบบแนะนำหนังสือ เพลง วิดีโอ หรือสินค้า\cite{GhazanfarPrugel-Bennett2014a},
การวิเคราะห์พฤติกรรมลูกค้า\cite{KatanyukulPonsawat2017a},
การจำแนกหรือระบุหัวข้อสำหรับข้อความ\cite{BleiEtAl2003a},
การเพิ่มประสิทธิภาพของงานของระบบควบคุมหรือระบบตัดสินใจที่ซับซ้อน% ระบบควบคุมการระบายอากาศ-เครื่องทำความร้อน--เครื่องปรับอากาศ
\cite{AndersonEtAl2004a,KatanyukulEtAl2011a, KatanyukulEtAl2012a, Katanyukul2013a, KatanyukulChong2014a, ChanlohaEtAl2014a}
%ระบบควบคุมสินค้าคงคลัง\cite{KatanyukulEtAl2011a, KatanyukulEtAl2012a, Katanyukul2013a, KatanyukulChong2014a}
%ระบบควบคุมการจราจร\cite{ChanlohaEtAl2014a}.
ไปจนถึง การช่วยปรับปรุงคุณภาพชีวิตผู้พิการ\cite{NakjaiKatanyukul2019a}.

%\begin{minipage}{5.5in}
{\small
	\begin{shaded}
		\paragraph{\small เกร็ดความรู้ รูปแบบของลูคีเมียและยารักษา}
%		\index{thai}{การรู้จำรูปแบบ}
%		\index{english}{pattern recognition}
		\index{english}{side story}
		\index{english}{side story!CML and Cure}
		\index{thai}{เกร็ดความรู้}
		\index{thai}{เกร็ด!มะเร็งและยารักษา}
		
	เรียบเรียงจากบางส่วนของ \cite{TheSerengetiRules}.
	รูปแบบมักซ่อนความสัมพันธ์หรือกลไกที่สำคัญอยู่เบื้องหลัง.
	เจเน็ต โรวลี่ (Janet Rowley) คุณแม่ลูกสี่ เลี้ยงลูกเป็นหลัก และทำงานเสริมกับโรงพยาบาลวิจัยมะเร็งอาร์กอน.
	เธอ ทำงานศึกษาตัวอย่างเซลล์จากคนไข้ที่ป่วยด้วยโรคเลือดต่าง ๆ แล้วในช่วงต้นปี ค.ศ. 1972 เธอก็สังเกตพบรูปแบบผิดปกติในเซลล์ของคนไข้ที่ป่วยด้วยโรคลูคีเมียเฉียบพลันชนิดไมลิลอยด์ 
	คือ ดูเหมือนว่า มีบางส่วนของโครโมโซมที่แปด และบางส่วนของโครโมโซมที่ยี่สิบเอ็ด สลับกัน (เรียกว่า การสลับตำแหน่งทางพันธุกรรม หรือ translocation).
	โรวลี่ดีใจมาก และคิดว่าเธออาจพบสาเหตุของลูคีเมีย ซึ่งเป็นมะเร็งเม็ดเลือดขาว.
	ณ เวลานั้น ถึงแม้วงการแพทย์จะรู้แล้วว่า เซลล์มะเร็งมักมีโครโมโซมที่แปลกไป 
	แต่ก็ยังไม่มีใครพบรูปแบบที่เด่นชัด 
และส่วนใหญ่ (ณ ตอนนั้น) ก็เชื่อกันว่า โครโมโซมที่แปลกไปเป็นผลมาจากมะเร็ง 
ไม่ใช่เป็นสาเหตุของมะเร็ง.	
	 โรวลี่ได้เขียนบทความรายงานเรื่องนี้ไปที่วารสารการแพทย์นิวอิงแลนด์ ซึ่งเป็นวารสารชั้นนำ แต่กลับถูกปฏิเสธ ด้วยเหตุที่วารสารชี้แจงว่า สิ่งที่โรวลี่พบไม่สำคัญ. โรวลี่ส่งบทความนั้นไปที่วารสารเล็ก ๆ แทน.
	 
	หลังจากนั้นไม่นาน โรวลี่ ก็ได้ศึกษา เซลล์มะเร็งจากคนไข้ ที่ป่วยด้วยโรคลูคีเมียเรื้อรังชนิดไมอิลอยด์ (Chronic Myelogenous Leukemia คำย่อ CML). 
	แม้จะเป็นงานเสริม โรวลี่ ก็สนุกกับงานที่ทำมาก.
	เธอนำภาพถ่ายของโครโมโซม จากเซลล์ของคนไข้ กลับไปดูที่บ้านด้วยในวันหยุด.
	ภาพถ่ายของโครโมโซม เป็นคู่ ๆ เหมือนปลาท่องโก๋ มีจุดร่วมกันตรงกลาง ๆ และมีแขนยื่นข้างบน มีขายื่นข้างล่าง.
	โรวลี่วางภาพถ่ายกระจายเต็มโต๊ะอาหารที่บ้าน จนลูก ๆ ของเธอแซวว่า แม่กำลังเล่นกับตุ๊กตากระดาษอยู่.
	โรวลี่ดูภาพถ่ายเหล่านั้นอย่างละเอียด ซึ่งเป็นภาพที่ถ่ายเซลล์ที่ผ่านวิธีย้อมแบบใหม่ เธอพบว่าโครโมโซมที่เก้าในเซลล์มะเร็งยาวกว่า โครโมโซมที่เก้าที่พบในเซลล์ปกติ.
	ก่อนหน้านั้นมีนักวิจัยจากฟิลาเดลเฟียพบว่า โครโมโซมที่ยี่สิบสองจะสั้นผิดปกติ ในเซลล์จากผู้ป่วยลูคีเมียเรื้อรังชนิดไมอิลอยด์ จนโครโมโซมที่ยี่สิบสองที่สั้นผิดปกติ ถูกเรียกว่า ฟิลาเดลเฟียโครโมโซม.
	
	สำรวจต่อ เธอพบว่า โครโมโซมที่ยี่สิบสอง กับโครโมโซมที่เก้า มีการสลับชิ้นส่วนกัน 
	และชิ้นส่วนที่สลับกันยาวไม่เท่ากัน
	จึงทำให้โครโมโซมที่เก้ายาวขึ้น
	และโครโมโซมที่ยี่สิบสองสั้นลง.
	%
	โรวลี่ตรวจสอบเซลล์อื่น ๆ ที่ปกติของคนไข้ พบว่า เซลล์ปกติไม่ได้มีการสลับชิ้นส่วนกันแบบนี้ มีพบแต่เฉพาะในเซลล์มะเร็ง.
	คราวนี้ โรวลี่ส่งบทความไปตีพิมพ์กับวารสารเนเชอร์ (Nature) ซึ่งเป็นวารสารวิชาการทางวิทยาศาสตร์ชั้นนำ แม้วารสารเนเชอร์จะปฏิเสธโรวลี่ในครั้งแรก แต่ด้วยหลักฐานที่เพิ่มขึ้น ที่สุด วารสารเนเชอร์ก็ยอมรับตีพิมพ์รายงานของโรวลี่.
	%
	ต่อจากนั้น โรวลี่ก็พบการสลับตำแหน่งทางพันธุกรรมระหว่างโครโมโซมที่สิบห้า และโครโมโซมที่สิบเจ็ด ในเซลล์มะเร็งจากผู้ป่วยลูคีเมียเฉียบพลันชนิดโปรไมอิโลไซติก.
	
	โรวลี่สงสัยว่า การสลับตำแหน่งทางพันธุกรรมน่ามีส่วนในการก่อมะเร็ง
	แต่ว่า ณ ตอนนั้น มันยากที่จะพิสูจน์ประเด็นนี้ 
	แม้งานก่อนหน้านี้ของ เพย์ตัน รูส (Peyton Rous) ที่พบว่า ไวรัสสามารถก่อให้เกิดมะเร็งซาร์โคม่าในไก่ ก็ยังไม่ได้รับการยอมรับเท่าที่ควร (ณ ตอนนั้น).
	อย่างไรก็ดี การค้นพบไวรัสก่อมะเร็งในสัตว์ เป็นเรื่องสำคัญ ที่จะช่วยไขปริศนาต้นกำเนิดมะเร็ง เพราะว่า ไวรัสที่รูสศึกษา ซึ่งเรียกว่า รูสซาร์โคมาไวรัส (Rous sarcoma virus คำย่อ RSV)
	รูสซาร์โคมาไวรัสมียีนอยู่แค่สี่ยีน 
	ทำให้พอจะมีแนวทางค้นหา ว่ายีนตัวไหนที่มีส่วนในการก่อมะเร็ง.
	สตีฟ มาร์ติน (Steve Martin) นักศึกษาจากมหาวิทยาลัยแคลิฟอร์เนีย ที่เบิร์คลี่ ได้ แยกรูสซาร์โคมาไวรัสที่กลายพันธุ์ออกมา 
	และได้ขยายพันธุ์ออกมาเป็นเซลล์ 
	แต่เซลล์ที่ได้กลับไม่เป็นเซลล์มะเร็ง 
	เพราะมีการกลายพันธุ์ในไวรัสของมาร์ติน.
	ตำแหน่งที่กลายพันธุ์ในไวรัสของมาร์ติน อยู่ในยีนที่เรียกว่า ซาร์ค (src).
	ยีนซาร์คที่สมบูรณ์จะทำให้เกิดเซลล์มะเร็ง ซาร์คจึงถูกเรียกว่า ยีนมะเร็ง (oncogene).
	การค้นพบยีนมะเร็งในรูสซาร์โคมาไวรัส อาจจะช่วยอธิบายและยืนยันการค้นพบของรูส แต่ยังไม่ได้ช่วยอธิบายสาเหตุของมะเร็งในมนุษย์สักเท่าไร
	
	การค้นพบซาร์ค ทำให้ทีมวิจัยของฮาโรล์ด วาร์มูส (Harold Varmus) และเจ ไมเคิล บิชอบ (J. Michael Bishop) ศึกษา และ	สงสัยว่า แล้วซาร์คไปอยู่ในไวรัสได้อย่างไร ในเมื่อตัวไวรัสเองไม่ได้ต้องการยีนนี้เลย ไวรัสไม่ได้ต้องการซาร์คเพื่อการยึดเซลล์ ไวรัสไม่ได้ต้องการซาร์คเพื่อการแบ่งเซลล์.
	วาร์มูสและบิชอบคิดว่า ไวรัสน่าจะได้ซาร์คมาโดยบังเอิญ 
	จากเซลล์ไหนสักเซลล์ที่มันเคยไปยึดมา 
	ถ้าเป็นแบบนั้นจริง มันก็น่าจะมีซาร์คปรากฏอยู่ในเซลล์ปกติด้วย.
	
	แต่ก็เกือบ ๆ สี่ปีหลังจากนั้น 
	กว่าที่มีการพบซาร์คในเซลล์ปกติ ซาร์คในเซลล์ปกตินี้ เรียกว่า {ซีซาร์ค} (c-src สำหรับ cellular src) เพื่อระบุให้ต่างจาก {วีซาร์ค} (v-src) ที่เป็นยีนมะเร็ง.
	ปรากฏว่า {ยีนซีซาร์ค}ไม่ได้มีเฉพาะในไก่ แต่มีการพบในสัตว์หลาย ๆ ชนิด รวมถึงมนุษย์ด้วย.
	การค้นพบนี้ ทำให้ วาร์มูสและบิชอบคิดต่อไปว่า
	{ซีซาร์ค}คงมีหน้าที่สำคัญอะไรสักอย่างในเซลล์ปกติ
	และตอนที่ไวรัสได้ซาร์คไป 
	อาจจะไปเปลี่ยนแปลงบางอย่างในซาร์ค จนทำให้มันกลายเป็นยีนมะเร็ง.
	
	ซาร์คเป็นยีนแรก และหลังจากนั้นก็มีการค้นพบยีนมะเร็งจากไวรัสอื่น ๆ และเช่นเดียวกับวีซาร์ค ที่มีซีซาร์ค
	หลาย ๆ ยีนมะเร็ง ก็พบว่ามียีนแบบนั้น ๆ ได้ในเซลล์ปกติด้วย และพบในสัตว์หลายชนิด รวมถึงมนุษย์ด้วย. 
	ยีนเหล่านั้น เช่น 
	ยีนเอมวายซี (myc)
	ยีนเอบีเอล (abl)
	ยีนอาร์เอเอส (ras)
	ช่วยยืนยันว่า แนวคิดว่า ยีนมะเร็งของไวรัส มีที่มาจากยีนของเซลล์ปกติ.
	ยีนแบบเดียวกับยีนมะเร็งแต่พบในเซลล์ปกติ จะเรียกว่า ยีนก่อนมะเร็ง (proto-oncogene).
	
	วีเอบีเอล (v-abl) ในไวรัสของหนู เป็นหนึ่งในยีนมะเร็งที่ค้นพบหลังจากยีนซาร์ค
	และยีนก่อนมะเร็ง ที่เป็นคู่ของมัน คือ ซีเอบีเอล (c-abl) ก็พบได้ในเซลล์ปกติของหนู 
	และก็ยังพบได้ในเซลล์ปกติของมนุษย์ด้วย.
	ยีนซีเอบีเอล พบในโครโมโซมที่เก้าของมนุษย์ 
	ซึ่ง เป็นโครโมโซมเดียวกับที่โรวลี่พบ การสลับตำแหน่งทางพันธุกรรมในผู้ป่วยลูคีเมียฉีบพลับชนิดไมอิลอยด์.
	เรื่องนี้ทำให้ทีมนักวิจัยสงสัยและสืบต่อไปที่โครโมโซมที่ยี่สิบสอง 
	จนพบว่า ในเซลล์มะเร็ง ยีนซีเอบีเอลได้ย้ายจากโครโมโซมที่เก้า ไปอยู่โครโมโซมที่ยี่สิบสอง
	และยังย้ายไปอยู่ตำแหน่งเดียวกันหมด ในเซลล์จากผู้ป่วยทั้งสิบเจ็ดคนที่ตรวจสอบ.
	ตำแหน่งที่ย้ายไป คือ 
	ยีนซีเอบีเอล ย้ายไปต่อจากยีนบีซีอาร์ (bcr) แล้วรวมกัน (เป็น บีซีอาร์ต่อเอบีเอล หรือ bcr-abl)
	ซึ่งเมื่อเซลล์นำยีนไปสร้างโปรตีน จะได้โปรตีนที่ผิดปกติ โดยต่อสายโปรตีนจากซีเอบีเอล เข้ากับสายโปรตีนจากบีซีอาร์.
	ผลคือโปรตีนสายยาวพิเศษจากบีซีอาร์ต่อเอบีเอล.
	
	หมายเหตุ ชีววิทยายึดหลักว่า กลไกหลักของชีวิตคือ 
	ดีเอ็นเอจะถูกถอดรหัสเป็นอาร์เอ็นเอ และอาร์เอ็นเอจะถูกแปลรหัสเพื่อไปสร้างโปรตีน.
	และโปรตีนเป็นเครื่องมือและกลไกหลักในการทำงานของชีวิต.
	ยีน ซึ่งเป็นลักษณะที่ถ่ายทอดทางพันธุกรรม จะถูกบันทึกไว้ด้วยดีเอ็นเอ.
	ถ้าเปรียบดีเอ็นเอเปรียบเหมือนโปรแกรมคอมพิวเตอร์ 
	(ซึ่งโปรแกรมคอมพิวเตอร์จะประกอบด้วยตรรกะของโปรแกรม ไวยากรณ์ของภาษา รวมถึง\textit{ข้อคิดเห็น} หรือ code comments)
	ยีนก็จะเปรียบเหมือนตรรกะของโปรแกรม. 
	
	\textbf{กลไกเบื้องหลังรูปแบบที่แสดง.}
	จากการศึกษาพฤติกรรมของโปรตีน
	โปรตีนจากซีเอบีเอล จะเป็นเอนไซม์ไทโรซีนคินเนส (tyrosine kinase).
	เอนไซม์ไทโรซีนคินเนสทำหน้าที่เพิ่มฟอสเฟตให้กับโปรตีน.
	การเพิ่มหรือลดฟอสเฟตกับโปรตีน เป็นเสมือนการเปิดหรือปิดการทำงานของโปรตีน แต่เปิดหรือปิดขึ้นกับชนิดของโปรตีน.
	โปรตีนจากซีเอบีเอล (โปรตีนจากเซลล์ปกติ) จะไม่ค่อยทำงาน
	ในขณะที่ โปรตีนจากบีซีอาร์ต่อเอบีเอล (โปรตีนจากเซลล์มะเร็ง)
	จะทำงานเกือบตลอดเวลา ทำงานเพิ่มฟอสเฟต. 
	%ส่งสัญญาณให้เซลล์แบ่งตัวตลอดเวลา.
	%นอกจากนั้น ยังส่งผลอีกอย่างคือ
	โปรตีนจากบีซีอาร์ต่อเอบีเอล จะเพิ่มฟอสเฟตไปให้กับ อาร์บีโปรตีน (Rb protein)
	ซึ่ง การเพิ่มฟอสเฟตมาก ๆ ให้กับอาร์บีโปรตีน เหมือนการปิดการทำงานของอาร์บีโปรตีน.
	อาร์บีโปรตีน ทำหน้าที่สำคัญในกระบวนการแบ่งตัวของเซลล์.
	เซลล์จะแบ่งตัวโดย การทำสำเนาดีเอ็นเอก่อน แล้วค่อยแบ่งตัว.
	กระบวนการนี้จะถูกควบคุมอย่างเป็นระเบียบ.
	อาร์บีโปรตีน ทำหน้าที่หยุดการสำเนาดีเอ็นเอในเซลล์. % หากพบความผิดปกติเกิดขึ้น.
เอนไซม์ไทโรซีนคินเนส ทำงานมากเกินไป 
ส่งผลเท่ากับ การปิดการทำงานของอาร์บีโปรตีน.
	การปิดการทำงานของอาร์บีโปรตีน ส่งผลเท่ากับ
	การปิดกลไกควบคุมการแบ่งตัวของเซลล์.
	%
	รูปแบบที่ผิดปกติในโครโมโซมที่โรวลี่พบบนโต๊ะกินข้าว เป็นสาเหตุของลูคีเมีย. 
	การสลับตำแหน่งทางพันธุกรรมทำให้เกิดยีนผิดปกติ.
	ยีนผิดปกติส่งผลให้เกิดเอมไซม์ผิดปกติ.
	เอนไซม์ผิดปกติส่งผลให้เกิดการปิดกลไกหยุดการแบ่งตัวของเซลล์
	และท้ายสุด ส่งผลให้เกิดโรคลูคีเมีย.
	%ความผิดปกติในการแบ่งตัวของเซลล์ และ
	%ความผิดปกติใน
	
	\textbf{ความเข้าใจปัญหานำไปสู่วิธีแก้ที่ดีกว่า.}
	วิธีการที่การแพทย์ใช้กับมะเร็ง แต่เดิมมีอยู่สามแนวทางหลัก คือ
	การผ่าตัด การฉายรังสี และการจัดยาสารพัดอย่าง เพื่อจะฆ่าเซลล์มะเร็ง.
	ก่อนความรู้เกี่ยวกับมะเร็งข้างต้นนี้
	วิธีการจัดยานี้ ตัวยาไม่ได้เจาะจงเฉพาะกับเซลล์มะเร็ง ดังนั้นผลลัพธ์ก็ต่าง ๆ กันไป และมีผลข้างเคียงสูง.
	
	ด้วยความเข้าใจกลไกและสาเหตุของลูคีเมียเรื้อรังชนิดไมอิลอยด์ 
	นักวิทยาศาสตร์ที่บริษัทไซบาไกกี สวิตเซอร์แลนด์ ได้แก่
	นิก ไลดอน (Nick Lydon) และอเล็กซ์ มาตเตอร์ (Alex Matter) 
	คิดว่า ถ้ายีนมะเร็งสร้างเอมไซม์ผิดปกติออกมา เป็นสาเหตุของโรค.
	เอมไซม์ผิดปกตินี้ ทำงานมากเกินไป 
	ดังนั้น
	สารที่ยับยั้งเอนไซม์นี้ได้ อาจช่วยหยุดการเติบโตของมะเร็งได้.
	แทนที่ ไลดอนกับมาตเตอร์จะใช้วิธีค้นหาสารนี้จากธรรมชาติ หรือใช้วิธีลองผิดลองถูก ตามวิถีของการค้นหายา ณ ยุคนั้น
	ไลดอนกับมาตเตอร์ใช้วิธีการออกแบบตามเหตุผล (rational design)
	ใช้กระบวนการทางเคมี เพื่อสังเคราะห์โมเลกุล 
	ที่จะเข้าไปจับกับตำแหน่งออกฤทธิ์ (active site) ของเอมไซม์ไทโรซีนคินเนสที่ผิดปกติ   และหยุดการทำงานของไทโรซีนคินเนสที่ผิดปกติ.

	หลังจากหลายปีที่ทำการทดลองทางเคมีและทดสอบ ไลดอนกับมาตเตอร์ก็ได้สารต่าง ๆ ที่ผ่านการคัดเลือกเบื้องต้น มาจำนวนหนึ่ง
	และทั้งคู่ได้ติดต่อกับนายแพทย์ไบอัน ดรุกเกอร์ (Brian Druker) ซึ่งทำงานที่มหาวิทยาลัยวิทยาศาสตร์สุขภาพโอเรกอน เพื่อทดสอบกับเซลล์จากผู้ป่วย.
	ดรุกเกอร์พบว่า มีสารอยู่ตัวหนึ่งจากที่ไลดอนกับมาตเตอร์ให้มา สามารถฆ่าเซลล์มะเร็งได้ โดยไม่ฆ่าเซลล์ปกติ เมื่อใช้ที่ความเข้มข้นต่ำ ๆ
	
	\textbf{กว่าจะได้เป็นยา.}
	ดรุกเกอร์ ไลดอน กับมาตเตอร์ ดีใจมากกับผลที่ได้ แต่ผู้บริหารของไซบาไกกีกลับไม่ค่อยสนใจมาก
	ไลดอนกับมาตเตอร์ ใช้เวลาเกือบปีในการโน้มน้าวผู้บริหาร ให้อนุมัติการทดสอบต่อในสัตว์.
	แต่ผลทดสอบพิษวิทยาในสุนัข ทำให้นักพิษวิทยาค่อนข้างเป็นห่วง
	เรื่องความปลอดภัยในมนุษย์.
	หลังจากนั้นไม่นาน บริษัทไซบาไกกีควบกิจการ เข้ากับบริษัทซานโดซ
	รวมเป็น บริษทโนวาร์ทิส.
	ไลดอนลาออก.
	โนวาร์ทิสทดสอบยาอีกครั้งในสัตว์ แต่นักพิษวิทยาก็ยังไม่สนับสนุน
	การทดลองยาในมนุษย์.
	
	ดรุกเกอร์มองจากมุมที่ต่างไป. โอกาสรอดของคนไข้ที่ดรุกเกอร์เห็น มันริบรี่มาก คนไข้ราว ๆ 25 ถึง 50 เปอร์เซ็นต์ ตายภายในหนึ่งปี หลังจากพบว่าเป็นมะเร็ง และสิ่งที่ดรุกเกอร์ทำได้ ก็แค่ยื้อเวลาเท่านั้น ไม่สามารถรักษาได้เลย.
	ดรุกเกอร์คิดว่าพิษจากยา น่าจะพอจัดการได้ โดยการติดตามผลที่ตัวคนไข้ และการปรับขนาดยา.
	ดรุกเกอร์ขอร้องมาตเตอร์ 
	และมาตเตอร์ก็ยืนยันกับโนวาร์ทิส ถึงความต้องการของยา 
	จนในที่สุด ผู้บริหารยอมให้มีการทดสอบทางคลีนิคในมนุษย์.
	การทดสอบเริ่มในปี ค.ศ. 1998 เกือบห้าปีหลังจากที่ดรุกเกอร์ได้ทดสอบยากับเซลล์มะเร็ง.
	
	ทีมของดรุกเกอร์ทดสอบยา กับคนไข้ลูคีเมียเรื้อรังชนิดไมอิลอยด์จำนวนหนึ่ง
	โดยค่อย ๆ เพิ่มขนาดยา พร้อมกับ ติดตามอาการของโรคและผลข้างเคียง อย่างใกล้ชิด.
	ประสิทธิผลของยา วัดได้จากการลดจำนวนลงของเซลล์เม็ดเลือดขาว.
	ในคนปกติ เซลล์เม็ดเลือดขาวจะอยู่ที่ 4000 ถึง 6000 เซลล์ต่อเลือดหนึ่งไมโครลิตร 
	แต่ในผู้ป่วยลูคีเมียเรื้อรังชนิดไมอิลอยด์ เซลล์เม็ดเลือดขาวจะอยู่ที่ 100000 ถึง 500000 เซลล์ต่อเลือดหนึ่งไมโครลิตร.
	ที่ยาปริมาณน้อย ๆ ทีมของดรุกเกอร์ไม่เห็นผลที่แตกต่าง
	แต่เมื่อเพิ่มปริมาณยาขึ้น คนไข้บางคน เริ่มมีจำนวนเม็ดเลือดลดลงสู่ช่วงปกติ.
	พอนำเลือดของคนไข้ไปตรวจสอบ ก็พบว่า
	สัดส่วนของเซลล์ที่มีฟิลาเดลเฟียโครโมโซมก็ลดลงด้วย.
	ยาทำงานได้ดี.

	โนวาร์ทิสสนัยสนุนยาในขั้นตอนต่อมาอย่างเต็มที่.
	คนไข้ในโครงการถูกติดตามผลต่อไปอีกหลายเดือน.
	เก้าสิบเจ็ดเปอร์เซ็นต์ของคนไข้ที่ได้รับยาในขนาดสูงสุด มีจำนวนเม็ดเลือดขาวกลับสู่ระดับปกติ ภายในเวลาสี่ถึงหกสัปดาห์.
เมื่อตรวจสอบเซลล์จากคนไข้ ก็พบว่า
คนไข้สามในสี่คน ไม่มีฟิลาเดลเฟียโครโมโซมอีกแล้ว.
	ผลลัพธ์ดีเยี่ยม และดีที่สุด ในประวัติของการรักษามะเร็งด้วยการจัดยา.
	โนวาร์ทิสยื่นจดทะเบียนยา ในชื่อ กลีเวค (Gleevec ชื่อสามัญ  Imatinib) กับสำนักงานอาหารและยาของสหรัฐอเมริกา
	และยาได้รับการรับรองในปี ค.ศ. 2001.
	
	กลีเวคเปลี่ยนสถานะการณ์ การรักษาลูคีเม่ียเรื้อรังชนิดไมอิลอยด์หน้ามือเป็นหลังมือ.
	โอกาสรอดระยะยาว (ยาวกว่า 8 ปี) เพิ่มขึ้นจากราว 45 เปอร์เซ็นต์ก่อนการรับรองกลีเวค ไปถึงเกือบ 90 เปอร์เซ็นต์ด้วยการใช้กลีเวค.
	
	ปัจจุบัน เชื่อว่า ร่างกายมนุษย์ ประกอบด้วย	เซลล์สองร้อยกว่าชนิด จากเซลล์ทั้งหมด
	จำนวนกว่าสามสิบเจ็ดล้านล้านเซลล์.
	จากยีนทั้งหมดราว ๆ สองหมื่นยีนของมนุษย์
	มีราว ๆ หนึ่งร้อยสี่สิบยีนที่มักกลายพันธุ์ และอาจก่อให้เกิดมะเร็ง.
	ร้อยสี่สิบยีนเหล่านี้ เป็นส่วนหนึ่งในกระบวนการที่ควบคุมการเปลี่ยนสภาพ หรือการอยู่รอดของเซลล์.
	มะเร็งส่วนใหญ่จะเกี่ยวข้องกับการกลายพันธุ์ สองถึงแปดยีน จากหนึ่งร้อยสี่สิบยีน.
	การรู้ว่ายีนไหน เกี่ยวข้องกับเนื้องอก หรือมะเร็งชนิดไหน จะช่วยให้เราสามารถจำแนกชนิดเนื้องอก ชนิดมะเร็ง ตามเงื่อนไขทางพันธุกรรม
	จะช่วยให้เราเข้าใจ และสามารถเชื่อมโยงการกลายพันธุ์ กับพฤติกรรมของมะเร็ง ไปจนถึงสามารถหาวิธีรักษาตามการกลายพันธุ์นั้น ๆ ได้.
	ปัจจุบัน มียากว่าสามสิบชนิด ที่รักษามะเร็งตามการกลายพันธุ์
	และก็ยังมีอีกมากที่อยู่ในกระบวนการวิจัย.
	
	เจเน็ต โรวลี่ เสียชีวิตในปีค.ศ. 2013 จากโรคมะเร็งรังไข่.
	ก่อนเสียชีวิต เธอได้ทำการนัดการผ่าตรวจสอบศพหลังจากที่เธอเสียชีวิต เพื่อที่นักวิจัยจะได้ศึกษาโรคต่อไป.
		
	\vspace{0.5cm}
	% start two column, bilingual environment
	\begin{Parallel}[c]{0.5\textwidth}{0.42\textwidth}
	\selectlanguage{english}
	\ParallelLText{
		``[T]he most critical thing we have learned about human life at the molecular level is that \textit{everything is regulated}.''
		\begin{flushright}
		---Sean B. Carroll
		\end{flushright}
	}
	\selectlanguage{thai}
	\ParallelRText{
		``สิ่งสำคัญที่สุดที่เราได้เรียนรู้เกี่ยวกับชีวิตมนุษย์ในระดับโมเลกุล คือ ทุก ๆ อย่างถูกควบคุมจัดระเบียบอย่างดี.''
		\begin{flushright}
		----ฉอน บี คาโรล
		\end{flushright}
	}
	\end{Parallel}
	\index{english}{quote!regulation}
	\index{english}{words of wisdom!Sean B. Carroll}


		
	\end{shaded}
}
%\end{minipage}


\section{การรู้จำตัวเลขลายมือ}
\label{sec: intro hand-written digit recognition}
\index{english}{hand-written digit recognition}
\index{thai}{การรู้จำตัวเลขลายมือ}

โปรแกรมรู้จำตัวเลขลายมือ\cite{MNIST20150311}
เป็นตัวอย่าง\textit{การรู้จำรูปแบบ}ด้วยวิธี\textit{การเรียนรู้ของเครื่อง}
ที่นิยมอ้างถึงกันมาก เพราะภารกิจช่วยให้เข้าใจภาพรวมได้ดี 
และงานไม่ซับซ้อนเกินไป 
มีข้อมูลเข้าถึงได้ง่าย สามารถใช้เป็นตัวอย่างทดลองปฏิบัติได้.
%
\textbf{การรู้จำตัวเลขลายมือ} (handwritten digit recognition) 
มีภารกิจ $T$ คือ จากภาพ (ข้อมูลสำรวจ) ซึ่งคอมพิวเตอร์มองเห็นเป็นค่าความเข้มของพิกเซลต่าง ๆ
แล้วให้โปรแกรมทาย ว่าภาพนั้นเป็นภาพแทนตัวเลขอะไร (ระบุประเภทรูปแบบ)
โดยภาพของตัวเลข เป็นภาพลายมือเขียนตัวเลขต่าง ๆ จากเลข 0 ถึงเลข 9 ดังแสดงในรูปที่~\ref{fig: example ZIP image digits}.
รูปตัวอย่างต่าง ๆ พร้อมเฉลย  
สามารถนำมาใช้ช่วยพัฒนาโปรแกรมได้ (ประสบการณ์ $E$).
สมรรถนะ $P$ วัดได้จากจำนวนรูปภาพที่ืทายได้ถูกต้อง.

%
\begin{figure}
\begin{center}
\includegraphics[width=0.75\textwidth]
{01Intro/hdrecog.png}
\end{center}
\caption[ตัวอย่างรูปตัวเลขจากลายมือเขียน]{ตัวอย่างรูปตัวเลขจากลายมือเขียน. แถวบนแสดงตัวอย่างข้อมูลนำเข้า ซึ่งเป็นภาพ.
แถวล่างแสดงเฉลยของแต่ละภาพ ซึ่งเป็นฉลากของแต่ละรูปแบบ}
\label{fig: example ZIP image digits}
\end{figure}
%

%หัวข้อชี้แจง ภาพรวมของการเรียนรู้ของเครื่อง.
%ตลอดตำราเล่มนี้ ผู้เขียนใช้ ฟอนต์เข้ม เช่น $\mathbf{x}$ เพื่อระบุถึงตัวแปรที่เป็นเวกเตอร์ หรือ เมทริกซ์, เพื่อเตือนให้ผู้อ่านนึกภาพตามได้ถูกต้อง เปรียบเทียบกับตัวแปรค่าเดี่ยว เช่น $x$.
%และให้ฟอนต์ \verb|sigmoid| เพื่อแสดงโค้ด ซึ่งต่างจาก $\mathbf{sigmoid}$ ที่แสดงถึงฟังก์ชันทางคณิตศาสตร์.

%\section{ภาพรวมของการเรียนรู้ของเครื่อง}
%\label{sec: ML overview}

รูปที่~\ref{fig: example ZIP image digits} แถวบน แสดงตัวอย่างรูปภาพ ที่เป็น\textbf{ข้อมูลนำเข้า} (หรืออินพุต input)
ของ\textit{โปรแกรมรู้จำตัวเลขลายมือ}.
แถวล่างแสดงตัวอย่าง\textit{ฉลาก}
ของ\textit{เฉลย}สำหรับข้อมูลนำเข้าที่อยู่ด้านบน.
%
%ฉลาก ซึ่งกรณีนี้ คือ 0 หรือ 1 หรือ 2 ... หรือ 9.
\textbf{ฉลาก} (label) จะระบุประเภทของ\textit{รูปแบบ}ที่สนใจ
ในกรณีนี้ มีสิบรูปแบบ.
รูปแบบของเลขศูนย์
รูปแบบของเลขหนึ่ง
รูปแบบของเลขสอง
ไปจนถึง
รูปแบบของเลขเก้า.
\textbf{เฉลย} (ground truth) คือฉลากที่ถูกต้อง.
\textbf{เฉลย} มีประโยชน์มาก โดยเฉพาะช่วยให้การวัด{สมรรถะนะ}ทำได้ง่าย.
สำหรับภาพใดก็ตาม หาก\textit{ฉลาก}ที่ทาย ตรงกับ\textit{เฉลย} ก็คือทายถูก
และในทางตรงกันข้าม หากไม่ตรง ก็คือทายผิด.

\textit{ฉลาก}ที่ทายจากโปรแกรม
บางครั้งอาจเรียกในชื่อที่ทั่วไปกว่า ว่า \textbf{ข้อมูลนำออก} (หรือเอาต์พุต output).
จากมุมมองของระบบแล้ว ภาพ คือ\textit{ข้อมูลนำเข้า}
ระบบ(โปรแกรมการรู้จำตัวเลขลายมือ) รับ\textit{ข้อมูลนำเข้า} 
ประมวลผล
และให้ค่า\textit{ข้อมูลนำออก} ซึ่งคือฉลากของรูปแบบเลขออกมา.
รูปที่~\ref{fig: intro system handwritten digit recog} 
แสดงแผนภาพ\textit{โปรแกรมการรู้จำตัวเลขลายมือ}จากมุมมองระบบ.

%
\begin{figure}
	\begin{center}
		\includegraphics[width=0.75\textwidth]
		{01Intro/system1.png}
	\end{center}
	\caption[แผนภาพแสดงระบบรู้จำตัวเลขลายมือ]{แผนภาพแสดงระบบรู้จำตัวเลขลายมือ 
		โดยมีระบบประมวลผล (ที่ทำนายฉลาก โดยสร้างตามแนวทางของการเรียนรู้ของเครื่อง) รับข้อมูลนำเข้าเป็นภาพ  และให้ข้อมูลนำออก ซึ่งเป็นฉลาก.}
	\label{fig: intro system handwritten digit recog}
\end{figure}
%

จากมุมมองนี้
\textit{ระบบประมวลผล} $f$ ทำหน้าที่แปลง
ข้อมูลนำเข้า $\bm{x}$ ที่เป็นภาพ
ไปเป็นข้อมูลนำออก $y$ ที่เป็นฉลาก.
และ
เพื่อให้เห็นภาพชัดเจน พิจารณา\textit{การรู้จำตัวเลขลายมือ} 
ที่ออกแบบสำหรับชุดข้อมูล\textit{เอมนิสต์}.
ชุดข้อมูล\textbf{เอมนิสต์}\cite{MNIST20150311} (MNIST)
เป็นข้อมูลขนาดใหญ่ของภาพพร้อมเฉลยของตัวเลขลายมือเขียน
ข้อมูลชุดนี้นิยมใช้ สำหรับทั้งศึกษาพัฒนา\textit{ระบบประมวลผลภาพ}
และ\textit{การเรียนรู้ของเครื่อง}
โดยข้อมูลได้ปรับปรุงจากข้อมูลของ\textit{สถาบันมาตราฐานและเทคโนโลยีแห่งชาตฺิ} (National Institute of Standards and Technology) สหรัฐอเมริกา.
ข้อมูลประกอบด้วย ภาพตัวเลขลายมือเขียนจำนวน 70,000 ภาพ%
\footnote{60,000 ภาพสำหรับ\textit{ชุดฝึกหัด} 
	และ 10,000 ภาพสำหรับ\textit{ชุดทดสอบ}.}
พร้อมเฉลย
แต่ละภาพมีขนาด 28x28 พิกเซล และเป็น\textit{ภาพขาวดำสองระดับค่า} (bi-level image) 
นั่นคือ แต่ละพิกเซลมีค่าเป็น 0 หรือ 1.
ดังนั้น หากเขียน
ระบบรู้จำตัวเลขลายมือเอมนิสต์นี้ เป็นฟังก์ชันคณิตศาสตร์ จะได้
$f:$ $\bm{x}$ $\mapsto y$ โดย $\bm{x}$ $\in \{0,1\}^{28 \times 28}$ และ $y \in \{0, 1, 2, \ldots, 9\}$.
%
\index{english}{MNIST}
\index{thai}{เอมนิสต์}
\index{thai}{ชุดข้อมูล!เอมนิสต์}
\index{english}{dataset!MNIST}


%
\begin{figure}
	\begin{center}
		\includegraphics[width=0.75\textwidth]
		{01Intro/MLnaiveSearch.png}
	\end{center}
	\caption[แผนภาพการค้นหาฟังก์ชันรู้จำตัวเลขลายมือ]{แผนภาพการค้นหาฟังก์ชันรู้จำตัวเลขลายมือ. 
	ด้วยข้อมูลตัวอย่าง โปรแกรมค้นหาฟังก์ชันคณิตศาสตร์ ที่สามารถแปลงข้อมูลนำเข้า ไปเป็น ข้อมูลนำออกที่ตรงกับเฉลยมากที่สุด.}
	\label{fig: intro ML naive search}
\end{figure}
%

จากมุมมองนี้ ปัญหาการสร้างระบบรู้จำตัวเลขลายมือ 
ถูกจำกัดกรอบลงมาเป็นการค้นหาฟังก์ชันคณิตศาสตร์ $f$ แทน.
แนวทาง\textit{การเรียนรู้ของเครื่อง}
ที่อาจทำได้คือ ใช้โปรแกรมค้นหาฟังก์ชัน $f$ นี้.
จากตัวอย่างข้อมูลภาพและเฉลยจำนวนมากที่มี
โปรแกรมจะหาฟังก์ชันคณิตศาสตร์ที่สามารถแปลงจากภาพในตัวอย่างไปเป็นฉลากที่ถูกต้องได้มากที่สุด.
รูป~\ref{fig: intro ML naive search} แสดงภาพตามแนวคิดนี้.


อย่างไรก็ตาม การค้นหาฟังก์ชันคณิตศาสตร์ใด ๆ นั้นมี\textbf{ปริภูมิค้นหา} (search space) ที่กว้างขวางมาก
จนในทางปฏิบัติแล้ว วิธีนี้ทำงานไม่ได้เลย.
วิธีแก้ปัญหาคือ
แทนที่จะค้นหาฟังก์ชันคณิตศาสตร์ใด ๆ 
แนวทาง\textit{การเรียนรู้ของเครื่อง}ที่ใช้งานได้ผล
คือ จะเลือก\textit{ฟังก์ชันคณิตศาสตร์อิงพารามิเตอร์} (parametric model)
ที่ พฤติกรรมการแปลงสามารถควบคุมได้
จากค่าของพารามิเตอร์ทีเลือกใช้
และใช้โปรแกรมค้นหาค่าของพารามิเตอร์แทน.
นั่นคือ สมมติมีฟังก์ชันคณิตศาสตร์ $f(\bm{x}, \bm{w})$ 
ที่พฤติกรรมการแปลงค่าข้อมูลนำเข้า $\bm{x}$ ไปเป็นข้อมูลนำออก $y$
เปลี่ยนแปลงและควบคุมได้จากค่า\textbf{พารามิเตอร์} (parameter) $\bm{w}$.
ดังนั้นแทนที่จะให้โปรแกรมค้นหาสมการคณิตศาสตร์ใด ๆ ที่เป็นได้ (ซึ่งมีจำนวนเกินคณานับ) 
และการค้นหามีโอกาสสำเร็จน้อยมาก
เปลี่ยนมาเป็น ให้โปรแกรมค้นหาค่าของพารามิเตอร์ $\bm{w}$ แทน
จะช่วยลดขนาดของปริภูมิิค้นหาลงมหาศาล และเพิ่มโอกาสสำเร็จขึ้นมาก.
นอกจากนั้น หาก\textit{ฟังก์ชันคณิตศาสตร์อิงพารามิเตอร์} ที่มักเรียกว่า \textbf{แบบจำลอง} (model) มีความสามารถในการปรับการแปลงมาก ๆ
การเลือกค่าพารามิเตอร์ ก็สามารถจะให้ผลได้ใกล้เคียงกับการค้นหาฟังก์ชันคณิตศาสตร์ใด ๆ ภายใต้บริบทของภารกิจที่ทำงานอยู่.
รูป~\ref{intro: parametric-model approach}
แสดงแนวทางของการใช้\textit{ฟังก์ชันคณิตศาสตร์อิงพารามิเตอร์}.
\index{english}{model}
\index{thai}{แบบจำลอง}

เนื่องจาก \textit{แบบจำลอง} ทำหน้าที่แปลงจากค่าอินพุตไปหาค่าที่จะทำนายสำหรับเอาต์พุต
ดังนั้น จึงอาจมองว่า แบบจำลอง\textit{ทำนาย}ค่าเอาต์พุต จากค่าอินพุตได้.
\textit{การรู้จำรูปแบบ}
ก็อาจมองจากมุมนี้ได้ว่า คือ การทำนาย (prediction ซึ่งบางครั้งเรียกว่า การอนุมาน inference)
รูปแบบที่สนใจ (เอาต์พุต) จากข้อมูล (อินพุต).
\index{english}{pattern recognition}
\index{thai}{การรู้จำรูปแบบ}
\index{english}{model!mapping}
\index{english}{model!prediction}
\index{english}{model!inference}
\index{thai}{แบบจำลอง!การแปลงค่า}
\index{thai}{แบบจำลอง!การทำนาย}
\index{thai}{แบบจำลอง!การอนุมาน}
รายละเอียดของแบบจำลองสำหรับการรู้จำตัวเลขลายมือเขียน
และวิธีการหาค่าพารามิเตอร์ที่ทำงานได้
จะอภิปรายโดยละเอียดในบทที่~\ref{chapter: ANN}.

%
\begin{figure}
	\begin{center}
		\includegraphics[width=0.75\textwidth]
		{01Intro/MLparametric}
	\end{center}
	\caption[การค้นหาค่าพารามิเตอร์ของแบบจำลอง]{แผนภาพการค้นหาฟังก์ชันรู้จำตัวเลขลายมือ. 
		ด้วยข้อมูลตัวอย่าง โปรแกรมค้นหาค่าพารามิเตอร์ของแบบจำลอง แทนการค้นหาฟังก์ชันคณิตศาสตร์.}
	\label{intro: parametric-model approach}
\end{figure}
%

\section{ประเภทของการเรียนรู้ของเครื่อง}
\label{sec: intro ML types} 

จากตัวอย่างการรู้จำตัวเลขลายมือเขียน
การวัดสมรรถนะสามารถทำได้ตรงมาตรงไป
เพราะว่า
รูปแบบของภาพตัวเลขลายมือเขียนมีเฉลย.
การประยุกต์ใช้\textit{การเรียนรู้ของเครื่อง} 
ไม่ได้จำกัดอยู่เฉพาะกับภารกิจที่มีเฉลยเท่านั้น.
แต่การที่มีเฉลย
ช่วยทำให้การวัดสมรรถนะสามารถทำได้อย่างตรงมาตรงไป.
ภารกิจชนิดที่มีเฉลยมาให้ด้วย จะเรียกว่า 
\textbf{การเรียนรู้แบบมีผู้สอน} (supervised learning).
การเรียนรู้แบบมีผู้สอนเอง ก็ยังอาจจำแนกออกได้เป็นหลายประเภท
ส่วนใหญ่นิยมจำแนกตามลักษณะ\textit{ข้อมูลนำออก}.
หาก\textit{ข้อมูลนำออก}เป็นการทายฉลาก 
หรือทาย\textit{ค่าวิยุต} (discrete value) ที่มีจำนวนจำกัด
นั่นคือ \textit{ข้อมูลนำออก} $y \in \{\alpha_1, \alpha_2, \ldots, \alpha_K \}$ เมื่อ $K$ แทนจำนวนค่าวิยุตทั้งหมดที่เป็นไปได้ และ $\alpha_i$ แทนค่าวิยุตต่าง ๆ ($i = 1, \ldots, K$)
ดังเช่น การทายฉลากของตัวเลขลายมือ $y \in \{0, 1, \ldots, 9\}$
กลุ่มนี้จะเรียกว่า ภารกิจ\textbf{การจำแนกกลุ่ม} (classification).
แต่หาก\textit{ข้อมูลนำออก}เป็นการทาย\textit{ค่าต่อเนื่อง} (continuous value)
นั่นคือ \textit{ข้อมูลนำออก} $y \in \mathbb{R}$
ดังเช่น การทายค่าดัชนีการเติบโตทางเศรษฐกิจ ซึ่งอาจเป็น 3.2 หรือ 4.5 หรือ ค่าจำนวนจริงใด ๆ (ซึ่งคงไม่เกิน 20 และหวังว่าจะไม่เป็น 0 หรือติดลบ)
กลุ่มนี้จะเรียกว่า ภารกิจ\textbf{การหาค่าถดถอย} (regression).
ค่าเฉลยสำหรับภารกิจ\textit{การหาค่าถดถอย} ก็จะเป็นค่าจำนวนจริงใด ๆ (ไม่ใช่ฉลาก แบบการรู้จำตัวเลขลายมือ).
\index{english}{supervised learning}
\index{english}{classification}
\index{english}{regression}
\index{english}{machine learning!supervised learning}
\index{english}{machine learning!supervised learning!classification}
\index{english}{machine learning!supervised learning!regression}
\index{thai}{การเรียนรู้แบบมีผู้ช่วยสอน}
\index{thai}{การจำแนกกลุ่ม}
\index{thai}{การหาค่าถดถอย}

หากภารกิจที่ทำไม่มีเฉลยจริง ๆ เลย 
ประเภทนี้เรียกว่า \textbf{การเรียนรู้แบบไม่มีผู้ช่วยสอน} (unsupervised learning).
\textit{การเรียนรู้แบบไม่มีผู้ช่วยสอน}
มีลักษณะที่หลากหลาย และวิธีการวัดสมรรถนะก็แตกต่างไปตามลักษณะเฉพาะ.
ตัวอย่างภารกิจต่าง ๆ ที่มีลักษณะแบบนี้ ได้แก่
\textit{การจัดกลุ่มข้อมูล} (clustering) ซึ่งคือ การจัดค่าข้อมูลต่าง ๆ ที่มีลักษณะคล้ายกัน ให้อยู่ในกลุ่มเดียวกัน
และค่าข้อมูลต่าง ๆ ที่มีลักษณะต่างกัน ให้อยู่ต่างกลุ่มกัน,
\textit{การประมาณความหนาแน่นของข้อมูล} (density estimation)
ซึ่งคือการเรียนรู้ค่าความน่าจะเป็นของข้อมูล,
\textit{การจัดลำดับข้อมูล} (ranking) ซึ่งคือ การเรียงลำดับข้อมูลตามเงื่อนไขที่ต้องการ,
\textit{การสร้างแบบจำลองหัวข้อ} (topic modeling) ซึ่งคือ การหาหัวข้อ (หรือตัวแทนหัวข้อ) ที่เหมาะสมกับเนื้อหาข้อความ,
\textit{การลดมิติของข้อมูล} (dimension reduction) ซึ่งคือ การลดจำนวนตัวแปรหรือส่วนประกอบของแต่ละจุดข้อมูลลง 
เพื่อให้การประมวลผลสามารถดำเนินการได้สะดวกรวดเร็วขึ้น,
\textit{การอนุมานข้อมูลขึ้นใหม่} (generative model) 
ที่สามารถใช้สร้างข้อมูลขึ้นมาใหม่ในรูปแบบเดิม หรือเปลี่ยนรูปแบบใหม่ในบริบทเดิม ซึ่งนำไปสู่การซ่อม การสร้าง หรือการดัดแปลง ภาพ เพลง ข้อความ ไปจนถึงวิดีโอต่าง ๆ ซึ่งกำลังได้รับความสนใจอย่างมากจากวงการศิลปะ ดนตรี บันเทิง และการออกแบบ แต่ก็เป็นส่วนที่สร้างความกังวลไม่น้อยให้กับวงการสื่อสารมวลชน กฎหมาย และการพิสูจน์หลักฐาน,
\textit{การเรียนรู้คุณลักษณะตัวแทน} (representation learning\cite{BengioEtAL2014a}),
\textit{การตรวจหารูปแบบใหม่} (novelty detection\cite{PimentelEtAl2014a})
และ\textit{การตรวจหารูปแบบผิดปกติ} (anomaly detection\cite{AnomalyDetectionSurvey2009}) เป็นต้น.
ภารกิจที่ไม่มีเฉลยนี้ครอบคลุมกว้างขวางมาก ซึ่งอาจรวมไปถึง
\textit{การหาค่าดีที่สุดด้วยวิธีการค้นหาเชิงศึกษาสำนึก} (optimization with heuristic search) ด้วย.
\textit{การหาค่าดีที่สุดด้วยวิธีการค้นหาเชิงศึกษาสำนึก}เอง ก็มีการประยุกต์ใช้อย่างกว้างขวางมาก 
และมีรูปแบบวิธีการที่หลากหลาย อาทิ
\textit{วิธีซิมมูเลทเต็ดแอนนิลลิง} (simulated annealing\cite{KirkpatrickEtAl1983a})
และ
\textit{จีเนติกอัลกอริทึม} (genetic algorithm\cite{Whitley1994}) เป็นต้น.
%เราจะศึกษา การเรียนรู้แบบไม่มีผู้ช่วยสอน ใน บทที่~\ref{chapter: others}.
\index{english}{unsupervised learning}
\index{thai}{การเรียนรู้แบบไม่มีผู้สอน}

นอกจาก\textit{การเรียนรู้แบบมีผู้สอน}ที่มีเฉลยชัดเจน
และ\textit{การเรียนรู้แบบไม่มีผู้สอน}ที่ไม่มีเฉลยเลย
ยังมีภารกิจอีกหลายประเภทที่ไม่อาจจัดอยู่ในสองกลุ่มข้างต้นได้
เช่น 
\textit{การเรียนรู้แบบกึ่งมีผู้ช่วยสอน} (semi-supervised Learning) ที่เป็นลักษณะภารกิจที่มีเฉลย แต่ข้อมูลที่ได้ มีทั้งส่วนที่มีเฉลย
และส่วนที่ไม่มีเฉลย และยังต้องการใช้ข้อมูลที่มีอยู่ให้คุ้มค่า โดยไม่ทิ้งข้อมูลที่ไม่มีเฉลยไปเฉย ๆ.
\textit{การเรียนรู้การแนะนำสินค้า} (recommendation learning\cite{Recommender2011, CollaborativeFiltering2009})
ที่สามารถใช้ผลการประเมินความพอใจจากตัวลูกค้าเอง กับสินค้าบางรายการ ประกอบกับ ผลประเมินจากลูกค้าคนอื่น ๆ เพื่อประเมินความชอบของลูกค้า กับสินค้ารายการที่ลูกค้าไม่ได้ประเมิน.
\textit{การเรียนรู้การแนะนำสินค้า} อาจมีลักษณะคล้าย ๆ \textit{การเรียนรู้คุณลักษณะตัวแทน} 
ที่พยายามเรียนรู้\textit{คุณลักษณะภายใน}ต่าง ๆ ของสินค้าที่ลูกค้าชอบ
แต่\textit{การเรียนรู้การแนะนำสินค้า}
มีการใช้ค่าเฉลยของบางคนกับบางรายการ และไม่ได้มีค่าเฉลยของทุกคนทุกรายการ
เพื่อไปทำนายความพอใจของลูกค้นทุกคน ในทุกรายการที่ไม่มีผลเฉลยได้.
\textit{การเรียนรู้แบบเสริมกำลัง} (reinforcement learning)
ที่เป็นภารกิจการตัดสินใจในแต่ละคาบเวลา ซึ่งอาจจะสามารถเห็นผลระยะสั้นได้ (การเรียนรู้แบบเสริมกำลังแบบสังเกตได้สมบูรณ์ หรือ fully observable reinforcement learning) หรืออาจต้องประมาณผลระยะสั้นด้วย
ซึ่งอาจจะประมาณผลระยะสั้นบางส่วน
หรืออาจจะต้องประมาณผลระยะสั้นทั้งหมดเลย (การเรียนรู้แบบเสริมกำลังแบบสังเกตได้บางส่วน หรือ partially observable reinforcement learning)
แต่เป้าหมายของภารกิจจริง ๆ คือการได้ผลประโยชน์ระยะยาวที่ดี
หรืออาจเป็นการหาสมดุลที่ดีระหว่างผลประโยชน์ระยะสั้น และผลประโยชน์ระยะยาว ซึ่งแม้จะมีผลลัพธ์ระยะสั้นมาให้สังเกตได้ 
แต่การประเมินผลประโยชน์ระยะยาวก็ไม่ได้ตรงมาตรงไป และไม่มีเฉลยจริง ๆ ของการตัดสินใจต่าง ๆ ให้ตรวจสอบ.
\textit{การเรียนรู้แบบเสริมกำลัง}ที่ดี จะต้องรักษาสมดุลระหว่างการเลือกการกระทำเพื่อที่จะได้ผลที่ดูเหมือนดีที่สุด กับการเลือกการกระทำเพื่อเรียนรู้ผลจากกระทำต่าง ๆ ในสถานะการณ์ต่าง ๆ.
ประเด็นความสมดุลนี้เรียกว่า \textit{ประเด็นของการใช้งานและการเรียนรู้} (issue of exploitation and exploration).
ลักษณะเด่นชัดอีกอย่าง ก็คือการที่ระบบการเรียนรู้แบบเสริมกำลัง มีปฏิสัมพันธ์กับสิ่งแวดล้อม หรือกล่าวได้ว่า ผลของการกระทำที่ระบบเลือกมีผลต่อประสบการณ์ที่ระบบจะเรียนรู้ (ดู~\cite{KatanyukulEtAl2011a} หรือ \cite{SuttonBarto1998a} สำหรับรายละเอียดเพิ่มเติม).
\index{english}{semi-supervised learning}
\index{english}{recommendation learning}
\index{english}{reinforcement learning}
\index{thai}{การเรียนรู้แบบกึ่งมีผู้ช่วยสอน}
\index{thai}{การเรียนรู้การแนะนำสินค้า}
\index{thai}{การเรียนรู้แบบเสริมกำลัง}



\section{การเรียนรู้ของเครื่องและศาสตร์ที่เกี่ยวข้อง}

การเรียนรู้ของเครื่องมักถูกเชื่อมโยงกับปัญญาประดิษฐ์.
\textbf{ปัญญาประดิษฐ์} (artificial intelligence หรือ คำย่อ AI)
เป็นศาสตร์ที่เป้าหมายคือการสร้างคอมพิวเตอร์ที่มีเหตุมีผล
เพื่อภารกิจเป้าหมาย โดยคอมพิวเตอร์จะสามารถเลือกการกระทำที่ช่วยให้ภารกิจมีโอกาสสำเร็จมากที่สุด บนพื้นฐานของสถานการณ์ที่รับรู้ และความรู้เดิมที่มี แม้จะมีความไม่แน่นอนเกี่ยวข้องอยู่.
\index{english}{artificial intelligence}
\index{thai}{ปัญญาประดิษฐ์}

รัสเซลและนอร์วิค\cite{RussellNorvig2009}
ได้ยกตัวอย่างศาสตร์ต่าง ๆ ที่จัดอยู่ภายใต้ขอบเขตของปัญญาประดิษฐ์
ได้แก่
ศาสตร์การเรียนรู้ของเครื่อง
ศาสตร์การแทนความรู้ (knowledge representation)
ศาสตร์การประมวลผลภาษาธรรมชาติ (natural language processing)
ศาสตร์คอมพิวเตอร์วิทัศน์ (computer vision)
และศาสตร์วิทยาการหุ่นยนต์ (robotics)
เป็นต้น.

แม้ว่าปัจจุบัน โดยเฉพาะในวงการธุรกิจมักใช้
คำว่าการเรียนรู้ของเครื่องและคำว่าปัญญาประดิษฐ์แทนกัน.
อย่างไรก็ตาม
ปัญญาประดิษฐ์ เน้นที่เป้าหมาย แต่ไม่ได้กำหนดวิธีการ และวิธีการหลาย ๆ อย่างของปัญญาประดิษฐ์ ไม่ได้สามารถจัดเป็นการเรียนรู้ของเครื่อง
ในขณะที่การเรียนรู้ของเครื่อง มีความหมายที่เน้นถึงแนวทางวิธีการที่จะทำภารกิจที่ต้องการ.
และแม้ศาสตร์และศิลป์ปัจจุบันของการเรียนรู้ของเครื่อง
จะได้สร้างความตื่นตัวอย่างมากกับสังคม
แต่ก็ยังไม่อาจนำปัญญาประดิษฐ์ไปสู่ศักยภาพสูงสุด
ซึ่งคือ การสร้างสติปัญญาระดับเดียวกับมนุษย์ ได้
โดยเฉพาะ 
เรื่องสามัญสำนึก (common sense) เรื่องการเข้าใจภาษาธรรมชาติ (natural language understanding)
เรื่องการเข้าใจความหมายระดับสูง เข้าใจสิ่งที่เป็นนามธรรม เป็นต้น.

\textbf{การทำเหมืองข้อมูล} (data mining)
\index{english}{data mining}\index{thai}{การทำเหมืองข้อมูล}
เป็นกระบวนการค้นหารูปแบบจากฐานข้อมูลขนาดใหญ่
ซึ่งมีหลายแง่มุมที่คล้ายกับการเรียนรู้ของเครื่อง โดยเฉพาะหลาย ๆ วิธีการของการทำเหมืองข้อมูลก็เป็นวิธีการเดียวกับวิธีการที่ใช้ในศาสตร์การเรียนรู้ของเครื่อง.
%
ในมุมมองหนึ่ง การทำเหมืองข้อมูลจะเน้นที่ รูปแบบที่จะได้มาจากข้อมูล 
ซึ่งโดยส่วนใหญ่
ข้อมูลก็จะอยู่ในรูปของ\textit{ฐานข้อมูลแบบสัมพันธ์} (relational database)
ในขณะที่การเรียนรู้ของเครื่องจะเน้นที่วิธีการ หรือมักเรียกว่า \textit{ขั้นตอนวิธี} (algorithm) มากกว่า.
%
อย่างไรก็ตาม ในหลาย ๆ ภารกิจ มันก็อาจยากที่จะวางเส้นแบ่งที่ชััดเจนได้
แต่ก็มีงานบางอย่างที่แสดงลักษณะเด่นของการทำเหมืองข้อมูล เช่น การหากฎความสัมพันธ์ (association rules)
และงานบางอย่างที่แสดงลักษณะเด่นของการเรียนรู้ของเครื่อง
เช่น การเรียนรู้แบบเสริมกำลัง.

อีกประเด็นหนึ่งที่อาจจะเป็นจุดต่างที่สำคัญ คือ ในขณะที่การเรียนรู้ของเครื่อง จะเน้นที่การค้นหารูปแบบโดยอัตโนมัติอย่างชัดเจน
แต่การทำเหมืองข้อมูลนั้นอาจทำโดยอาศัยมนุษย์เป็นหลัก
หรือใช้มนุษย์อยู่ในกระบวนการทำเหมืองข้อมูลอย่างมากได้.
ตัวอย่างเช่น ในการหากฎความสัมพันธ์นั้น ขั้นตอนวิธีการหากฎความสัมพันธ์ อาจจะช่วยหาความสัมพันธ์ระหว่างสินค้าต่าง ๆ ที่ซื้อด้วยกันได้
จากความถี่ที่สินค้าเหล่านั้นปรากฏอยู่บ่อย ๆ ในรายการซื้อเดียวกัน.
แต่หากจะใช้หาความสัมพันธ์ระหว่างคุณลักษณะของพนักงาน
กับพฤติกรรมการทำงาน
อาจจะต้องอาศัยมนุษย์ช่วยกลั่นกรอง\textit{ความสัมพันธ์ที่ไม่เป็นสาระ} (trivial association) ออก 
เช่น 
ความสัมพันธ์ที่พบว่าพนักงานที่ทำงานน้อยกว่าสามเดือนไม่เคยลากิจ 
ซึ่งเหตุผลจริง ๆ เป็นเพราะว่า เขายังไม่มีสิทธิลา 
แต่หากสรุปผลไปผิดว่า พนักงานใหม่ขยันกว่า เพราะไม่เคยลากิจเลย 
%อย่างดีก็อาจเป็นเรื่องขำขันในสำนักงาน หรืออย่างแย่คือ
ซึ่งอาจทำให้เกิดการเข้าใจผิดได้ หรือ 
ความสัมพันธ์ที่พบว่าพนักงานที่ลาคลอดทั้งหมดเป็นผู้หญิง 
ซึ่งแม้เป็นความจริง แต่ก็ไม่ได้มีสาระประโยชน์อะไร จึงจำเป็นต้องอาศัยมนุษย์ช่วยกลั่นกรองรูปแบบความสัมพันธ์ที่พบ.

%
%
%
%จริง ๆ แล้วมีหลาย ๆ วิธี ที่มันเกิดมาก่อนที่จะมีคำว่าการทำเหมืองข้อมูล หรือคำว่าการเรียนรู้ของเครื่องอีก เช่น \textit{วิธีการแบ่งกลุ่มแบบเคมีนส์} (k-means clustering)
%\textit{วิธีการวิเคราะห์ส่วนประกอบหลัก} (principal component analysis หรือ คำย่อ PCA) เป็นต้น.
%แต่ก็มีบางวิธีที่เป็นวิธีที่เกิดมาพร้อม ๆ กับชื่อการทำเหมืองข้อมูล หรือเกิดมาพร้อม ๆ กับชื่อการเรียนรู้ของเครื่อง
%และก็ยังมีอีกหลาย ๆ วิธีที่เกิดขึ้นมาหลังจากศาสตร์เหล่านี้เป็นที่รู้จักแล้ว.

นอกจากปัญญาประดิษฐ์
และการทำเหมืองข้อมูลแล้ว
ยังมีศาสตร์อื่นอีกที่มีความหมายทับซ้อนคลุมเครือกับการเรียนรู้ของเครื่อง.
\textbf{วิทยาการข้อมูล} (data science)
\index{english}{data science}\index{thai}{วิทยาการข้อมูล}
รวมศาสตร์ต่าง ๆ
เพื่อวิเคราะห์ข้อมูล ทำความเข้าใจเรื่องราว และทำนายประเด็นที่สนใจ
ไปจนถึงแสดงข้อมูล แสดงมุมมองและนำเสนอผลวิเคราะห์.
ด้วยลักษณะของวิทยาการข้อมูล
วิทยาการข้อมูลครอบคลุมเนื้อหาส่วนหนึ่งของสถิติศาสตร์,
การเรียนรู้ของเครื่อง,
การทำเหมืองข้อมูล,
การจัดการฐานข้อมูล,
และ\textit{การสร้างมโนภาพ}สำหรับข้อมูลและสารสนเทศ
(data and information visualization)
รวมถึงเทคโนโลยีต่าง ๆ ที่ใช้จัดการข้อมูลปริมาณมหาศาล 
เช่น แมปรีดิวซ์ (MapReduce).
%
%\textbf{ประสาทวิทยาศาสตร์เชิงคำนวณ} (computational neuroscience)
%เป็นศาสตร์
%ที่ใช้ทฤษฎีและเครื่องมือทางคณิตศาสตร์
%เพื่อศึกษาการทำงานของสมอง
%ซึ่งอาจเกี่ยวข้องกับหลาย ๆ แนวทาง 
%รวมถึงวิศวกรรมไฟฟ้า
%วิทยาการคอมพิวเตอร์
%ฟิสิกส์.
%ในขณะที่มีหลายขั้นตอนวิธีของการเรีียนรู้ของเครื่องได้รับแรงบันดาลใจจากการศึกษาการทำงานของสมอง
%แต่จุดประสงค์หลัก คือ การทำภารกิจที่ต้องการ เช่นการรู้จำรูปแบบต่าง ๆ.
%สิ่งนี้ต่างจาก ประสาทวิทยาศาสตร์เชิงคำนวณ ที่มีจุดประสงค์เพื่อ

หมายเหตุ การแบ่งแยกหรือจัดฉลากสำหรับศาสตร์ต่าง ๆ เหล่านี้ไม่ได้มีเส้นแบ่งที่ชัดเจน และในทางปฏิบัติก็ไม่ได้มีเส้นแบ่ง หรือไม่ได้มีข้อจำกัด หรือไม่ได้มีความจำเป็นใดที่ต้องแบ่งให้เด็ดขาด.
ภารกิจที่ทำ เป็นสิ่งสำคัญที่สุด. 
นั่นหมายถึงว่า เทคนิคใด ๆ ก็ตามที่เป็นประโยชน์ ที่ใช้งานได้ ที่เหมาะสมกับงาน ถือว่าดีทั้งนั้น ไม่ว่ามันจะเรียกหรือจัดเป็นศาสตร์ใด หรือแม้แต่มันจะเป็นแนวทางใหม่ที่อาจยากที่จะถูกจัดให้อยู่ภายใต้ศาสตร์ใดก็ตาม.


บางครั้ง การเรียนรู้ของเครื่อง
ถูกสับสนกับการเรียนรู้เชิงลึก.
\textbf{การเรียนรู้เชิงลึก} (deep learning)
\index{english}{deep learning}
\index{thai}{การเรียนรู้เชิงลึก}
เป็น
การเรียนรู้ของเครื่อง
ที่เน้นการใช้แบบจำลอง 
ที่มีความสามารถในการแปลงข้อมูลสูง (model with high representative power)
โดยใช้การประมวลผลเป็นลำดับชั้น
เรียกว่า แบบจำลองเชิงลึก (deep model) หรือโครงข่ายเชิงลึก (deep network).
ความสามารถของแบบจำลองเชิงลึก
ได้มาจากการที่แบบจำลองมีโครงสร้างที่มีการคำนวณในลักษณะเป็นขั้น ๆ ลำดับชั้น. ผลจากขั้นหนึ่งส่งไปคำนวณต่อที่อีกขั้นหนึ่ง
และทำการคำนวณเช่นนี้ต่อไปหลาย ๆ ขั้น (ที่มาของคำว่า ลึก).
%แม้ว่า การเรียนรู้เชิงลึก เป็นแนวทางที่ช่วยให้สามารถสร้างแบบจำลองความสามารถสูง และสามารถทำภารกิจหลายอย่างได้ดี โดยเฉพาะภารกิจที่เกี่ยวกับคอมพิวเตอร์วิทัศน์.
%แต่มีภารกิจอีกจำนวนมาก ที่ไม่มีความจำเป็นต้องใช้\textit{การเรียนรู้ลึก}.
%ตำราเล่มนี้ เริ่มด้วย การอธิบาย พื้นฐานที่จำเป็น
%การเรียนรู้ของเครื่องโดยทั่วไป
%กลไก และประเด็นที่สำคัญ
%ซึ่งเป็นเรื่องทั่วไปที่ควรเข้าใจ
%ไม่ว่าแบบจำลองที่ใช้จะลึก หรือไม่
%แล้วตามด้วยการอภิปราย
บท~\ref{chapter: Deep Learning}
อภิปราย\textit{การเรียนรู้เชิงลึก} ในรายละเอียด.

นอกจากศาสตร์ต่าง ๆ ดังกล่าวแล้ว
ประเด็นของ\textit{ข้อมูลมหัต} (big data)
\index{thai}{ข้อมูลมหัต}
\index{english}{big data}
เป็นอีกหนึ่งเรื่องที่มักถูกสับสนกับการเรียนรู้ของเครื่อง.
\textit{ข้อมูลมหัต} อ้างถึงชุดข้อมูลที่มีปริมาณข้อมูลขนาดใหญ่ และมีความหลากหลายของชนิดข้อมูล 
โดยลักษณะสำคัญของข้อมูลที่เป็น\textit{ข้อมูลมหัต} คือ ข้อมูลมีปริมาณมาก
ข้อมูลเพิ่มขึ้นอย่างรวดเร็ว
และข้อมูลมีความหลากหลายมาก
(ซึ่งมักถูกอ้างถึง โดยย่อว่า 3 Vs สำหรับ
high volume, high velocity, และ high variety).
เมื่อเปรียบเทียบ\textit{ข้อมูลมหัต}กับการเรียนรู้ของเครื่อง
กล่าวโดยง่าย 
คือ
ในขณะที่\textit{ข้อมูลมหัต}เน้นที่ลักษณะและความท้าทายของการจัดการกับข้อมูลในเชิงปริมาณ
\textit{การเรียนรู้ของเครื่อง}เน้นที่ภารกิจที่จะทำ โดยมักใช้ข้อมูลประกอบ
เพื่อการบรรลุภารกิจ.
การทำ\textit{ข้อมูลมหัต} อาจต้องการเพียงฮาร์ดแวร์  ระบบฐานข้อมูล รวมถึงโครงสร้างข้อมูลที่มีประสิทธิภาพขึ้น เพื่อรองรับความท้าทายเชิงปริมาณของข้อมูล.
การทำ\textit{ข้อมูลมหัต} อาจจะใช้หรือไม่ใช้แนวทาง\textit{การเรียนรู้ของเครื่อง}ก็ได้ ขึ้นกับจุดประสงค์.
\textit{การเรียนรู้ของเครื่อง}เอง เมื่อใช้งานกับข้อมูลที่มีลักษณะ\textit{ข้อมูลมหัต}
อาจต้องการเทคนิคและกลไกที่ช่วยจัดการความท้าทายเชิงปริมาณ
และอาจต้องการ\textit{ขั้นตอนวิธี}ใหม่ ที่เหมาะกับกับปริมาณ ความเร็ว และความหลากหลายของ\textit{ข้อมูลมหัต}.
หากเปรียบเทียบ\textit{ข้อมูลมหัต}เป็นถนนรุกรังที่ยาวมาก ๆ
\textit{การเรียนรู้ของเครื่อง}ก็อาจเปรียบเป็นรถยนต์เบนซิน.
บางครั้งก็ทำงานด้วยกัน บางครั้งก็ไม่.
แต่เมื่อทำงานด้วยกัน หากเปลี่ยนเป็นยางสำหรับถนนรุกรัง
เปลี่ยนช่วงล่างให้ทนทานขึ้น
และเตรียมน้ำมันเชื้อเพลิงเผื่อไว้ให้เพียงพอ อาจจะช่วยให้ขับผ่านไปสู่เป้าหมายได้โดยสวัสดิภาพ.

เนื้อหาของตำราเล่มนี้ เน้นพื้นฐาน และศาสตร์และศิลป์ที่สำคัญ
ของการเรียนรู้ของเครื่อง
ซึ่งแม้หลาย ๆ เรื่อง
จะเป็นเนื้อหาของศาสตร์อื่น ๆ เช่นกัน
แต่ตำรานี้ไม่ได้มีจุดประสงค์เพื่อ
ครอบคลุมปัญญาประดิษฐ์ 
การทำเหมืองข้อมูล หรือศาสตร์อื่น ๆ ที่เกี่ยวข้อง.
ผู้อ่านที่สนใจศาสตร์ที่เกี่ยวข้องเหล่านี้ สามารถศึกษาได้จากตำรา
และแหล่งเรียนรู้เฉพาะของแต่ละศาสตร์.

%\subsection{ศาสตร์พื้นฐาน}
%
%\textbf{การหาค่าดีที่สุด} (optimization)
%
%\textbf{ทฤษฎีตัดสินใจ} (decision theory)
%
%\textbf{ความน่าจะเป็น} (probability theory)
%
%\textbf{ทฤษฎีสารสนเทศ} (information theory)
%




%\section{Glossary}
\section{อภิธานศัพท์}

\begin{description}
	
	\item[รูปแบบ (pattern):] การซ้ำเชิงโครงสร้าง
	\index{english}{pattern}
	\index{thai}{รูปแบบ}

	\item[การรู้จำรูปแบบ (pattern recognition):] 
	การทายค่าหรือระบุฉลากของรูปแบบ จากข้อมูลนำเข้า
	\index{english}{pattern recognition}
	\index{thai}{การรู้จำรูปแบบ}
	
	\item[การเรียนรู้ของเครื่อง (machine learning):] ศาสตร์ของการทำให้คอมพิวเตอร์มีความสามารถที่จะเรียนรู้ 
	ที่จะทำนาย หรือตัดสินใจได้ โดยที่ไม่ต้องเขียนโปรแกรมวิธีทำตรง ๆ
\index{english}{machine learning}
\index{thai}{การเรียนรู้ของเครื่อง}

    \item[การรู้จำตัวเลขลายมือ (handwritten digit recognition):]
โปรแกรมทายภาพ ว่าภาพนั้นแทนตัวเลขอะไร
โดยภาพของตัวเลข เป็นภาพลายมือเขียนตัวเลขต่าง ๆ จากเลข 0 ถึงเลข 9    
\index{thai}{การรู้จำตัวเลขลายมือ}  
\index{english}{handwritten digit recognition}  

    \item[ข้อมูลนำเข้า หรืออินพุต (input):]
ตัวแปรต้น หรือข้อมูลที่โปรแกรมรับเข้า
\index{english}{input}
\index{thai}{ข้อมูลนำเข้า}
\index{thai}{อินพุต}


    \item[ข้อมูลนำออก หรือเอาต์พุต (output):]
ตัวแปรตาม หรือค่าข้อมูลที่โปรแกรมต้องให้ออกมา
\index{english}{output}
\index{thai}{ข้อมูลนำออก}
\index{thai}{เอาต์พุต}

    \item[ฉลาก (label):]
ตัวแปรตาม หรือข้อมูล ที่ระบุประเภท หรือชื่อของรูปแบบที่สนใจ
\index{english}{label}
\index{thai}{ฉลาก}

	\item[เอมนิสต์ (MNIST):]
ข้อมูลขนาดใหญ่ของภาพพร้อมเฉลยของตัวเลขลายมือเขียน ซึ่งนิยมใช้ทดสอบระบบการรู้จำตัวเลขลายมือ
\index{english}{MNIST}
\index{thai}{เอมนิสต์}

	\item[แบบจำลอง (model):] สมการคณิตศาสตร์ที่ใช้คำนวณค่าข้อมูลนำออก จากค่าข้อมูลนำเข้า
	และค่าพารามิเตอร์ที่เลือกใช้ ซึ่งข้อมูลนำออกมักเป็นค่าทำนายของสิ่งที่สนใจ  
\index{english}{model}
\index{thai}{แบบจำลอง}


	\item[พารามิเตอร์ (parameter):]
	ตัวแปรที่ค่าของมัน สามารถปรับเปลี่ยนการทำนายของแบบจำลอง โดยเปลี่ยนพฤติกรรมการแปลง\textit{ค่าข้อมูลนำเข้า}ไปเป็น\textit{ข้อมูลนำออก}
\index{english}{parameter}
\index{english}{weight}
\index{thai}{พารามิเตอร์}
\index{thai}{ค่าน้ำหนัก}	
	
	
	\item[การเรียนรู้แบบมีผู้สอน (supervised learning):]
	ภารกิจการทำนายหรือตัดสินใจ ที่มีเฉลยที่ถูกต้องให้
\index{english}{supervised learning}
\index{thai}{การเรียนรู้แบบมีผู้สอน}


	\item[การเรียนรู้แบบไม่มีผู้สอน (unsupervised learning):]
    ภารกิจการทำนายหรือตัดสินใจ ที่ไม่มีเฉลยที่ถูกต้องให้
\index{english}{unsupervised learning}
\index{thai}{การเรียนรู้แบบไม่มีผู้สอน}
    
    \item[การจำแนกกลุ่ม (classification):]
    ภารกิจการทำนายฉลาก หรือทำนายค่าวิมุตที่มีจำนวนจำกัด
\index{english}{classification}
\index{thai}{การจำแนกกลุ่ม}
    
    
    \item[การหาค่าถดถอย (regression):]
    ภารกิจการทำนายค่าที่เป็นจำนวนจริง
\index{english}{regression}
\index{thai}{การหาค่าถดถอย}


%	\item[ปัญญาประดิษฐ์ (artificial intelligence):] ความหมาย
%\index{artificial intelligence}
%\index{ปัญญาประดิษฐ์}

%	\item[การทำเหมืองข้อมูล (dataming):]
%
%	\item[วิทยาการข้อมูล (data science):]
%
%	\item[ประสาทวิทยาศาสตร์เชิงคำนวณ (computational neuroscience):]

	
\end{description}


