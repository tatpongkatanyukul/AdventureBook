\section{แบบฝึกหัด}

\begin{Parallel}[c]{0.48\textwidth}{0.34\textwidth}
	\selectlanguage{english}
	\ParallelLText{
		``Sail away from the safe harbor. 
		Catch the trade winds in your sails.  
		Explore. Dream. Discover.'' 
		\begin{flushright}
			---Mark Twain
		\end{flushright}
	}
	\selectlanguage{thai}
	\ParallelRText{
		``ออกเรือไปจากอ่าวที่ปลอดภัย 
		กางใบไปกับลมสำเภา 
		ออกสำรวจ ออกฝัน ออกค้นพบ.
		''
		\begin{flushright}
			---มาร์ค ทเวน
		\end{flushright}
	}
\end{Parallel}
\index{english}{words of wisdom!Mark Twain}
\index{english}{quote!courage!explore}
\vspace{1cm}



เพื่อเป็นการทบทวนทักษะการเขียนโปรแกรม
%บทนี้แบ่งแบบฝึกหัดเป็นสองส่วน
%ส่วนแรกเป็น
แบบฝึกหัดทบทวนการเขียนโปรแกรมทั่วไป. 
%และส่วนที่สองเป็นแบบฝึกหัดเพื่อเสริมความเข้าใจของเนื้อหาในบท และปูพื้นฐานสำหรับเนื้อหาในบทต่อ ๆ ไป
%
%\subsection{แบบฝึกหัดเขียนโปรแกรม}
%
แม้ ตำรา จะอภิปรายเนื้อหา \textit{ศาสตร์การรู้จำรู้แบบและการเรียนรู้ของเครื่อง} โดยทั่วไป
แต่เพื่อให้ผู้อ่านเข้าใจอย่างชัดเจน ตัวอย่างโปรแกรมที่ใช้จะแสดงด้วยภาษาไพธอน (เวอร์ชั่นสาม).
แบบฝึกหัดเขียนโปรแกรมนี้ออกแบบมา
เพื่อทบทวนทักษะการเขียนโปรแกรมด้วยภาษาไพธอน

\begin{Exercise}
\label{prog: print}
จงเขียนโปรแกรมเพื่อพิมพ์ข้อความต่อไปนี้
ออกมาที่หน้าจอ โดยให้มีการขึ้นบรรทัดตามที่แสดง 

\begin{verbatim}
Bruce Lee:
Knowing is not enough, we must apply.
Willing is not enough, we must do.
\end{verbatim}

\textit{คำใบ้} ลองคำสั่ง \texttt{print}

\end{Exercise}

\begin{Exercise}
\label{prog: input}
จงเขียนโปรแกรมเพื่อรับตัวเลขจำนวนเต็มจากผู้ใช้ และพิมพ์ตัวเลขนั้น พร้อมค่ากำลังสองออกมา ดังตัวอย่าง

\begin{verbatim}
Enter a number: 4
4 is squared to 16
\end{verbatim}
เมื่อ \verb|4| ในท้ายบรรทัดแรกเป็นอินพุตจากผู้ใช้

\textit{คำใบ้} 
(1) ลองคำสั่ง \texttt{input}, 
(2) เปรียบเทียบผลลัพธ์ของ \verb|"3"+"5"|
กับของ \verb|int("3") + 5|
และ (3) ลองคำสั่ง \verb|5**2|
	
\end{Exercise}

\begin{Exercise}
\label{prog: assignment cast away}
จากภาพยนต์เรื่องคนหลุดโลก (Cast Away ค.ศ. 2000)
ชัค โนแลนด์ รอดชีวิตจากเครื่องบินตก และติดอยู่ที่เกาะร้าง
เขาลองคำนวณหาโอกาส
ที่ทีมค้นหาจะพบเขาที่เกาะร้าง
ดังนี้
(1) เครื่องบินด้วยความเร็ว $v$ ไมล์ต่อชั่วโมง.
(2) เครื่องบินติดต่อกับหอควบคุมการบินไม่ได้ เป็นเวลา $T$ ชั่งโมงก่อนจะตก.
ชัคต้องการคำนวณหาพื้นที่ที่ทีมกู้ภัยต้องค้นหา.

จงเขียนโปรแกรม เพื่อคำนวณพื้นที่ค้นหา โดยรับความเร็วเครื่องบิน $v$ ไมล์ต่อชั่วโมง
และเวลา $T$ ชั่วโมง จากที่ขาดการติดต่อจนถึงเครื่องตก.
โปรแกรมรายงานออกมาเป็นพื้นที่ตารางไมล์  และเปรียบเทียบกับพื้นที่ของประเทศไทย โดยพื้นที่ประเทศไทย มีขนาดประมาณ 513120 ตารางกิโลเมตร หรือ 198120 ตารางไมล์.

ตัวอย่างโปรแกรม
\begin{verbatim}
Plane speed (mph): 475
Time from the last contact to crash (h): 1
Search area = 708821.84 sq.mi.
That is 3.58 times the size of Thailand.
\end{verbatim}
เมื่อ 475 ในบรรทัดแรก และ 1 ในบรรทัดที่สอง เป็นอินพุตจากผู้ใช้
และ 708821.84 กับ 3.58 เป็นผลการคำนวณ

\textit{คำใบ้} 
(1) พื้นที่ค้นหา $a$ ตารางไมล์ คำนวณได้จาก $a = \pi r^2$ เมื่อ $r = v \cdot T$.
(2) คำสั่ง \texttt{round} สามารถใช้ช่วยปัดเศษได้
เช่น \verb|round(21.842,2)| จะให้ผลลัพธ์เป็น $21.84$ (ปัดเป็นเลขทศนิยมสองตำแหน่ง).
(3) มอดูล \texttt{math} มีฟังก์ชันและค่าคงที่ทางคณิตศาสตร์ต่าง ๆ ที่มีประโยชน์.
มอดูล \texttt{math} จะถูกนำเข้ามาใช้งานได้ โดยคำสั่ง \verb|import math|
และค่า $\pi$ สามารถเรียกได้จาก \verb|math.pi|

\end{Exercise}

\begin{Exercise}
	\label{prog: assignment tennis}

จงเขียนฟังก์ชันคำนวณเวลาที่ลูกเทนนิสวิ่งจากหน้าไม้ของผู้เซิร์ฟไปถึงท้ายสนามเทนนิสฝั่งผู้รับ
และคำนวณพลังงานที่ใช้ในการเซิร์ฟ ในหน่วยจูล (Joules ตัวย่อ J) 
และในหน่วยแคลอรี่ (calories ตัวย่อ cal)
โดยฟังก์ชันรับ
ค่าน้ำหนักของลูกบอล $m$ กรัม
ค่าความเร็วสูงสุดของลูกบอล $v$ ในหน่วยกิโลเมตรต่อชั่วโมง
และความยาวของสนามเทนนิส $d$ เมตร.
สมมติว่าไม่มีแรงต้านทางอากาศ
ไม่มีผลจากแรงดึงดูดของโลก
ไม่มีผลจากการกระเด้งที่ผิวสนาม
และคิดประมาณระยะทางเฉพาะในแนวราบทิศทางความยาวสนาม.

ตัวอย่างการเรียกใช้ฟังก์ชัน
\begin{verbatim}
time, energy, cal = serve(200, 23.8, 58)
print(time, 's')
print(energy, 'J')
print(cal, 'cal')
\end{verbatim}
เมื่อ 200 คือความเร็วสูงสุดของลูกบอล ในหน่วยกิโลเมตรต่อชั่วโมง
23.8 คือความยาวสนาม ในหน่วยเมตร
58 คือน้ำหนักลูกบอล ในหน่วยกรัม
และ \verb|serve| คือฟังก์ชันที่ใช้คำนวณ.
ผลลัพธ์คือ
\begin{verbatim}
0.8568 s
89.51 J
21.39 cal
\end{verbatim}

\textit{คำใบ้}
(1) ระยะที่ลูกบอลเดินทางประมาณจากความยาวสนาม.
(2)
เวลาที่ลูกบอลวิ่ง $t$ คำนวณจาก
$d = v_0 + \frac{1}{2} \cdot a \cdot t^2$ 
เมื่อ $v_0$ คือความเร็วต้น (ประมาณเป็นศูนย์ ขณะลูกกระทบหน้าไม้)
และ $a$ เป็นความเร่งเฉลี่ยของลูกบอล ในหน่วย เมตรต่อวินาทีกำลังสอง.
(3) ความเร่งเฉลี่ยของลูกบอล $a$ ประมาณได้จาก $a = v/t$.
(4) แรงเฉลี่ยที่ใช้ $f$ ในหน่วยนิวตัน คำนวณได้จาก $f = m \cdot a$.
(5) พลังงานที่ใช้ $e$ ในหน่วยจูล ประมาณได้จาก $e = f \cdot d$.
(6) หนึ่งแคลอรี่เท่ากับ 4.184 จูล.
(7) ไพธอนกำหนดฟังก์ชันด้วยไวยากรณ์
\begin{verbatim}
def func_name(arg1, arg2, arg3):
    # function body
    ...
    return output1, output2
\end{verbatim}
(8)
แนวทางปฏิบัติที่ดีในการเขียนโปรแกรมไพธอน คือ ส่วนของโปรแกรมหลักจะเขียนอยู่ในรูปแบบ
\begin{verbatim}
if __name__ == '__main__':
    # main program
    ...
\end{verbatim}

\end{Exercise}

\begin{Exercise}
	\label{prog: conditions delivery}
บริษัทขนส่งแห่งหนึ่ง 
คิดค่าบริการซึ่งประกอบด้วย
ค่าบริการส่ง (คิดตามพื้นที่)
และค่าส่งของ (คิดตามน้ำหนัก)
โดย
ค่าบริการส่ง
คิด 50 บาท ถ้าส่งในเขตจังหวัดขอนแก่น
และคิด 100 บาท ถ้าส่งนอกเขตจังหวัดขอนแก่น.
ค่าส่งของ คิดดังนี้
(1) คิด 8 บาทต่อกิโลกรัม สำหรับของน้ำหนักไม่เกิน 10 กิโลกรัม
(2) คิด 12 บาทต่อกิโลกรัม สำหรับของน้ำหนักเกิน 10 กิโลกรัม แต่ไม่เกิน 20 กิโลกรัม
และ (3) คิด 15 บาทต่อกิโลกรัม สำหรับของน้ำหนัก 20 กิโลกรัมขึ้นไป.

จงเขียนฟังก์ชันรับที่อยู่ และน้ำหนักของ แล้วคำนวณค่าส่งของบริษัทแห่งนี้.

ตัวอย่างการเรียกใช้ฟังก์ชัน
\begin{verbatim}
cost = delivery_kk("Khon Kaen", 14)
print(cost)
\end{verbatim}
เมื่อ \verb|"Khon Kaen"| คือพื้นที่ส่ง (อยู่ในเขตจังหวัดขอนแก่น)
\verb|14| คือน้ำหนักของที่ต้องการส่ง
และฟังก์ชัน \verb|delivery_kk| ทำหน้าที่คำนวณค่าจัดส่ง.
ผลลัพธ์คือ
\verb|218| ซึ่งคือค่าจัดส่ง $50 + 14 \cdot 12 = 218$ บาท.

\textit{คำใบ้} ไพธอนใช้ไวยากรณ์เงื่อนไข ดังนี้
\begin{verbatim}
if cond:
    # if body
    ...
# statement after condition
\end{verbatim}

หากเป็นเงื่อนไขทางเลือก ใช้ไวยากรณ์ดังนี้
\begin{verbatim}
if cond:
    # if body
    ...
else:
    # else body
    ...
# statement after condition
\end{verbatim}

หากเป็นเงื่อนไขทางเลือกหลายทาง ใช้ไวยากรณ์ดังนี้
\begin{verbatim}
if cond:
    # if body
    ...
elif cond:
    # elif body
    ...
else:
    # else body
    ...
# statement after condition
\end{verbatim}


\end{Exercise}

\begin{Exercise}
	\label{prog for RMS}
	จงเขียนโปรแกรมเพื่อคำนวณค่ารากกำลังสองเฉลี่ย (root mean square คำย่อ RMS) โดยรับจำนวนของค่าที่ต้องการนำมาคำนวณ และรับค่าเหล่านั้นทีละค่าจนครบ และคำนวณค่ารากกำลังสองเฉลี่ย เมื่อได้รับค่าต่าง ๆ ครบตามจำนวนแล้ว.

ตัวอย่างโปรแกรม
\begin{verbatim}
Number of values: 4
value 1: 10
value 2: 2
value 3: 0.4
value 4: 3.8
RMS = 5.445181356024793
\end{verbatim}
เมื่อ 4 ในบรรทัดแรก เป็นอินพุตที่ผู้ใช้ระบุจำนวนค่า
และค่า 10 ค่า 2 ค่า 0.4 และ 3.8 เป็นอินพุตที่ผู้ใช้ป้อน
ส่วน \verb|value 1| ไปจนถึง \verb|value 4| เป็นสิ่งที่โปรแกรมพิมพ์ออกไปหน้าจอ
และ 5.445181356024793 เป็นผลลัพธ์การคำนวณ $\sqrt{\frac{10 + 2 + 0.4 + 3.8}{4}} = 5.445181356024793$.

\textit{คำใบ้}
(1) ค่ารากกำลังสองเฉลี่ย $\mathrm{rms}$ คำนวณจาก $rms = \sqrt{\frac{1}{N} \sum_{i=1}^N x_i^2}$ เมื่อ $N$ เป็นจำนวนค่า และ $x_i$ เป็นค่าต่าง ๆ ที่ต้องการนำมาคำนวณ.
(2) มอดูล \verb|math| มีฟังก์ชัน \verb|math.sqrt| เพื่อใช้คำนวณค่าราก.
(3) ตัวอย่างรูปแบบไวยากรณ์ไพธอนสำหรับการวนซ้ำ คือ
\begin{verbatim}
for i in range(num):
    # statement to be repeated
    ...
# statement after for loop
\end{verbatim}
เมื่อ \verb|num| เป็นจำนวนครั้งที่ต้องการวนซ้ำ
และตัวแปร \verb|i| เป็นดัชนีของการวนซ้ำ.
(4) ไพธอนมีวิธีจัดรูปแบบข้อมูลสายอักขระ (string) ได้หลายแบบ 
(4.1) ใช้ตัวดำเนินการ \verb|%| เช่น
\verb|"value %d"%8| ซึ่งจะแสดงผลเป็น \verb|value 8|
หรือ (4.2) ใช้เมท็อด \verb|format| ของข้อมูลสายอักขระ เช่น
\verb|"value {}".format(8)| ซึ่งจะแสดงผลเป็น \verb|value 8| เช่นกัน.
(5) ตัวดำเนินการ \verb|=| เป็นตัวดำเนินการกำหนดค่า (assignment operator)
ซึ่งทำงาน โดย ประเมินค่าจากนิพจน์ที่อยู่ทางซ้ายมือ และนำค่าไปเก็บไว้ในตัวแปรที่อยู่ทางขวา เช่น
\verb|x = 3 + 4| คือ การกำหนดค่าให้ตัวแปร \verb|x| เป็นค่า 7 ซึ่งได้จากการประเมินนิพจน์ \verb|3+4|.
ทำนองเดียวกัน
\verb|x = x + 1| คือ การกำหนดค่าให้ตัวแปร \verb|x| เป็น 
ค่าจากการประเมินนิพจน์ \verb|x + 1|
ดังนั้นหากรันคำสั่ง \verb|x = x + 1| นี้แล้ว ตัวแปร \verb|x| จะมีค่าเพิ่มจากเดิมขึ้นหนึ่ง.


\end{Exercise}

\begin{Exercise}
	\label{prog list prob}
	จงเขียนฟังก์ชัน เพื่อประมาณค่าความน่าจะเป็นของเหตุการณ์ต่าง ๆ จากจำนวนครั้งที่พบ.

ตัวอย่างการเรียกใช้ฟังก์ชัน
\begin{verbatim}
count = [0, 8, 20, 4, 12, 1, 5]
p = est_prob(count)
print(p)
\end{verbatim}
เมื่อ \verb|count| คือ ตัวแปรของข้อมูลชนิดลิสต์ (list) ที่เก็บจำนวนครั้งของเหตุกาณ์ 7 เหตุการณ์ โดย ตัวเลขในแต่ละตำแหน่ง แทนจำนวนครั้งที่พบเหตุการณ์นั้น เช่น เหตุการณ์ที่ 1 ไม่พบเลย เหตุการณ์ที่ 2 พบ 8 ครั้ง. 
ส่วน \verb|est_prob| คือฟังก์ชันที่ประมาณความน่าจะเป็น.
ผลลัพธ์คือ
\begin{verbatim}
[0.0, 0.16, 0.4, 0.08, 0.24, 0.02, 0.1]
\end{verbatim}
ซึ่งหมายถึง
ความน่าจะเป็นที่คำนวณได้ สำหรับเหตุการณ์ต่าง ๆ ตามลำดับ
เช่น เหตุการณ์ที่ 1 มีความน่าจะเป็น เป็น 0
เหตุการณ์ที่ 2 มีความน่าจะเป็น เป็น 0.16.

\textit{คำใบ้}
(1) ความน่าจะเป็น $p_i$ ประมาณได้จาก $p_i = \frac{c_i}{\sum_{j=1}^N c_j}$ เมื่อ $c_i$ คือจำนวนครั้งที่พบเหตุการณ์ $i$ และ $N$ คือจำนวนเหตุการณ์ทั้งหมด.
(2) แต่ละค่าของลิสต์สามารถนำออกมาได้ โดยการใช้ดัชนี เช่น \verb|count[2]| จะได้ค่า 20 ออกมา (ดัชนีแรก เริ่มที่ 0).
(3) ฟังก์ชัน \verb|len| สามารถช่วยนับจำนวนรายการทั้งหมดในลิสต์ได้.
(4) คำสั่ง \verb|for| สามารถทำงานกับลิสต์ได้โดยตรง เช่น 
\begin{verbatim}
for c in count: 
    print(c)
\end{verbatim}
(5) ลิสต์ว่าง สามารถสร้างได้ เช่น \verb|prob = []| กำหนดตัวแปร \verb|prob| ให้มีค่าเป็นลิสต์ว่าง.
(6) ลิสต์ สามารถเพิ่มรายการเข้าไปได้ เช่น \verb|prob.append(0.1)| เป็นการเพิ่มรายการ 0.1 เข้าไปในลิสต์ของตัวแปร \verb|prob|.

\end{Exercise}


\begin{Exercise}
\label{prog dic word freq}
จงเขียนฟังก์ชันที่รับข้อความ และนับความถี่ของคำต่าง ๆ ในข้อความ แล้วส่งผลการนับความถี่ออกมา.

ตัวอย่างการเรียกใช้ฟังก์ชัน
\begin{verbatim}
txt = "Evil is done by oneself; " + \
"by oneself is one defiled. "+ \
"Evil is left undone by oneself; " + \
"by oneself is one cleansed. "

wf = word_freq(txt)
print(wf)
\end{verbatim}
เมื่อ \verb|txt| คือ ตัวแปรที่เก็บข้อความ
และ \verb|word_freq| คือฟังก์ชันที่นับความถี่ของคำ ในข้อความของ \verb|txt|.
ผลลัพธ์คือ
\begin{verbatim}
{'is': 4, 'left': 1, 'done': 1, 'Evil': 2, 'one': 2, 
'cleansed': 1, 'oneself': 4, 'undone': 1, 'defiled': 1, 
'by': 4}
\end{verbatim}
ซึ่งอยู่ในรูปของไพธอนดิกชั่นนารี (dictionary).

\textit{คำใบ้}
(1) ใช้ฟังก์ชันข้างล่าง เพื่อจัดการคำต่าง ๆ ให้เรียบร้อย
\begin{verbatim}
def clean_txt(msg):
    msg = msg.replace('.', ' ')
    msg = msg.replace(';', ' ')
    msg = msg.replace('\n', ' ')
    msg = msg.replace('  ', ' ')

    return msg
\end{verbatim}
(2) เมท็อด \verb|split| ของข้อมูลสายอักขระ 
สามารถช่วยแตกคำต่าง ๆ ออกมาจากข้อความได้สะดวก เช่น
\verb|"Evil is left".split()| จะให้ลิสต์
\verb|['Evil', 'is', 'left']| ออกมา.\\
(3) เมท็อด \verb|strip| 
ช่วยตัดช่องว่างรอบคำออกได้สะดวก.
(4) ดิกชันนารีว่าง สามารถสร้างได้ เช่น \verb|w = {}| จะสร้างดิกชันนารีว่าง ให้กับตัวแปร \verb|w|.
(5) การอ้างอิงรายการของดิกชันนารี จะใช้กุญแจดัชนี ซึ่งเป็นเช่นเดียวกับดัชนีของลิสต์ เพียงแต่ กุญแจดัชนีของดิกชันนารีสามารถใช้เป็นสายอักขระได้ เช่น \verb|w['Evil'] = 1| เป็นการกำหนดค่า $1$ ให้กับรายการที่มีกุญแจดัชนีเป็น \verb|'Evil'| ซึ่งหากยังไม่มีรายการของกุญแจนี้อยู่ ไพธอนจะสร้างขึ้นมาใหม่ แต่หากมีอยู่แล้วค่า 1 ก็จะไปแทนที่ค่าเดิมของรายการนี้.
กลไกนี้ทำให้ดิกชันนารีสะดวกมากกับการใช้นับความถี่คำในลักษณะเช่นนี้.
(6) เช่นเดียวกับตัวแปรเดี่ยว รายการของดัชนีสามารถใช้ในลักษณะการเปลี่ยนแปลงค่าได้ เช่น \verb|w['Evil'] += 1| จะเป็นการเพิ่มค่าของรายการของกุญแจดัชนี \verb|'Evil'| จากเดิม ขึ้นไปหนึ่ง.
\end{Exercise}

\begin{Exercise}
	\label{prog file codon}
    ถอดรหัสดีเอ็นเอ.
%	รหัสพันธุกรรม (genes) ถูกบันทึกไว้ด้วยดีเอ็นเอ (DNA) ซึ่งถูกเซลล์นำไปใช้ในกระบวนการสร้างโปรตีน.
	ดีเอ็นเอประกอบด้วย ฐานนิวคลีโอไทด์ (nucleotide bases) สี่ชนิด ได้แก่ 
	อะดีนีน (adenin ตัวย่อ A) 
	ไซโตซีน (cytosine ตัวย่อ C)
	กัวอานีน (guanine ตัวย่อ G)
	และ ไธมีน (thymine ตัวย่อ T).
	ลำดับของฐานนิวคลีโอไทด์ต่าง ๆ จะเป็นข้อมูลที่เซลล์นำไปใช้ ในกระบวนการสร้างโปรตีน.
	นั่นคือ ลำดับของฐานนิวคลีโอไทด์สามตัว จะบอกชนิดของกรดอะมิโน (amino acid) ที่จะเซลล์จะสร้างเพื่อไปประกอบเป็นโปรตีน (หรืออาจจะเป็นรหัส เพื่อบอกการจบของลำดับสายกรดอะมิโน).
	ชุดของฐานนิวคลีโอไทด์สามตัว จะเรียกว่า โคดอน (codon).
	โคดอน จะถูกอ่านตามลำดับ และจะไม่มีอ่านดีเอ็นเอซ้อนกัน เช่น `AAGGGC' จะอ่านเป็นโคดอนสองชุด คือ  `AAG' และ `GGC'.
	
	จงเขียนฟังก์ชัน เพื่อแปลงจากลำดับดีเอ็นเอ ไปเป็นโปรตีน ซึ่งคือ สายของกรดอะมิโน
	โดย
	ฟังก์ชันรับไฟล์ตารางโคดอน ที่เป็นตารางการแปลงโคดอนไปเป็นกรดอะมิโน
	และรับไฟล์ลำดับดีเอ็นเอ
	แล้วถอดลำดับดีเอ็นเอ ทีละสามฐาน และส่งออกผลที่แปลงออกมาได้.
	
ตัวอย่างไฟล์ตารางโคดอนและตัวอย่างไฟล์ดีเอ็นเอ สามารถดาวน์โหลดได้จาก \url{http://degas.en.kku.ac.th/coewiki/doku.php?id=pr:advbook} (ภายใต้หัวข้อ ข้อมูลประกอบแบบฝึกหัด).
ตัวอย่างการเรียกใช้ฟังก์ชัน
\begin{verbatim}
protein = codon('codons.txt', 'homo_sapiens_mitochondrion.txt')
print(protein)
\end{verbatim}
เมื่อ \verb|codons.txt| คือ ชื่อไฟล์ตารางแปลงโคดอน
\verb|homo_sapiens_mitochondrion.txt| คือชื่อไฟล์ลำดับของดีเอ็นเอ ที่ต้องการแปลง
และ \verb|codon| คือฟังก์ชันที่แปลงโคดอนเป็นโปรตีน.
ผลลัพธ์คือ
\begin{verbatim}
['Lysine', 'Glycine', 'Leucine', 'Alanine', 'stop', 'Leucine', 
'Lysine', 'Tryptophan', 'Leucine', 'Isoleucine', 'Cysteine', 
'Valine', 'Glutamine', 'Leucine', 'Methionine', 'Glutamine', 
'Serine', 'Glycine', 'Valine', 'Leucine', 'Glutamine', 
'Serine', 'Leucine']
\end{verbatim}
ซึ่งอยู่ในรูปของลิสต์.

\textit{คำใบ้}
(1) เปิดดูเนื้อหาในไฟล์ก่อน เพื่อเข้าในรูปแบบของข้อมูลที่เก็บ.
(2) ไพธอนใช้ไวยากรณ์ดังนี้ในการเปิดอ่านไฟล์
\begin{verbatim}
with open('filename', 'r') as f:
   file_content = f.read()
   # ... process file_content
\end{verbatim}
โดย \verb|'filename'| แทนชื่อไฟล์ที่ต้องการเปิดอ่าน (ระบุด้วย \verb|'r'|)
และใช้ตัวแปร \verb|f| เป็นตัวจัดการไฟล์ (file handle).
เมท็อด \verb|read| ใช้อ่านเนื้อหาทั้งหมดของไฟล์ออกมา.
(3) การอ่านดีเอ็นเอมาทีละชุด ชุดละสาม สามารถทำได้หลายวิธี หนึ่งในเทคนิคที่สะดวกคือ
(3.1) ใช้ \verb|range(0, len(dna), 3)| เพื่อหาตำแหน่งเริ่มต้นของแต่ละชุดโคดอน เมื่อ \verb|dna| เป็นข้อมูลสายอักขระที่เก็บลำดับของดีเอ็นเอ
(3.2) ใช้เทคนิคการตัด (slicing) เช่น \verb|dna[i:(i+3)]| เพื่อดึงโคดอนออกมา เมื่อ \verb|i| เป็นตำแหน่งเริ่มของโคดอน.
(4) ถ้าอ่านตารางโคดอนและจัดทำเป็นดิกชันนารีไว้ก่อน จะทำให้การแปลงสะดวกมาก.

\end{Exercise}

%\begin{Exercise}
%	\label{pat motif}
%	Pattern recognition in Bioinformatics --> check out BIOINFO courses	
%\end{Exercise}



\begin{Exercise}
	\label{prog file musical diatonic}


โน้ตดนตรีในระดับเสียงเต็มรูป (diatonic notes) คือ โน้ตดนตรี $7$ ตัวโน๊ตในระดับเสียง (scale).
ตัวโน๊ตทั้งเจ็ดนี้ จะนิยามต่างกันไปสำหรับแต่ละกุญแจเสียง.
ตัวอย่าง เช่น ระดับเสียงหลัก (major scale) ของกุญแจเสียง C (key of C) จะมีโน๊ต C, D, E, F, G, A และ  B.
ระดับเสียงหลักของกุญแจเสียง G (key of G) จะมีโน๊ต G, A, B, C, D, E และ  F\#.
ระดับเสียงหลัก นิยามระดับเสียงเต็มรูป ตามเกณฑ์ดังนี้
 
\begin{tabular}{|l|l|l|}
	\hline 
โน้ตดนตรีในระดับเสียงเต็มรูป &  กุญแจเสียง C & กุญแจเสียง G \\ 
	\hline 
ตัวแรก เป็นโน๊ตของกุญแจเสียง	&  C & G \\ 
	\hline 
ตัวที่สอง เสียงสูงขึ้น $2$ \textit{ครึ่งขั้น} จากตัวแรก	&  D & A \\ 
\hline 
ตัวที่สาม เสียงสูงขึ้น $2$ \textit{ครึ่งขั้น} จากตัวที่สอง	&  E & B\\ 
\hline 
ตัวที่สี่ เสียงสูงขึ้น $1$ \textit{ครึ่งขั้น} จากตัวที่สาม	&  F 
%(ไม่มีก้านดีดดำระหว่าง E และ F) 
& C 
%(ไม่มีก้านดีดดำระหว่าง B และ C) 
\\ 
\hline 
ตัวที่ห้า เสียงสูงขึ้น $2$ \textit{ครึ่งขั้น} จากตัวที่สี่	&  G & D \\ 
\hline 
ตัวที่หก เสียงสูงขึ้น $2$ \textit{ครึ่งขั้น} จากตัวที่ห้า	&  A & E\\ 
\hline 
ตัวที่เจ็ด เสียงสูงขึ้น $2$ \textit{ครึ่งขั้น} จากตัวที่หก	&  B & F\# ($2$ \textit{ครึ่งขั้น} นั่นคือ E $\rightarrow$ F $\rightarrow$ F\#) \\ 
\hline 
\end{tabular} 
\\
หมายเหตุ \textit{ครึ่งขั้น} (half-step) กล่าวโดยง่าย คือ ห่างกัน $1$ ก้านดีดเปียโน 
(รวมก้านดีดทั้งสีขาวและดำ ดูรูป~\ref{fig: ex piano halfstep} ประกอบ).

จงเขียนฟังก์ชัน \texttt{diatonic} ที่รับอาร์กิวเมนต์ \verb|scale_key| สำหรับกุญแจเสียง 
แล้วให้ค่าโน๊ตดนตรีในระดับเสียงเต็มรูปออกมา โดยใช้เลขจำนวนเต็มแทนโน๊ตดนตรีต่าง ๆ ดังนี้
เลข $1$ แทนโน๊ต C,
เลข $2$ แทนโน๊ต C\#,
เลข $3$ แทนโน๊ต D,
เลข $4$ แทนโน๊ต D\# เป็นต้น.

\textit{คำใบ้} มอดุโล (modulo) หรือการหารเอาเศษ ซึ่งใช้ตัวดำเนินการ \verb|%| อาจช่วยให้ทุกอย่างง่ายขึ้น

ตัวอย่างผลการทำงาน
\begin{verbatim}
>>> diatonic(1)
(1, 3, 5, 6, 8, 10, 12)
>>> diatonic(5)
(5, 7, 9, 10, 12, 2, 4)
>>> diatonic(10)
(10, 12, 2, 3, 5, 7, 9)
\end{verbatim}

\end{Exercise}

\begin{figure}[H]
	\begin{center}
		\includegraphics[width=0.75\textwidth]
		{01Intro/Musicscale2.png}
	\end{center}
	\caption[แบบฝึกหัดโน้ตดนตรีในระดับเสียงเต็มรูป]{
		ภาพแถวบนสุดซ้าย แสดงก้านดีดเปียโน พร้อมโน๊ตดนตรี. 
		ภาพซ้ายแถวสอง แถวสาม และแถวสี่ แสดงลูกศรระบุระดับเสียงครึ่งขั้น.
		%	ได้แก่ C\# สูงกว่า C หนึ่งครึ่งขั้น (ภาพในแถวสองจากบน)
		%	D สูงกว่า C\# หนึ่งครึ่งขั้น (ภาพในแถวสามจากบน) เป็นต้น.
		สังเกต E $\rightarrow$ F และ  B $\rightarrow$ C เพิ่มระดับเสียงแค่ครึ่งขั้น (ไม่มีก้านดีดดำอยู่ตรงกลาง).
		ภาพบนทางขวา แสดงก้านดีดเปียโน พร้อมโน๊ตดนตรี และลูกศรแสดงการเพิ่มระดับเสียงทีละครึ่งขั้น จากกุญแจเสียง C
		เปรียบเทียบกับภาพล่างทางขวา ที่แสดงการหาโน๊ตในระดับเสียงเต็มรูป เมื่อใช้กุญแจเสียง G.
	}
	\label{fig: ex piano halfstep}
\end{figure}

%\subsection{แบบฝึกหัดการเสริมความเข้าใจ}




