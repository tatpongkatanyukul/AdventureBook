
\section{ความน่าจะเป็น}
\label{section: Probability}

ในการประยุกต์ใช้การเรียนรู้ของเครื่อง ความไม่แน่นอน (uncertainty) ในข้อมูล. 
ความไม่แน่นอน เป็น ประเด็นหลัก ของการประยุกต์ใช้การเรียนรู้ของเครื่อง เพราะถ้า ทุกๆรูปแบบที่เราต้องการให้เครื่องคอมพิวเตอร์แยกแยะ ทำนาย หรือ ควบคุม เพียงถ้าข้อมูลที่เราทำงานด้วย มีความแน่นอน, งานเราจะเป็นอีกลักษณะ เช่น อาจจะง่ายกว่า ที่ เราจะเขียนเป็นกฎ (เช่น ใช้ rule-based system) แทนที่จะใช้ การเรียนรู้ของเครื่อง.
ความไม่แน่นอน มาได้จากสัญญาณรบกวนที่อาจเกิดในกระบวกการวัดสังเกตุค่า หรือ อาจจะเป็นผลมาจาก ขนาดที่จำกัดของข้อมูลที่เรามี.

ทฤษฎีของความน่าจะเป็น เป็นกรอบความคิด ที่ใช้ประมาณและจัดการกับความไม่แน่นอน และ เบื้องหลัง ก็เป็นพื้นฐานหลัก ของ การเรียนรู้ของเครื่อง ที่เมื่อประกอบกับ decision theory จะช่วยให้เรามีวิธีที่จะจัดการกับข้อมูลที่ไม่สมบูรณ์ หรือ ข้อมูลที่สับสน.

ตอนนี้เราจะทบทวนทฤษฎีความน่าจะเป็นเบื้องต้นก่อน.

BREAK HERE. Mar 9th, 2014.

batch แบบกลุ่ม / เป็นชุด
sequential แบบลำดับ / เชิงลำดับ / โดยลำดับ

likelihood function ฟังก์ชันควรจะเป็น

mean ค่าเฉลี่ย
variance ความแปรปรวน
covariance ความแปรปรวนร่วมเกี่ยว
determinant ดีเทอร์มิแนนต์ / ตัวกำหนด

expectation ค่าคาดหมาย

probability density function (pdf) ฟังก์ชันความหนาแน่นของความน่าจะเป็น

deterministic model ตัวแบบเชิงกำหนด / แบบจำลองเชิงกำหนด 

noise สัญญาณรบกวน

\begin{eqnarray}
   \mathcal{N}(x | \mu, \sigma^2)
   = \frac{1}{\sqrt{2 \pi \sigma^2}} \exp \left\{ -\frac{1}{2 \sigma^2} (x - \mu)^2 \right\}
\label{eq: one-D gaussian distribution}
\end{eqnarray}


\begin{eqnarray}
   \mathcal{N}(\mathbf{x} | \mathbf{\mu}, \Sigma)
   = \frac{1}{(2 \pi)^{D/2}} \frac{1}{|\Sigma|^{1/2}} \exp \left\{ -\frac{1}{2} (\mathbf{x} - \mathbf{\mu})^T \Sigma^{-1} (\mathbf{x} - \mathbf{\mu}) \right\}
\label{eq: gaussian distribution}
\end{eqnarray}
เมื่อ $\mathbf{\mu}$ คือ เวกเตอร์ $D$ มิติของค่าเฉลี่ย (mean),
$\Sigma$ คือ $D \times D$ เมตริกซ์ ของความแปรปรวนร่วมเกี่ยว (covariance),
$|\Sigma|$ คือ ดีเทอร์มิแนนต์ของ $\Sigma$.

TO BE FINISHED, but later. (Jan 14th, 2014)
